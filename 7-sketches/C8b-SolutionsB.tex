\documentclass[7Sketches]{subfiles}
\begin{document}


%======== Section =========%
\section[Solutions for Chapter 4]{Solutions for \cref{chap.codesign}.}

\sol{exc.a_profunctor}{
Suppose we have the preorders
\[
\begin{tikzpicture}
	\node (cat) {category};
	\node[below right=1 of cat] (pos) {preorder};
	\node[below left=1 of cat] (mon) {monoid};
	\draw[->] (pos) -- (cat);
	\draw[->] (mon) -- (cat);
	\node[draw, fit=(cat) (pos) (mon)] (X) {};
	\node[left=0 of X] {$\cat{X}\coloneqq$};
%
	\node[right=2 of pos] (0) {nothing};
	\node at (0|-cat) (1) {this book};
	\draw[->] (0) -- (1);
	\node[draw, fit=(0) (1)] (Y) {};
	\node[left=0 of Y] {$\cat{Y}\coloneqq$};
\end{tikzpicture}
\]
\begin{enumerate}
  \item Draw the Hasse diagram for the preorder $\cat{X}\op\times\cat{Y}$. 
  \item Write down a profunctor $\Lambda\colon\cat{X} \tickar \cat{Y}$ and, reading
  $\Lambda(x,y)=\true$ as ``my aunt can explain a $x$ given $y$,'' give an
  interpretation of the fact that the preimage of $\true$ forms an upper set in
  $\cat{X}\op\times\cat{Y}$. 
  
\end{enumerate}
}{
\begin{enumerate}
\item The Hasse diagram for $\cat{X}\op \times \cat{Y}$ is shown here (ignore the colors): 
\[
\begin{tikzpicture}[font=\scriptsize,xscale =2]
  	\node (a1) at (0,0) {(category, nothing)};
  	\node[blue] (c1) at (-1,1) {(monoid, nothing)};
  	\node (b1) at (1,1) {(preorder, nothing)};
  	\node[blue] (a2) at (0,2) {(category, this book)};
  	\node[blue] (c2) at (-1,3) {(monoid, this book)};
  	\node[blue] (b2) at (1,3) {(preorder, nothing)};
  	\node[draw, fit=(a1) (c1) (b1) (b2)] (A) {};
	\draw[->] (a1) to (c1);
	\draw[->] (a1) to (a2);
	\draw[->] (a1) to (b1);
	\draw[->] (a2) to (c2);
	\draw[->] (a2) to (b2);
	\draw[->] (c1) to (c2);
	\draw[->] (b1) to (b2);
\end{tikzpicture}
\]
\item There is a profunctor $\Lambda\colon \cat{X} \tickar \cat{Y}$, i.e.\ a functor $\cat{X}\op\times\cat{Y}\to\BB$, 
such that, in the picture above, blue is sent to true and black is sent to false, i.e.\
\begin{align*}
  \Lambda(\mbox{monoid, nothing})& = \Lambda(\mbox{monoid, this book}) \\
  &=\Lambda(\mbox{preorder, this book})
  &=\Lambda(\mbox{category, this book})
  =\true \\
  \Lambda(\mbox{preorder, nothing}) &=\Lambda(\mbox{category, nothing})
  =\false.
\end{align*}
\end{enumerate}
The preorder $\cat{X}\op\times\cat{Y}$ describes tasks in decreasing difficulty. For example, (we hope) it is easier for my aunt to explain a monoid given this
book than for her to explain a monoid without this book. The profunctor
$\Lambda$ describes possible states of knowledge for my aunt: she can describe monoids without help, categories with help from the book, etc. It is an upper set
because we assume that if she can do a task, she can also do any easier task.
}

\sol{exc.implies_is_hom}{
Show that $\imp$ as defined in \cref{eqn.implies} indeed satisfies \cref{eqn.implies_internal_hom}.
}{
We've done this one before! We hope you remembered how to do it. If not, see
\cref{exc.Bool_monoidal_closed}.
}

\sol{exc.profunctor_def_alt}{
Show that a $\cat{V}$-profunctor (\cref{def.enriched_profunctor}) is
the same as a function $\Phi\colon\Ob(\cat{X})\times\Ob(\cat{Y})\to V$ such that
for any $x,x'\in\cat{X}$ and $y,y'\in\cat{Y}$ the following inequality holds in
$\cat{V}$: 
\[
  \cat{X}(x',x)\otimes\Phi(x,y)\otimes\cat{Y}(y,y')\leq\Phi(x',y').
  
\]
}{
Recall from \cref{def.monoidal_functor} that a $\cat{V}$-functor $\Phi\colon
\cat{X}\op \times \cat{Y} \to \cat{V}$ is a function $\Phi\colon\Ob(\cat{X}\op \times
\cat{Y}) \to \Ob(\cat{V})$ such that for all $(x,y)$ and $(x',y')$ in
$\cat{X}\op\times \cat{Y}$ we have 
\[
(\cat{X}\op \times \cat{Y})\big((x,y),(x',y')\big) \le
\cat{V}\big(\Phi(x,y),\Phi(x',y')\big).
\]
Using the definitions of product $\cat{V}$-category (\cref{def.enriched_prod})
and opposite $\cat{V}$-category (\cref{def.enriched_op}) on the left hand side,
and using \cref{rem.quantale_enriches_itself}, which describes how we are
viewing the quantale $\cat{V}$ as enriched over itself, on the right hand side,
this unpacks to
\[
\cat{X}(x',x) \otimes \cat{Y}(y,y') \le \Phi(x,y) \multimap \Phi(x',y')
\]
Using symmetry of $\otimes$ and the definition of hom-element, \cref{eqn.monoidal_closed_adj}, we see that
$\Phi$ is a profunctor if and only if
\[
  \cat{X}(x',x)\otimes\Phi(x,y)\otimes\cat{Y}(y,y')\leq\Phi(x',y').
\]
}

\sol{exc.Bool_enriched_is_feas}{
Is it true that a $\Bool$-profunctor, as in \cref{def.enriched_profunctor} is exactly the same as a feasibility relation, as in \cref{def.feasibility_relationship}, once you peel back all the jargon? Or is there some subtle difference?
}{
Yes, since a $\Bool$-functor is exactly the same as a monotone map, the definition of
$\Bool$-profunctor and that of feasibility relation line up perfectly!
}

\sol{exc.feas_matrix}{
We can express $\Phi$ as a matrix where the $(m,n)$th entry is the value of
$\Phi(m,n)\in \BB$. Fill out the $\Bool$-matrix:
\[
\begin{array}{c|ccccc}
	\Phi&a&b&c&d&e\\\hline
  N&\?&\?&\?&\?&\true\\
  E&\true&\?&\?&\?&\?\\
  W&\?&\?&\?&\false&\?\\
  S&\?&\?&\?&\?&\?
\end{array}

\]
}{
The feasibility matrix for $\Phi$ is
\[
\begin{array}{c|ccccc}
	\Phi&a&b&c&d&e\\\hline
  N&\true&\false&\true&\false&\true\\
  E&\true&\true&\true&\true&\true\\
  W&\true&\false&\true&\false&\true\\
  S&\true&\true&\true&\true&\true
\end{array}
\]
}

\sol{exc.distance_matrix_Phi}{
Fill out the $\Cost$-matrix:
\[
\begin{array}{c|ccc}
  \Phi&x&y&z\\\hline
  A&\?&\?&20\\
  B&11&\?&\?\\
  C&\?&17&\?\\
  D&\?&\?&\?
\end{array}

\]
}{
The $\Cost$-matrix for $\Phi$ is
\[
\begin{array}{c|ccc}
  \Phi&x&y&z\\\hline
  A&17&20&20\\
  B&11&14&14\\
  C&14&17&17\\
  D&12&9&15
\end{array}
\]
}

\sol{exc.cost_matrix_mult}{
Calculate $M_X^4*M_\Phi*M_Y^3$, remembering to do matrix multiplication according to the $(\min,+)$-formula for matrix multiplication in the quantale $\Cost$; see \cref{eqn.quantale_matrix_mult}.

Your answer should agree with what you got in \cref{exc.distance_matrix_Phi}; does it?
}{
\begin{align*}
\Phi = M^3_X\ast M_\Phi \ast M_Y^2 
&= 
\begin{pmatrix}
  0&6&3&11\\
  2&0&5&5\\
  5&3&0&8\\
  11&9&6&0
\end{pmatrix}
\begin{pmatrix}
  \infty&\infty&\infty\\
  11&\infty&\infty\\
  \infty&\infty&\infty\\
  \infty&9&\infty
\end{pmatrix}
\begin{pmatrix}
  0&4&3\\
  3&0&6\\
  7&4&0
\end{pmatrix} \\
&=
\begin{pmatrix}
  17&20&\infty\\
  11&14&\infty\\
  14&17&\infty\\
  20&9&\infty
\end{pmatrix}
\begin{pmatrix}
  0&4&3\\
  3&0&6\\
  7&4&0
\end{pmatrix} \\
&=
\begin{pmatrix}
  17&20&20\\
  11&14&14\\
  14&17&17\\
  12&9&15
\end{pmatrix}
\end{align*}
}

\sol{exc.bad_humour}{
In the above diagram, the node (g/n, funny) has no dashed blue arrow emerging
from it. Is this valid? If so, what does it mean?
}{
Yes, this is valid: it just means that the profunctor $\Phi\colon (T \times E)
\tickar \$$ does not relate (good-natured, funny) to any element of $\$$. More
formally, it means that $\Phi((\mbox{good-natured, funny}),p) = \false$ for all
$p \in \$ = \{\mbox{\$100K, \$500K, \$1M}\}$. We can interpret this to mean that
it is not feasible to produce a good-natured, funny movie for any of the cost
options presented---so at least not for less than a million dollars.
}

\sol{exc.compose_Lawv_profs}{
Consider the $\Cost$-profunctors $\Phi\colon\cat{X}\tickar\cat{Y}$ and $\Psi\colon\cat{Y}\tickar\cat{Z}$ shown below:
\[
\begin{tikzpicture}[font=\scriptsize, x=1cm, every label/.style={font=\tiny}]
	\node[label={[above=-5pt]:$A$}] (a) {$\bullet$};
	\node[right=1 of a, label={[above=-5pt]:$B$}] (b) {$\bullet$};
	\node[below=1 of a, label={[below=5pt]:$C$}] (c) {$\bullet$};
	\node[right=1 of c, label={[below=5pt]:$D$}] (d) {$\bullet$};
	\draw[->] (c) to node[above left=-1pt and -1pt] {3} (b);
	\draw[bend right,->] (a) to node[left] {3} (c);
	\draw[bend left,->] (d) to node[below] {4} (c);
	\draw[bend right,->] (b) to node[above] {2} (a);
	\draw[bend left,->] (b) to node[right] {5} (d);
	\node[draw, inner sep=15pt, fit=(a) (b) (c) (d)] (X) {};
	\node[above=0 of X, font=\normalsize] {$\cat{X}\coloneqq$};
%
	\node[right=4 of a, label={[above=-5pt]:$x$}] (x) {$\bullet$};
	\node[below=1 of x, label={[below=5pt]:$y$}] (y) {$\bullet$};
	\node[label={[right=-5pt]:$z$}] at ($(x)!.5!(y)+(1.5cm,0)$) (z) {$\bullet$};
	\draw[bend left,->] (y) to node[left] {3} (x);
	\draw[bend left,->] (x) to node[right] {4} (y);	
	\draw[bend left,->] (x) to node[above] {3} (z);
	\draw[bend left,->] (z) to node[below right=-1pt and -1pt] {4} (y);
	\node[draw, inner sep=15pt, fit=(x) (y) (z.west)] (Y) {};
	\node[above=0 of Y, font=\normalsize] {$\cat{Y}\coloneqq$};
%
	\node [right=3.5 of x, label={[above=-5pt]:$p$}] (p) {$\bullet$};
	\node [right=1 of p, label={[above=-5pt]:$q$}] (q) {$\bullet$};
	\node [below=1 of q, label={[below=5pt]:$r$}]  (r) {$\bullet$};
	\node [left=1 of r, label={[below=5pt]:$s$}]   (s) {$\bullet$};
	\draw[bend left,->] (p) to node[above] {2} (q);
	\draw[bend left,->] (q) to node[right] {2} (r);
	\draw[bend left,->] (r) to node[below] {1} (s);
	\draw[bend left,->] (s) to node[left]  {1} (p);
	\node[draw, inner sep=15pt, fit=(p) (r)] (Z) {};
	\node[above=0 of Z, font=\normalsize] {$\cat{Z}\coloneqq$};
%
\begin{scope}[->, dashed, blue]
	\draw[bend left] (b) to node[above] {11} (x);
	\draw[bend right] (d) to node[below] {9} (y);
	\draw[bend left=20pt] (z) to node[above] {4} (p);
	\draw[bend right=20pt] (z) to node[below] {4} (s);
	\draw[bend right=20pt] (y) to node[below] {0} (r);
\end{scope}
\end{tikzpicture}
\]
Fill in the matrix:
\[
\begin{array}{c|cccc}
  \Phi\cp\Psi&p&q&r&s\\\hline
  A&\?&24&\?&\?\\
  B&\?&\?&\?&\?\\
  C&\?&\?&\?&\?\\
  D&\?&\?&9&\?
\end{array}

\]
}{
There are a number of methods that can be used to get the correct answer. One
way that works well for this example is to search for the shortest paths on the
diagram: it so happens that all the shortest paths go through the bridges from
$D$ to $y$ and $y$ to $r$, so in this case $(\Phi\cp\Psi)(-,-) =
\cat{X}(-,D)+9+\cat{Z}(r,-)$. This gives:
\[
\begin{array}{c|cccc}
  \Phi\cp\Psi&p&q&r&s\\\hline
  A&22&24&20&21\\
  B&16&18&14&15\\
  C&19&21&17&18\\
  D&11&13&9&10
\end{array}
\]
A more methodical way is to use matrix multiplication. Here's one way you might
do the multiplication, using a few tricks.
\begin{align*}
\Phi\cp\Psi &= (M_X^3 \ast M_\Phi \ast M_Y^2) \ast (M_Y^2 \ast M_\Psi \ast M_Z^3) \\
&= M_X^3 \ast M_\Phi \ast M_Y^4 \ast M_\Psi \ast M_Z^3 \\
&= (M_X^3 \ast M_\Phi \ast M_Y^2) \ast M_\Psi \ast M_Z^3 \\
&= \Phi \ast M_\Psi \ast M_Z^3 \\
&=
\begin{pmatrix}
  17&20&20\\
  11&14&14\\
  14&17&17\\
  12&9&15
\end{pmatrix}
\begin{pmatrix}
  \infty&\infty&\infty&\infty\\
  \infty&\infty&0&\infty\\
  4&\infty&\infty&4
\end{pmatrix}
\begin{pmatrix}
  0&2&4&5\\
  4&0&2&3\\
  2&4&0&1\\
  1&3&5&0
\end{pmatrix} \\
&=
\begin{pmatrix}
  17&20&20\\
  11&14&14\\
  14&17&17\\
  12&9&15
\end{pmatrix}
\begin{pmatrix}
  \infty&\infty&\infty&\infty\\
  2&4&0&1\\
  4&6&8&4
\end{pmatrix} \\
&=
\begin{pmatrix}
  22&24&20&21\\
  16&18&14&15\\
  19&21&17&18\\
  11&13&9&10
\end{pmatrix}
\end{align*}
}

\sol{exc.draw_a_bridge}{
Choose a not-too-simple $\Cost$-category $\cat{X}$. Give a bridge-style diagram for the unit profunctor $U_{\cat{X}}\colon\cat{X}\tickar\cat{X}$.
}{
We choose the $\Cost$-category $\cat{X}$ from
\cref{eqn.cities_distances}. The unit profunctor $U_\cat{X}$ on $\cat{X}$ is
described by the bridge diagram
\[
\begin{tikzpicture}[font=\scriptsize, x=1cm, every label/.style={font=\tiny}]
	\node[label={[above=-5pt]:$A$}] (a) {$\bullet$};
	\node[right=1 of a, label={[above=-5pt]:$B$}] (b) {$\bullet$};
	\node[below=1 of a, label={[below=5pt]:$C$}] (c) {$\bullet$};
	\node[right=1 of c, label={[below=5pt]:$D$}] (d) {$\bullet$};
	\draw[->] (c) to node[above left=-1pt and -1pt] {3} (b);
	\draw[bend right,->] (a) to node[left] {3} (c);
	\draw[bend left,->] (d) to node[below] {4} (c);
	\draw[bend right,->] (b) to node[above] {2} (a);
	\draw[bend left,->] (b) to node[right] {5} (d);
	\node[draw, inner sep=15pt, fit=(a) (b) (c) (d)] (X) {};
%
	\node[right=4 of b, label={[above=-5pt]:$A$}] (a1) {$\bullet$};
	\node[right=1 of a1, label={[above=-5pt]:$B$}] (b1) {$\bullet$};
	\node[below=1 of a1, label={[below=5pt]:$C$}] (c1) {$\bullet$};
	\node[right=1 of c1, label={[below=5pt]:$D$}] (d1) {$\bullet$};
	\draw[->] (c1) to node[above left=-1pt and -1pt] {3} (b1);
	\draw[bend right,->] (a1) to node[left] {3} (c1);
	\draw[bend left,->] (d1) to node[below] {4} (c1);
	\draw[bend right,->] (b1) to node[above] {2} (a1);
	\draw[bend left,->] (b1) to node[right] {5} (d1);
	\node[draw, inner sep=15pt, fit=(a1) (b1) (c1) (d1)] (X) {};
%
	\draw[->, dashed, blue, bend left=20] (a) to node[above,pos=.47] {0} (a1);
	\draw[->, dashed, blue, bend left=20] (b) to node[above,pos=.53] {0} (b1);
	\draw[->, dashed, blue, bend right=20] (c) to node[below,pos=.47] {0} (c1);
	\draw[->, dashed, blue, bend right=20] (d) to node[below,pos=.53] {0} (d1);
\end{tikzpicture}
\]
}

\sol{exc.prof_unitality}{
\begin{exercise} \label{exc.prof_unitality}
\begin{enumerate}
  \item Justify each of the four steps $(=, \leq, \leq, =)$ in
  \cref{eqn.direction_rand597}.
  \item In the case $\cat{V}=\Bool$, we can directly show each of the four steps
  in \cref{eqn.direction_rand597} is actually an equality. How? 
  \item Justify each of the three steps $(=,\leq,\leq)$ in \cref{eqn.rand_16749}.
\end{enumerate}
}{
\begin{enumerate}
  \item The first equality is the unitality of $\cat{V}$
  (\cref{def.symm_mon_structure}(b)). The second step uses the monotonicity of
  $\otimes$ (\cref{def.symm_mon_structure}(a)) applied to the inequalities $I \le
  \cat{P}(p,p)$ (the identity law for $\cat{P}$ at $p$,
  \cref{def.cat_enriched_mpos}(a)) and $\Phi(p,q) \le \Phi(p,q)$ (reflexivity of
  preorder $\cat{V}$, \cref{def.preorder}(a)). The third step uses the definition
  of join: for all $x$ and $y$, since any $x \le x$, we have $x \le x \vee y$. The
  final equality is just the definition of profunctor composition,
  \cref{def.composite_profunctor}.
  
  \item Note that in $\Bool$, $I=\true$. Since the identity law at $p$ says $\true
  \le \cat{P}(p,p)$, and $\true$ is the largest element of the preorder $\Bool$, we
  thus have $\cat{P}(p,p)=\true$ for all $p$. This shows that the first inequality
  in \cref{eqn.direction_rand597} is an equality.
  
  The second inequality is more involved. Note that by the above, we can assume
  the left hand side of the inequality is equal to $\Phi(p,q)$. We split into two
  cases. Suppose $\Phi(p,q)= \true$. Then, again since $\true$ is the largest element of
  $\BB$, we must have equality. 
  
  Next, suppose $\Phi(p,q) = \false$. Note that since $\Phi$ is a monotone map
  $\cat{P}\op \times \cat{Q} \to \Bool$, if $p \le p_1$ in $\cat{P}$, then
  $\Phi(p_1,q) \le \Phi(p,q)$ in $\Bool$. Thus if $\cat{P}(p,p_1) = \true$ then
  $\Phi(p_1,q) = \Phi(p,q) =\false$. This implies that for all $p_1 \in \cat{P}$,
  we have either $\cat{P}(p,p_1) =\false$ or $\Phi(p_1,q) = \false$, and hence
  $\bigvee_{p_1\in \cat{P}} \cat{P}(p,p_1) \wedge \Phi(p_1,q) = \bigvee_{p_1 \in
  \cat{P}} \false = \false$.
  
  Thus, in either case, we see that $\Phi(p,q) = \bigvee_{p_1\in \cat{P}}
  \cat{P}(p,p_1) \wedge \Phi(p_1,q)$, as required.
  \item The first equation is unitality in monoidal categories, $v\otimes I=v$ for any $v\in V$. The second is that $I\leq\cat{Q}(q,q)$ by unitality of enriched categories, see \cref{def.cat_enriched_mpos}, together with monotonicity of monoidal product: $v_1\leq v_2$ implies $v\otimes v_1\leq v\otimes v_2$. The third was shown in \cref{exc.profunctor_def_alt}.
\end{enumerate}
}

\sol{exc.prof_associativity}{
Prove \cref{lemma:assoc_serial}. (Hint: remember to use the fact that $\cat{V}$
is skeletal.)
}{
This is very similar to \cref{exc.matrix_mult2}: we exploit the associativity of
$\otimes$. Note, however, we also require $\cat{V}$ to be symmetric monoidal
closed, since this implies the distributivity of $\otimes$ over $\vee$
(\cref{prop.properties_closed_mon_preorders} 2), and $\cat{V}$ to be skeletal,
so we can turn equivalences into equalities.
\begin{align*}
((\Phi\cp \Psi) \cp \Upsilon)(p,s) 
&= \bigvee_{r \in \cat{R}} \bigg(\bigvee_{q \in \cat{Q}} \Phi(p,q) \otimes
\Psi(q,r)\bigg) \otimes \Upsilon(r,s) \\
&= \bigvee_{r \in \cat{R}, q \in \cat{Q}} \Phi(p,q) \otimes
\Psi(q,r) \otimes \Upsilon(r,s) \\
&= \bigvee_{q \in \cat{Q}} \Phi(p,q) \otimes\bigvee_{r \in \cat{R}}
\bigg(\Psi(q,r) \otimes \Upsilon(r,s)\bigg) \\
&= (\Phi\cp(\Psi \cp \Upsilon))(p,s)
\end{align*}
}

\sol{exc.unit_companion}{
Check that the companion $\comp{\id}$ of $\id\colon\cat{P}\to\cat{P}$ really has the formula given in \cref{eqn.unit_profunctor}.
}{
This is very straightforward. We wish to check $\comp{\id}\colon \cat{P} \tickar
\cat{P}$ has the formula $\comp{\id}(p,q) = \cat{P}(p,q)$. By
\cref{def.companion_conjoint}, $\comp\id(p,q) \coloneqq \cat{P}(\id(p),q) =
\cat{P}(p,q)$. So they're the same.
}

\sol{exc.plus_conjoint}{
	Let $+\colon\RR\times\RR\times\RR\to\RR$ be as in \cref{ex.plus_3}. What is its conjoint $\conj{+}$?
}{
The conjoint $\conj{+}\colon \RR \tickar \RR \times \RR \times \RR$ sends
$(a,b,c,d)$ to $\RR(a,b+c+d)$, which is $\true$ if $a \le b+c+d$, and false
otherwise. 
}

\sol{exc.adjoint_comp_conj}{
Let $\cat{V}$ be a skeletal quantale, let $\cat{P}$ and $\cat{Q}$ be $\cat{V}$-categories, and let $F\colon\cat{P}\to\cat{Q}$ and $G\colon\cat{Q}\to\cat{P}$ be $\cat{V}$-functors.
\begin{enumerate}
	\item Show that $F$ and $G$ are $\cat{V}$-adjoints (as in
	\cref{eqn.adjoint_V_funs}) if and only if the companion of the former
	equals the conjoint of the latter: $\comp{F}=\conj{G}$.
	\item Use this to prove that $\comp{\id}=\conj{\id}$, as was stated in \cref{ex.unit_profunctor}.
	

\end{enumerate}
}{
\begin{enumerate}
	\item By \cref{def.companion_conjoint}, $\comp{F}(p,q) = \cat{Q}(F(p),q)$ and
$\conj{G}(p,q) = \cat{Q}(p,G(q))$. Since $\cat{V}$ is skeletal, $F$ and $G$ are
$\cat{V}$ adjoints if and only if $\cat{Q}(F(p),q) = \cat{Q}(p,G(q))$. Thus $F$
and $G$ are adjoints if and only if $\comp{F} = \conj{G}$.
	\item Note that $\id\colon \cat{P} \to \cat{P}$ is $\cat{V}$-adjoint to
	itself, since both sides of \cref{eqn.adjoint_V_funs} then equal
	$\cat{P}(p,q)$. Thus $\comp{\id} = \conj{\id}$.
\end{enumerate}
}

\sol{exc.collage_practice}{
Draw a Hasse diagram for the collage of the profunctor shown here:
\[
\begin{tikzpicture}[font=\scriptsize, x=1cm, every label/.style={font=\tiny}, baseline=(Y)]
	\node[label={[above=-5pt]:$A$}] (a) {$\bullet$};
	\node[right=1 of a, label={[above=-5pt]:$B$}] (b) {$\bullet$};
	\node[below=1 of a, label={[below=5pt]:$C$}] (c) {$\bullet$};
	\node[right=1 of c, label={[below=5pt]:$D$}] (d) {$\bullet$};
	\draw[->] (c) to node[above left=-1pt and -1pt] {3} (b);
	\draw[bend right,->] (a) to node[left] {3} (c);
	\draw[bend left,->] (d) to node[below] {4} (c);
	\draw[bend right,->] (b) to node[above] {2} (a);
	\draw[bend left,->] (b) to node[right] {5} (d);
	\node[draw, inner sep=15pt, fit=(a) (b) (c) (d)] (X) {};
	\node[left=0 of X, font=\normalsize] {$X\coloneqq$};
%
	\node[right=4 of b] (x) {$\LMO{x}$};
	\node[below=1 of x] (y) {$\LMO[under]{y}$};
	\node at ($(x)!.5!(y)+(1.5cm,0)$) (z) {$\LMO{z}$};
	\draw[bend left,->] (y) to node[left] {3} (x);
	\draw[bend left,->] (x) to node[right] {4} (y);	
	\draw[bend left,->] (x) to node[above] {3} (z);
	\draw[bend left,->] (z) to node[below right=-1pt and -1pt] {4} (y);
	\node[draw, inner sep=15pt, fit=(x) (y) (z.west)] (Y) {};
	\node[right=0 of Y, font=\normalsize] {$=:Y$};
%
	\draw[->, dashed, blue, bend left] (b) to node[above] {11} (x);
	\draw[->, dashed, blue, bend right] (d) to node[below] {9} (y);
\end{tikzpicture}
\]
}
{
The Hasse diagram for the collage of the given profunctor is quite simply this:
\[
\begin{tikzpicture}[font=\scriptsize, x=1cm, every label/.style={font=\tiny}]
	\node[label={[above=-5pt]:$A$}] (a) {$\bullet$};
	\node[right=1 of a, label={[above=-5pt]:$B$}] (b) {$\bullet$};
	\node[below=1 of a, label={[below=5pt]:$C$}] (c) {$\bullet$};
	\node[right=1 of c, label={[below=5pt]:$D$}] (d) {$\bullet$};
	\draw[->] (c) to node[above left=-1pt and -1pt] {3} (b);
	\draw[bend right,->] (a) to node[left] {3} (c);
	\draw[bend left,->] (d) to node[below] {4} (c);
	\draw[bend right,->] (b) to node[above] {2} (a);
	\draw[bend left,->] (b) to node[right] {5} (d);
%
	\node[right=2 of b, label={[above=-5pt]:$x$}] (x) {$\bullet$};
	\node[below=1 of x, label={[below=5pt]:$y$}] (y) {$\bullet$};
	\node[label={[above=-5pt]:$z$}] at ($(x)!.5!(y)+(1.5cm,0)$) (z) {$\bullet$};
	\draw[bend left,->] (y) to node[left] {3} (x);
	\draw[bend left,->] (x) to node[right] {4} (y);	
	\draw[bend left,->] (x) to node[above] {3} (z);
	\draw[bend left,->] (z) to node[below right=-1pt and -1pt] {4} (y);
%
	\draw[->, bend left=15pt] (b) to node[above] {11} (x);
	\draw[->, bend right=15pt] (d) to node[below] {9} (y);
	\node[draw, inner sep=15pt, fit=(a) (c) (z)] {};
\end{tikzpicture}
\]
}

\sol{exc.mon_preorder_is_cat}{
Check that monoidal categories generalize monoidal preorders: a monoidal preorder is a
monoidal category $(\cat{P},I,\otimes)$ where, for every $p,q\in\cat{P}$, the
set $\cat{P}(p,q)$ has at most one element.
}{
Since we only have a rough definition, we can only roughly check this: we won't
bother with the notion of well-behaved. Nonetheless, we can still compare
\cref{def.symm_mon_structure} with \cref{rdef.sym_mon_cat}.

First, recall from \cref{subsubsec.pos_free_spectrum} that a preorder is a
category $\cat{P}$ such that for every $p,q \in \cat{P}$, the set $\cat{P}(p,q)$
has at most one element. 

On the surface, all looks promising: both definitions have two constituents and
four properties. In constituent (i), both \cref{def.symm_mon_structure}
and \cref{rdef.sym_mon_cat} call for the same: an element, or object, of the
preorder $\cat{P}$. So far so good. Constituent (ii), however, is where it gets
interesting: \cref{def.symm_mon_structure} calls for merely a function
$\otimes\colon \cat{P} \times \cat{P} \to \cat{P}$, while
\cref{rdef.sym_mon_cat} calls for a \emph{functor}.

Recall from \cref{ex.preorder_functor} that functors between preorders are
exactly monotone maps. So we need for the function $\otimes$ in
\cref{def.symm_mon_structure} to be a monotone map. This is exactly property (a)
of \cref{def.symm_mon_structure}: it says that if $(x_1,x_2) \le (y_1,y_2)$ in
$\cat{P} \otimes \cat{P}$, then we must have $x_1 \otimes x_2 \le y_1 \otimes
y_2$ in $\cat{P}$. So it is also the case that in \cref{def.symm_mon_structure}
that $\otimes$ is a functor.

The remaining properties compare easily, taking the natural isomorphisms to
be equality or equivalence in $\cat{P}$. Indeed, property (b) of
\cref{def.symm_mon_structure} corresponds to \emph{both} properties (a) and (b)
of \cref{rdef.sym_mon_cat}, and then the respective properties (c) and (d)
similarly correspond.
}

\sol{exc.read_string_diag}{
Consider the monoidal category $(\Cat{Set},1,\times)$, together with the diagram
\[
\begin{tikzpicture}[oriented WD, string decoration={}]
	\node[bb={1}{2}] (X11) {$f$};
	\node[bb={2}{2}, below right=of X11] (X12) {$g$};
	\node[bb={2}{1}, above right=of X12] (X13) {$h$};
	\node[bb={0}{0}, fit={($(X11.north west)+(.3,1.5)$) (X12)  ($(X13.east)+(-.3,0)$)}] (Y1) {};
	\begin{scope}[font=\small]
  	\draw[ar] (Y1.west|-X11_in1) to node[above] {$A$} (X11_in1);	
  	\draw[ar] (Y1.west|-X12_in2) to node[above] {$B$} (X12_in2);
  	\draw[ar] (X11_out1) to node[above] {$C$} (X13_in1);
  	\draw[ar] (X11_out2) to node[above=5pt] {$D$} (X12_in1);
  	\draw[ar] (X12_out1) to node[above=5pt] {$E$} (X13_in2);
  	\draw[ar] (X12_out2) to node[above] {$F$} (X12_out2-|Y1.east);
  	\draw[ar] (X13_out1) to node[above] {$G$} (X13_out1-|Y1.east);
	\end{scope}
\end{tikzpicture}
\]
Suppose that $A=B=C=D=F=G=\ZZ$ and $E=\BB=\{\true,\false\}$, 
and suppose that $f_C(a)=|a|$, $f_D(a)=a*5$, $g_E(d,b)=d\leq b$, $g_F(d,b)=d-b$, and $h(c,e)=\tn{if }e\tn{ then }c\tn{ else }1-c$.
\begin{enumerate}
	\item What are $g_E(5,3)$ and $g_F(5,3)$?
	\item What are $g_E(3,5)$ and $g_F(3,5)$?
	\item What is $h(5,\true)$?
	\item What is $h(-5,\true)$?\erase{Error!}
	\item What is $h(-5,\false)$?
\end{enumerate}
The whole diagram now defines a function $A\times B\to G\times F$; call it $q$.
\begin{enumerate}[resume]
	\item What are $q_G(-2,3)$ and $q_F(-2,3)$?
	\item What are $q_G(2,3)$ and $q_F(2,3)$?
	

\end{enumerate}
}{
\begin{enumerate}
	\item $g_E(5,3) = \false$, $g_F(5,3) = 2$.
	\item $g_E(3,5) = \true$, $g_F(3,5) = -2$.
	\item $h(5,\true) = 5$.
	\item $h(-5,\true) = -5$.
	\item $h(-5,\false) = 6$.
	\item $q_G(-2,3) = 2$, $q_F(-2,3) =-13$.
	\item $q_G(2,3) = -1$, $q_F(2,3) =7$.
\end{enumerate}
}

\sol{exc.cat_is_set_enriched}{
Recall from \cref{ex.set_as_mon_cat} that $\cat{V}=(\smset,\{1\},\times)$ is a
symmetric monoidal category. This means we can apply
\cref{def.enriched_in_mon_cat}. Does the (rough) definition roughly agree with
the definition of category given in \cref{def.category}? Or is there a subtle
difference?
}{
Yes, the rough definition roughly agrees: plain old categories are
$\smset$-categories! In detail, we need to compare
\cref{def.enriched_in_mon_cat} when $\cat{V} = (\smset,\{1\},\times)$ with \cref{def.category}.
In both cases, we see that (i) asks for a collection of objects and (ii) asks for,
for all pairs of objects $x,y$, a \emph{set} $\cat{C}(x,y)$ of morphisms.
Moreover, recall that the monoidal unit $I$ is the one element set $\{1\}$. This
means a morphism $\id_x\colon I \to \cat{C}(x,x)$ is a function $\id_x\colon
\{1\} \to \cat{C}(x,x)$. This is the same data as simply an element $\id_x =
\id_x(1) \in \cat{C}(x,x)$; we call this data the identity morphism on $x$.
Finally, a morphism $\cp\colon \cat{C}(x,y) \otimes \cat{C}(y,z) \to
\cat{C}(x,z)$ is a function $\cp \colon \cat{C}(x,y) \times \cat{C}(y,z) \to
\cat{C}(x,z)$; this is exactly the composite required in \cref{def.category}
(iv).

So in both cases the data agrees. In \cref{def.category} we also require this
data to satify two conditions, unitality and associativity. This is what is
meant by the last sentence of \cref{def.enriched_in_mon_cat}.
}

\sol{exc.metric_space_identities}{
What are identity elements in Lawvere metric spaces (that is,
$\Cost$-categories)? How do we interpret this in terms of distances?
}{
An identity element in a $\Cost$-category $\cat{X}$ is a morphism $I \to
\cat{X}(x,x)$ in $\Cost = ([0,\infty], \ge, 0, +)$, and hence the condition that
$0 \ge \cat{X}(x,x)$. This implies that $\cat{X}(x,x) = 0$. In terms of
distances, we interpret this to mean that the distance from any point to itself
is equal to $0$. We think this is a pretty sensible condition for a notion of
distance to obey.
}

\sol{exc.corelations}{
Consider the set $\ord{3}=\{1,2,3\}$.
\begin{enumerate}
	\item Draw a picture of the unit corelation $\varnothing\to\ord{3}\sqcup\ord{3}$.
	\item Draw a picture of the counit corelation $\ord{3}\sqcup\ord{3}\to\varnothing$.
	\item Check that the snake equations \eqref{eqn.yanking} hold. (Since every object is its own dual, you only need to check one of them.)

\end{enumerate}
}{
\begin{enumerate}
\item Here is a picture of the unit corelation $\varnothing \to \ord{3} \sqcup
\ord{3}$, where we have drawn the empty set with an empty dotted rectangle on the
left:
  \[
  \begin{tikzpicture}
	\begin{pgfonlayer}{nodelayer}
		\node [contact, outer sep=5pt] (1a) at (1, 1.25) {};
		\node [contact, outer sep=5pt] (2a) at (1, 0.75) {};
		\node [contact, outer sep=5pt] (3a) at (1, 0.25) {};
		\node [contact, outer sep=5pt] (1b) at (1, -0.25) {};
		\node [contact, outer sep=5pt] (2b) at (1, -0.75) {};
		\node [contact, outer sep=5pt] (3b) at (1, -1.25) {};
		\node [style=none] (11) at (-2.75, -0) {$\varnothing$};
		\node [style=none, right] (12) at (1.75, -0) {$\ord{3} \sqcup \ord{3}$};
		\coordinate (1outt) at ($(1a)!.5!(1b)+(-.7,.3)$);
		\coordinate (1outb) at ($(1a)!.5!(1b)+(-.7,-.3)$);
		\coordinate (1in) at ($(1a)!.5!(1b)-(.4,0)$);
		\coordinate (2outt) at ($(2a)!.5!(2b)+(-.7,.3)$);
		\coordinate (2outb) at ($(2a)!.5!(2b)+(-.7,-.3)$);
		\coordinate (2in) at ($(2a)!.5!(2b)-(.4,0)$);
		\coordinate (3outt) at ($(3a)!.5!(3b)+(-.7,.3)$);
		\coordinate (3outb) at ($(3a)!.5!(3b)+(-.7,-.3)$);
		\coordinate (3in) at ($(3a)!.5!(3b)-(.4,0)$);
	\end{pgfonlayer}
	\begin{pgfonlayer}{edgelayer}
		\draw [dotted] (-2.25,1.5) -- (-2.25,-1.5) -- (-1.75,-1.5) --
		(-1.75,1.5) -- cycle;
		\begin{scope}[rounded corners=5pt, densely dotted, thick]
		\draw[blue!50!black]
   (node cs:name=1a, anchor=south west) --
   (node cs:name=1a, anchor=south east) --
   (node cs:name=1a, anchor=north east) --
   (node cs:name=1a, anchor=north west) --
   (node cs:name=1outt) --
   (node cs:name=1outb) --
   (node cs:name=1b, anchor=south west) --
   (node cs:name=1b, anchor=south east) --
   (node cs:name=1b, anchor=north east) --
   (node cs:name=1b, anchor=north west) --
   (node cs:name=1in) --
   cycle;
		\draw[red!50!black]  
   (node cs:name=2a, anchor=south west) --
   (node cs:name=2a, anchor=south east) --
   (node cs:name=2a, anchor=north east) --
   (node cs:name=2a, anchor=north west) --
   (node cs:name=2outt) --
   (node cs:name=2outb) --
   (node cs:name=2b, anchor=south west) --
   (node cs:name=2b, anchor=south east) --
   (node cs:name=2b, anchor=north east) --
   (node cs:name=2b, anchor=north west) --
   (node cs:name=2in) --
   cycle;
		\draw [green!50!black]
   (node cs:name=3a, anchor=south west) --
   (node cs:name=3a, anchor=south east) --
   (node cs:name=3a, anchor=north east) --
   (node cs:name=3a, anchor=north west) --
   (node cs:name=3outt) --
   (node cs:name=3outb) --
   (node cs:name=3b, anchor=south west) --
   (node cs:name=3b, anchor=south east) --
   (node cs:name=3b, anchor=north east) --
   (node cs:name=3b, anchor=north west) --
   (node cs:name=3in) --
   cycle;
	\end{scope}
	\end{pgfonlayer}
\end{tikzpicture}
\]
\item Here is a picture of the counit corelation $\ord{3} \sqcup
\ord{3} \to \varnothing$:
  \[
  \begin{tikzpicture}[scale=-1]
	\begin{pgfonlayer}{nodelayer}
		\node [contact, outer sep=5pt] (1a) at (1, 1.25) {};
		\node [contact, outer sep=5pt] (2a) at (1, 0.75) {};
		\node [contact, outer sep=5pt] (3a) at (1, 0.25) {};
		\node [contact, outer sep=5pt] (1b) at (1, -0.25) {};
		\node [contact, outer sep=5pt] (2b) at (1, -0.75) {};
		\node [contact, outer sep=5pt] (3b) at (1, -1.25) {};
		\node [style=none] (11) at (-2.75, -0) {$\varnothing$};
		\node [style=none, left] (12) at (1.75, -0) {$\ord{3} \sqcup \ord{3}$};
		\coordinate (1outt) at ($(1a)!.5!(1b)+(-.7,.3)$);
		\coordinate (1outb) at ($(1a)!.5!(1b)+(-.7,-.3)$);
		\coordinate (1in) at ($(1a)!.5!(1b)-(.4,0)$);
		\coordinate (2outt) at ($(2a)!.5!(2b)+(-.7,.3)$);
		\coordinate (2outb) at ($(2a)!.5!(2b)+(-.7,-.3)$);
		\coordinate (2in) at ($(2a)!.5!(2b)-(.4,0)$);
		\coordinate (3outt) at ($(3a)!.5!(3b)+(-.7,.3)$);
		\coordinate (3outb) at ($(3a)!.5!(3b)+(-.7,-.3)$);
		\coordinate (3in) at ($(3a)!.5!(3b)-(.4,0)$);
	\end{pgfonlayer}
	\begin{pgfonlayer}{edgelayer}
		\draw [dotted] (-2.25,1.5) -- (-2.25,-1.5) -- (-1.75,-1.5) --
		(-1.75,1.5) -- cycle;
	\begin{scope}[rounded corners=5pt, densely dotted, thick]
		\draw [blue!50!black]
   (node cs:name=1a, anchor=south east) --
   (node cs:name=1a, anchor=south west) --
   (node cs:name=1a, anchor=north west) --
   (node cs:name=1a, anchor=north east) --
   (node cs:name=1in) --
   (node cs:name=1b, anchor=south east) --
   (node cs:name=1b, anchor=south west) --
   (node cs:name=1b, anchor=north west) --
   (node cs:name=1b, anchor=north east) --
   (node cs:name=1outb) --
   (node cs:name=1outt) --
   cycle;
		\draw  [red!50!black]
   (node cs:name=2a, anchor=south east) --
   (node cs:name=2a, anchor=south west) --
   (node cs:name=2a, anchor=north west) --
   (node cs:name=2a, anchor=north east) --
   (node cs:name=2in) --
   (node cs:name=2b, anchor=south east) --
   (node cs:name=2b, anchor=south west) --
   (node cs:name=2b, anchor=north west) --
   (node cs:name=2b, anchor=north east) --
   (node cs:name=2outb) --
   (node cs:name=2outt) --
   cycle;
		\draw  [green!50!black]
   (node cs:name=3a, anchor=south east) --
   (node cs:name=3a, anchor=south west) --
   (node cs:name=3a, anchor=north west) --
   (node cs:name=3a, anchor=north east) --
   (node cs:name=3in) --
   (node cs:name=3b, anchor=south east) --
   (node cs:name=3b, anchor=south west) --
   (node cs:name=3b, anchor=north west) --
   (node cs:name=3b, anchor=north east) --
   (node cs:name=3outb) --
   (node cs:name=3outt) --
   cycle;
	\end{scope}
	\end{pgfonlayer}
\end{tikzpicture}
\]
\item Here is a picture of the snake equation on the left of \cref{eqn.yanking}.
\[
  \begin{altikz}
	\begin{pgfonlayer}{nodelayer}
		\node [contact, outer sep=5pt] (1cA) at (-5, -1) {};
		\node [contact, outer sep=5pt] (2cA) at (-5, -1.5) {};
		\node [contact, outer sep=5pt] (3cA) at (-5, -2) {};
		%
		\node [contact, outer sep=5pt] (1aB) at (-2, 2) {};
		\node [contact, outer sep=5pt] (2aB) at (-2, 1.5) {};
		\node [contact, outer sep=5pt] (3aB) at (-2, 1) {};
		\node [contact, outer sep=5pt] (1bB) at (-2, 0.5) {};
		\node [contact, outer sep=5pt] (2bB) at (-2, 0) {};
		\node [contact, outer sep=5pt] (3bB) at (-2, -.5) {};
		\node [contact, outer sep=5pt] (1cB) at (-2, -1) {};
		\node [contact, outer sep=5pt] (2cB) at (-2, -1.5) {};
		\node [contact, outer sep=5pt] (3cB) at (-2, -2) {};
		%
		\node [contact, outer sep=5pt] (1aC) at (1, 2) {};
		\node [contact, outer sep=5pt] (2aC) at (1, 1.5) {};
		\node [contact, outer sep=5pt] (3aC) at (1, 1) {};
		%
		\coordinate (1outtB) at ($(1aB)!.5!(1bB)+(-.7,.3)$);
		\coordinate (1outbB) at ($(1aB)!.5!(1bB)+(-.7,-.3)$);
		\coordinate (1inB) at ($(1aB)!.5!(1bB)+(-.4,0)$);
		\coordinate (2outtB) at ($(2aB)!.5!(2bB)+(-.7,.3)$);
		\coordinate (2outbB) at ($(2aB)!.5!(2bB)+(-.7,-.3)$);
		\coordinate (2inB) at ($(2aB)!.5!(2bB)+(-.4,0)$);
		\coordinate (3outtB) at ($(3aB)!.5!(3bB)+(-.7,.3)$);
		\coordinate (3outbB) at ($(3aB)!.5!(3bB)+(-.7,-.3)$);
		\coordinate (3inB) at ($(3aB)!.5!(3bB)+(-.4,0)$);
		%
		\coordinate (1outt) at ($(1bB)!.5!(1cB)+(.7,.3)$);
		\coordinate (1outb) at ($(1bB)!.5!(1cB)+(.7,-.3)$);
		\coordinate (1in) at ($(1bB)!.5!(1cB)+(.4,0)$);
		\coordinate (2outt) at ($(2bB)!.5!(2cB)+(.7,.3)$);
		\coordinate (2outb) at ($(2bB)!.5!(2cB)+(.7,-.3)$);
		\coordinate (2in) at ($(2bB)!.5!(2cB)+(.4,0)$);
		\coordinate (3outt) at ($(3bB)!.5!(3cB)+(.7,.3)$);
		\coordinate (3outb) at ($(3bB)!.5!(3cB)+(.7,-.3)$);
		\coordinate (3in) at ($(3bB)!.5!(3cB)+(.4,0)$);
	\end{pgfonlayer}
	\begin{pgfonlayer}{edgelayer}
	\begin{scope}[rounded corners=5pt, densely dotted, thick]
	% unit
		\draw  [blue!50!black]
   (node cs:name=1aB, anchor=south west) --
   (node cs:name=1aB, anchor=south east) --
   (node cs:name=1aB, anchor=north east) --
   (node cs:name=1aB, anchor=north west) --
   (node cs:name=1outtB) --
   (node cs:name=1outbB) --
   (node cs:name=1bB, anchor=south west) --
   (node cs:name=1bB, anchor=south east) --
   (node cs:name=1bB, anchor=north east) --
   (node cs:name=1bB, anchor=north west) --
   (node cs:name=1inB) --
   cycle;
		\draw  [red!50!black]  
   (node cs:name=2aB, anchor=south west) --
   (node cs:name=2aB, anchor=south east) --
   (node cs:name=2aB, anchor=north east) --
   (node cs:name=2aB, anchor=north west) --
   (node cs:name=2outtB) --
   (node cs:name=2outbB) --
   (node cs:name=2bB, anchor=south west) --
   (node cs:name=2bB, anchor=south east) --
   (node cs:name=2bB, anchor=north east) --
   (node cs:name=2bB, anchor=north west) --
   (node cs:name=2inB) --
   cycle;
		\draw  [green!50!black]
   (node cs:name=3aB, anchor=south west) --
   (node cs:name=3aB, anchor=south east) --
   (node cs:name=3aB, anchor=north east) --
   (node cs:name=3aB, anchor=north west) --
   (node cs:name=3outtB) --
   (node cs:name=3outbB) --
   (node cs:name=3bB, anchor=south west) --
   (node cs:name=3bB, anchor=south east) --
   (node cs:name=3bB, anchor=north east) --
   (node cs:name=3bB, anchor=north west) --
   (node cs:name=3inB) --
   cycle;
   % counit
		\draw  [blue!50!black]
   (node cs:name=1bB, anchor=south east) --
   (node cs:name=1bB, anchor=south west) --
   (node cs:name=1bB, anchor=north west) --
   (node cs:name=1bB, anchor=north east) --
   (node cs:name=1outt) --
   (node cs:name=1outb) --
   (node cs:name=1cB, anchor=south east) --
   (node cs:name=1cB, anchor=south west) --
   (node cs:name=1cB, anchor=north west) --
   (node cs:name=1cB, anchor=north east) --
   (node cs:name=1in) --
   cycle;
		\draw  [red!50!black]  
   (node cs:name=2bB, anchor=south east) --
   (node cs:name=2bB, anchor=south west) --
   (node cs:name=2bB, anchor=north west) --
   (node cs:name=2bB, anchor=north east) --
   (node cs:name=2outt) --
   (node cs:name=2outb) --
   (node cs:name=2cB, anchor=south east) --
   (node cs:name=2cB, anchor=south west) --
   (node cs:name=2cB, anchor=north west) --
   (node cs:name=2cB, anchor=north east) --
   (node cs:name=2in) --
   cycle;
		\draw  [green!50!black]
   (node cs:name=3bB, anchor=south east) --
   (node cs:name=3bB, anchor=south west) --
   (node cs:name=3bB, anchor=north west) --
   (node cs:name=3bB, anchor=north east) --
   (node cs:name=3outt) --
   (node cs:name=3outb) --
   (node cs:name=3cB, anchor=south east) --
   (node cs:name=3cB, anchor=south west) --
   (node cs:name=3cB, anchor=north west) --
   (node cs:name=3cB, anchor=north east) --
   (node cs:name=3in) --
   cycle;
   % identities
		\draw  [blue!50!black]
   (node cs:name=1cA, anchor=south west) --
   (node cs:name=1cB, anchor=south east) --
   (node cs:name=1cB, anchor=north east) --
   (node cs:name=1cA, anchor=north west) --
   cycle;
		\draw  [red!50!black]  
   (node cs:name=2cA, anchor=south west) --
   (node cs:name=2cB, anchor=south east) --
   (node cs:name=2cB, anchor=north east) --
   (node cs:name=2cA, anchor=north west) --
   cycle;
		\draw  [green!50!black]
   (node cs:name=3cA, anchor=south west) --
   (node cs:name=3cB, anchor=south east) --
   (node cs:name=3cB, anchor=north east) --
   (node cs:name=3cA, anchor=north west) --
   cycle;
		\draw  [blue!50!black]
   (node cs:name=1aB, anchor=south west) --
   (node cs:name=1aC, anchor=south east) --
   (node cs:name=1aC, anchor=north east) --
   (node cs:name=1aB, anchor=north west) --
   cycle;
		\draw  [red!50!black]  
   (node cs:name=2aB, anchor=south west) --
   (node cs:name=2aC, anchor=south east) --
   (node cs:name=2aC, anchor=north east) --
   (node cs:name=2aB, anchor=north west) --
   cycle;
		\draw  [green!50!black]
   (node cs:name=3aB, anchor=south west) --
   (node cs:name=3aC, anchor=south east) --
   (node cs:name=3aC, anchor=north east) --
   (node cs:name=3aB, anchor=north west) --
   cycle;
	\end{scope}
	\end{pgfonlayer}
\end{altikz}
\quad
=
\quad
\begin{altikz}
	\begin{pgfonlayer}{nodelayer}
		\node [contact, outer sep=5pt] (1a) at (-2, 1) {};
		\node [contact, outer sep=5pt] (2a) at (-2, 0.5) {};
		\node [contact, outer sep=5pt] (3a) at (-2, 0) {};
		\node [contact, outer sep=5pt] (1b) at (1, 1) {};
		\node [contact, outer sep=5pt] (2b) at (1, 0.5) {};
		\node [contact, outer sep=5pt] (3b) at (1, 0) {};
%		\node [style=none] (11) at (-2.75, 0.5) {$\ord{3}$};
%		\node [style=none] (12) at (1.75, 0.5) {$\ord{3}$};
	\end{pgfonlayer}
	\begin{pgfonlayer}{edgelayer}
	\begin{scope}[rounded corners=5pt, densely dotted, thick]
		\draw  [blue!50!black]
   (node cs:name=1a, anchor=south west) --
   (node cs:name=1b, anchor=south east) --
   (node cs:name=1b, anchor=north east) --
   (node cs:name=1a, anchor=north west) --
   cycle;
		\draw  [red!50!black]  
   (node cs:name=2a, anchor=south west) --
   (node cs:name=2b, anchor=south east) --
   (node cs:name=2b, anchor=north east) --
   (node cs:name=2a, anchor=north west) --
   cycle;
		\draw  [green!50!black]  
   (node cs:name=3a, anchor=south west) --
   (node cs:name=3b, anchor=south east) --
   (node cs:name=3b, anchor=north east) --
   (node cs:name=3a, anchor=north west) --
   cycle;
	\end{scope}
	\end{pgfonlayer}
\end{altikz}
\]
\end{enumerate}
}

\sol{exc.explain_monoidal_prod_feas}{
Interpret the monoidal products in $\Prof_{\Bool}$ in terms of feasibility. That is, preorders represent resources ordered by availability ($x\leq x'$ means that $x$ is available given $x'$) and a profunctor is a feasibility relation. Explain why $\cat{X}\times\cat{Y}$ makes sense as the monoidal product of resource preorders $\cat{X}$ and $\cat{Y}$ and why $\Phi\times\Psi$ makes sense as the monoidal product of feasibility relations $\Phi$ and $\Psi$.
}
{
Given two resource preorders $\cat{X}$ and $\cat{Y}$, the preorder $\cat{X}\times\cat{Y}$ represents the set of all pairs of resources, $x\in \cat{X}$ and $y\in \cat{Y}$, with $(x,y)\leq(x',y')$ iff $x\leq x'$ and $y\leq y'$. That is, if $x$ is available given $x'$ and $y$ is available given $y'$, then $(x,y)$ is available given $(x',y')$.

Given two profunctors $\Phi\colon\cat{X}_1\tickar\cat{X}_2$ and $\Psi\colon\cat{Y}_1\tickar\cat{Y}_2$, the profunctor $\Phi\times\Psi$ represents their conjunction, i.e.\ AND. In other words, if $y_1$ can be obtained given $x_1$ AND $y_2$ can be obtained given $x_2$, then $(y_1,y_2)$ can be obtained given $(x_1,x_2)$.
}

\sol{exc.prof_monoidal_unit}{
In order for $\Cat{1}$ to be a monoidal unit, there are supposed to be
isomorphisms $\cat{X}\times\Cat{1}\tickar\cat{X}$ and
$\Cat{1}\times\cat{X}\tickar\cat{X}$, for any $\cat{V}$-category $\cat{X}$. What
are they?
}{
The profunctor $\cat{X} \times \Cat{1} \tickar \cat{X}$ defined by the functor
$\alpha\colon (\cat{X} \times \Cat{1})\op \times \cat{X} \to \cat{V}$ that maps
$\alpha((x,1),y)\coloneqq\cat{X}(x,y)$ is an isomorphism. It has inverse
$\alpha\inv\colon \cat{X} \tickar \cat{X} \times \Cat{1}$ defined by
$\alpha\inv(x,(y,1))\coloneqq\cat{X}(x,y)$. To see that $\alpha\inv \cp \alpha =
\Unit{\cat{X}}$, note first that the unit law for $\cat{X}$ at $z$ and the definition
of join imply
\[
\cat{X}(x,z) = \cat{X}(x,z) \otimes I \le \cat{X}(x,z) \otimes \cat{X}(z,z) \le
\bigvee_{y \in \cat{X}} \cat{X}(x,y) \otimes \cat{X}(y,z),
\]
while composition says $\cat{X}(x,y) \otimes \cat{X}(y,z) \le \cat{X}(x,z)$ and
hence
\[
\bigvee_{y \in \cat{X}} \cat{X}(x,y) \otimes \cat{X}(y,z) \le \bigvee_{y \in
\cat{X}} \cat{X}(x,z) = \cat{X}(x,z).
\]
Thus, unpacking the definition of composition of profunctor, we have
\[
(\alpha\inv\cp \alpha) (x,z) 
= \bigvee_{(y,1) \in \cat{X} \times \Cat{1}} \alpha(x,(y,1)) \otimes
\alpha\inv((y,1),z) 
= \bigvee_{y \in \cat{X}} \cat{X}(x,y) \otimes \cat{X}(y,z) = \cat{X}(x,z).
\]
Similarly we can show $\alpha\cp\alpha\inv = \Unit{\cat{X} \times \Cat{1}}$, and
hence that $\alpha$ is an isomorphism $\cat{X} \times \Cat{1} \tickar \cat{X}$.

Moreover, we can similarly show that $\beta((1,x),y) \coloneqq \cat{X}(x,y)$ defines an
isomorphism $\beta\colon \Cat{1} \times \cat{X} \tickar \cat{X}$.
}

\sol{exc.prof_duals}{
Check these proposed units and counits do indeed obey the snake equations
\cref{eqn.yanking}.
}{
We check the first snake equation, the one on the left hand side of
\cref{eqn.yanking}. The proof of the one on the right hand side is analogous.

We must show that the composite $\Phi$ of profunctors
\[
\cat{X} \xrightarrow{\alpha\inv} \cat{X} \times \Cat{1} \xrightarrow{\Unit{\cat{X}}
\times \eta_{\cat{X}}} \cat{X} \times \cat{X}\op \times \cat{X}
\xrightarrow{\epsilon_{\cat{X}} \times \Unit{\cat{X}}} \Cat{1} \times \cat{X}
\xrightarrow{\alpha} \cat{X}
\]
is itself the identity (ie. the unit profunctor on $\cat{X}$), where $\alpha$
and $\alpha\inv$ are the isomorphisms defined in the solution to
\cref{exc.prof_monoidal_unit} above. 

Freely using the distributivity of $\otimes$ over $\vee$, the value $\Phi(x,y)$ of this
composite at $(x,y) \in \cat{X}\op \times \cat{X}$ is given by
\begin{align*}
&\bigvee_{a,b,c,d,e \in \cat{X}} 
	\begin{multlined}[t][11cm]
  \alpha\inv(x,(a,1)) \otimes
  (\Unit{\cat{X}}\times \eta_{\cat{X}})((a,1),(b,c,d))\\
  \otimes (\epsilon_{\cat{X}} \times \Unit{\cat{X}})((b,c,d),(1,e)) \otimes
  \alpha((1,e),y)
  \end{multlined}
\\
=&\bigvee_{a,b,c,d,e \in \cat{X}} 
	\begin{multlined}[t][11cm]
  \alpha\inv(x,(a,1)) \otimes
  \Unit{\cat{X}}(a,b) \otimes \eta_{\cat{X}}(1,c,d) \\
  \otimes
  \epsilon_{\cat{X}}(b,c,1) \otimes \Unit{\cat{X}}(d,e) \otimes
  \alpha((1,e),y)
  \end{multlined}
\\
=& \bigvee_{a,b,c,d,e \in \cat{X}} \cat{X}(x,a) \otimes \cat{X}(a,b) \otimes
\cat{X}(c,d) \otimes \cat{X}(b,c) \otimes \cat{X}(d,e) \otimes \cat{X}(e,y) \\
=&\quad \cat{X}(x,y)
\end{align*}
where in the final step we repeatedly use the argument the
\cref{lemma:unital_serial} that shows that composing with the unit profunctor
$\Unit{\cat{X}}(a,b) = \cat{X}(a,b)$ is the identity.

This shows that $\Phi(x,y)$ is the identity profunctor, and hence shows the
first snake equation holds. Again, checking the other snake equation is
analogous.
}

\finishSolutionChapter
%======== Section =========%
\section[Solutions for Chapter 5]{Solutions for \cref{chap.SFGs}.}


\sol{exc.finset_as_prop}{
In \cref{ex.FinSet_Prop} we said that the identities, symmetries, and
composition rule in $\Cat{FinSet}$ ``are obvious.'' In math lingo, this just
means ``we trust that the reader can figure them out, if she spends the time tracking
down the definitions and fitting them together.''
\begin{enumerate}
	\item Draw a morphism $f\colon 3\to 2$ and a morphism $g\colon 2\to 4$ in $\finset$.
	\item Draw $f+g$.
	\item What is the composition rule for morphisms $f\colon m\to n$ and $g\colon n\to p$ in $\Cat{FinSet}$?
	\item What are the identities in $\Cat{FinSet}$? Draw some.
	\item Choose $m,n \in \NN$, and draw the symmetric map $\sigma_{m,n}$
	in $\Cat{FinSet}$?
\end{enumerate}
}{
\begin{enumerate}
	\item Below we draw a morphism $f\colon 3\to 2$ and a morphism $g\colon 2\to 4$ in $\finset$:
	\[
	\begin{tikzpicture}
		\boxofbullets{3}{(0,0)}{B1};
			\node[left=0.5 of box_B1] (lab_B1) {3};
		\boxofbullets{2}{(0,-1.5)}{B2};
			\node at (lab_B1|-box_B2) (lab_B2) {2};
		\boxofbullets{4}{(4,-1.5)}{C2};
			\node[right=0.5 of box_C2] (lab_C2) {4};
		\boxofbullets{2}{(4,0)}{C1};
			\node at (lab_C2|-box_C1) (lab_C1) {2};
		\draw[->] (lab_B1) to node[left] {$f$} (lab_B2);
		\draw[->] (lab_C1) to node[right] {$g$} (lab_C2);
		\begin{scope}[short=0pt, mapsto]
  		\draw (pt_B1_1) -- (pt_B2_1);
  		\draw (pt_B1_2) -- (pt_B2_1);
  		\draw (pt_B1_3) -- (pt_B2_2);
  		\draw (pt_C1_1) -- (pt_C2_3);
  		\draw (pt_C1_2) -- (pt_C2_3);
		\end{scope}
	\end{tikzpicture}
	\]
	\item Here is a picture of $f+g$
	\[
	\begin{tikzpicture}
		\boxofbullets{5}{(0,0)}{B1};
			\node[left=0.5 of box_B1] (lab_B1) {3+2};
		\boxofbullets{6}{(0,-1.5)}{B2};
			\node at (lab_B1|-box_B2) (lab_B2) {2+4};
		\draw[->] (lab_B1) to node[left] {$f+g$} (lab_B2);
		\begin{scope}[short=0pt, mapsto]
  		\draw (pt_B1_1) -- (pt_B2_1);
  		\draw (pt_B1_2) -- (pt_B2_1);
  		\draw (pt_B1_3) -- (pt_B2_2);
  		\draw (pt_B1_4) -- (pt_B2_5);
  		\draw (pt_B1_5) -- (pt_B2_5);
		\end{scope}
	\end{tikzpicture}
	\]
	\item The composite of morphisms $f\colon m\to n$ and $g\colon n\to p$ in $\Cat{FinSet}$ is the function $(f\cp g)\colon m\to p$ given by $(f\cp g)(i)=g(f(i))$ for all $1\leq i\leq m$.
	\item The identity $\id_m\colon m\to m$ is given by $\id_m(i)=i$ for all $1\leq i\leq m$. Here is a picture of $\id_2$ and $\id_8$:
	\[
	\begin{tikzpicture}
		\boxofbullets{2}{(0,0)}{B1};
			\node[left=0.5 of box_B1] (lab_B1) {2};
		\boxofbullets{2}{(0,-1.5)}{B2};
			\node[left=0.5 of box_B2] (lab_B2) {2};
		\boxofbullets{8}{(6,0)}{C1};
			\node[right=0.5 of box_C1] (lab_C1) {8};
		\boxofbullets{8}{(6,-1.5)}{C2};
			\node[right=0.5 of box_C2] (lab_C2) {8};
		\draw[->] (lab_B1) to node[left] {$\id_2$} (lab_B2);
		\draw[->] (lab_C1) to node[right] {$\id_8$} (lab_C2);
		\begin{scope}[short=0pt, mapsto]
  		\foreach \i in {1,2} {
  			\draw[mapsto] (pt_B1_\i) -- (pt_B2_\i);
  		}
  		\foreach \i in {1,...,8} {
  			\draw (pt_C1_\i) -- (pt_C2_\i);
  		}
		\end{scope}
	\end{tikzpicture}
	\]
	\item Here is a picture of the symmetry $\sigma_{3,5}\colon 8\to 8$:
	\[
	\begin{tikzpicture}
		\boxofbullets{8}{(6,0)}{C1};
			\node[right=0.5 of box_C1] (lab_C1) {8};
		\boxofbullets{8}{(6,-1.5)}{C2};
			\node[right=0.5 of box_C2] (lab_C2) {8};
		\draw[->] (lab_C1) to node[right] {$\sigma_{3,5}$} (lab_C2);
		\begin{scope}[short=0pt, mapsto]
  		\foreach \i/\j in {1/6,2/7,3/8} {
  			\draw (pt_C1_\i.south) -- (pt_C2_\j.north);
  		}
  		\foreach \i/\j in {4/1, 5/2, 6/3, 7/4, 8/5} {
  			\draw (pt_C1_\i.south) -- (pt_C2_\j.north);
  		}
		\end{scope}
	\end{tikzpicture}
	\]
\end{enumerate}
}


\sol{exc.posetal_prop}{
  A posetal prop is a prop that is also a poset. That is, a posetal prop is a
  symmetric monoidal preorder of the form $(\NN,\preceq)$, for some poset relation
  $\preceq$ on $\NN$, where the monoidal product on objects is addition. We've
  spent a lot of time discussing order structures on the natural numbers. Give
  three examples of a posetal prop.
}{
We need to give examples posetal props, i.e.\ each will be a poset whose set of objects is $\nn$, whose order is denoted $m\preceq n$, and with the property that whenever $m_1\preceq n_1$ and $m_2\preceq n_2$ hold then $m_1+m_2\preceq n_1+n_2$ does too.

The question only asks for three, but we will additionally give a quasi-example and a non-example. 
\begin{enumerate}
	\item Take $\preceq$ to be the discrete order: $m\preceq n$ iff $m=n$.
	\item Take $\preceq$ to be the usual order, $m\preceq n$ iff there exists $d\in\nn$ with $d+m=n$.
	\item Take $\preceq$ to be the reverse of the usual order, $m\preceq n$ iff there exists $d\in\nn$ with $m=n+d$.
	\item Take $\preceq$ to be the co-discrete order $m\preceq n$ for all $m,n$. Some may object that this is a preorder, not a poset, so we call it a quasi-example.
	\item (Non-example.) Take $\preceq$ to be the division order, $m\preceq n$ iff there exists $q\in\nn$ with $m*q=d$. This is a perfectly good poset, but it does not satisfy the monotonicity property: we have $2\preceq 4$ and $3\preceq 3$ but not $5\preceq^?7$.
\end{enumerate}
}

\sol{exc.prop_practice}{
Choose one of \cref{ex.function_bijection,ex.corelation,ex.relation} and explicitly provide the five aspects of props discussed below \cref{def.prop}.
}
{
\begin{description}
	\item[\cref{ex.function_bijection}:] The prop $\Cat{Bij}$ has
	\begin{enumerate}
		\item $\Cat{Bij}(m,n)\coloneqq\{f\colon\ord{m}\to\ord{n}\mid f\text{ is a bijection}\}$. Note that $\Cat{Bij}(m,n)=\varnothing$ if $m\neq n$ and it has $n!$ elements if $m=n$.
		\item The identity map $n\to n$ is the bijection $\ord{n}\to\ord{n}$ sending $i\mapsto i$.
		\item The symmetry map $m+n\to n+m$ is the bijection $\sigma_{m,n}\colon\ord{m+n}\to\ord{n+m}$ given by
		\[
  		\sigma_{m,n}(i)\coloneqq
  		\begin{cases}
  			i+n&\tn{ if }i\leq m\\
				i-m&\tn{ if }m+1\leq i
  		\end{cases}
		\]
		\item Composition of bijections $m\to n$ and $n\to p$ is just their composition as functions, which is again a bijection.
		\item Given bijections $f\colon m\to m'$ and $g\colon n\to n'$, their monoidal product $(f+g)\colon (m+n)\to(m'+n')$ is given by
		\[
			(f+g)(i)\coloneqq
			\begin{cases}
				f(i)&\tn{ if }i\leq m\\
				g(i-m)&\tn{ if }m+1\leq i
			\end{cases}
		\]
	\end{enumerate}
	\item[\cref{ex.corelation}:] The prop $\Cat{Corel}$ has
	\begin{enumerate}
		\item $\Cat{Corel}(m,n)$ is the set of equivalence relations on $\ord{m+n}$.
		\item The identity map $n\to n$ is the smallest equivalence relation, which is the smallest reflexive relation, i.e.\ where $i\sim j$ iff $i=j$.
		\item The symmetry map $\sigma_{m,n}$, as an equivalence relation on $\ord{m+n+n+m}$ is ``the obvious thing,'' namely ``equating corresponding $m$'s together and also equating corresponding $n$'s together.'' To be pedantic, $i\sim j$ iff either
		\begin{itemize}
			\item $|i-j|=m+n+n$, or
			\item $m+1\leq i\leq m+n+n$ and $m+1\leq j\leq m+n+n$ and $|i-j|=n$.
		\end{itemize}
		\item Composition of an equivalence relation $\sim$ on $\ord{m+n}$ and an equivalence relation $\dot\sim$ on $\ord{n+p}$ is the equivalence relation $\simeq$ on $\ord{m+p}$ given by $i\simeq k$ iff there exists $j\in\ord{n}$ with $i\sim j$ and $j\dot\sim k$.
		\item Given equivalence relations $\sim$ on $\ord{m+n}$ and $\sim'$ on $\ord{m'+n'}$, we need an equivalence relation $(\sim+\sim')$ on $\ord{m+n+m'+n'}$. We take it to be ``the obvious thing,'' namely ``using $\sim$ on the unprimed stuff and using $\sim'$ on the primed stuff, with no other interaction.'' To be pedantic, $i\sim j$ iff either
		\begin{itemize}
			\item $i\leq m+n$ and $j\leq m+n$ and $i\sim j$, or
			\item $m+n+1\leq i$ and $m+n+1\leq j$ and $i\sim'j$.
		\end{itemize}
	\end{enumerate}
	\item[\cref{ex.relation}:] The prop $\Cat{Rel}$ has
	\begin{enumerate}
		\item $\Cat{Rel}(m,n)$ is the set of relations on the set $\ord{m}\times\ord{n}$, i.e.\ the set of subsets of $\ord{m}\times\ord{n}$, i.e.\ its powerset.
		\item The identity map $n\to n$ is the subset $\{(i,j)\in\ord{n}\times\ord{n}\mid i=j\}$.
		\item The symmetry map $m+n\to n+m$ is the subset of pairs $(i,j)\in(\ord{m+n})\times(\ord{n+m})$ such that either
		\begin{itemize}
			\item $i\leq m$ and $m+1\leq j$ and $i+m=j$, or
			\item $m+1\leq i$ and $j\leq m$ and $j+m=i$.
		\end{itemize}
		\item Composition of relations is as in \cref{ex.relation}.
		\item Given a relation $R\ss\ord{m}\times\ord{n}$ and a relation $R'\ss\ord{m'}\times\ord{n'}$, we need a relation $(R+R')\ss\ord{m+m'}\times\ord{n+n'}$. As stated in the example (footnote), this can be given by a universal property: The monoidal product $R_1+R_2$ of relations $R_1\ss \ord{m_1}\times \ord{n_1}$ and $R_2\ss \ord{m_2}\times \ord{n_2}$ is given by $R_1\sqcup R_2\ss(\ord{m_1}\times \ord{n_1})\sqcup(\ord{m_2}\times \ord{n_2})\ss(\ord{m_1}\sqcup \ord{m_2})\times(\ord{n_1}\sqcup \ord{n_2})$.
	\end{enumerate}
\end{description}
}

\sol{exc.port_graph_comp}{
Describe how port graph composition looks, with respect to the visual
representation of \cref{ex.a_port_graph}, and give a nontrivial example.
}{
Composition of an $(m,n)$-port graph $G$ and an $(n,p)$-port graph $H$ looks visually like sticking them end to end, connecting the wires in order, removing the two outer boxes, and adding a new outer box.

For example, suppose we want to compose the following in the order shown:
\[
\begin{tikzpicture}[oriented WD, bbx=1.2cm, bb min width=.5cm, bb port length=2pt, bb port sep=.75, baseline=(X2.north)]
	\node[bb={1}{3}] (X1) {$a$};
	\node[bb={3}{3}, below right=-.5 and 1 of X1] (X2) {$b$};
	\node[bb={2}{1}, above right=-.5 and 1 of X2] (X3) {$c$};
	\node[bb={0}{0}, fit={($(X1.north west)+(.3,.0)$) (X2)  ($(X3.east)+(-.3,0)$)}] (X) {};
	\node[coordinate] at (X.west|-X1_in1) (X_in1) {};
	\node[coordinate] at (X.west|-X2_in3) (X_in2) {};
	\node[coordinate] at (X3_out1-|X.east) (X_out1) {};
	\node[coordinate] at (X2_out2-|X.east) (X_out2) {};
	\node[coordinate] at (X2_out3-|X.east) (X_out3) {};
	\draw (X_in1) to (X1_in1);	
	\draw (X_in2) to (X2_in3);
	\draw (X1_out1) to (X3_in1);
	\draw (X1_out2) to (X2_in2);
	\draw (X1_out3) to (X2_in1);
	\draw (X2_out1) to (X3_in2);
	\draw (X3_out1) to (X_out1);
	\draw (X2_out2) to (X_out2);
	\draw (X2_out3) to (X_out3);
%
	\node[bb={1}{1}, right=3 of X3] (Y1) {$d$};
	\node[bb={2}{1}, below right=-1 and .5 of Y1] (Y2) {$e$};
	\node[bb={0}{0}, fit={(Y1) ($(Y2.south east)+(0,-1)$)}] (Y) {};
	\node[coordinate] at (Y.west|-Y1_in1) (Y_in1) {};
	\node[coordinate] at (Y.west|-Y2_in2) (Y_in2) {};
	\node[coordinate] at ($(Y_in2)+(0,-1.5)$) (Y_in3) {};
	\node[coordinate] at (Y.east|-Y2_out1) (Y_out1) {};
	\node[coordinate] at (Y.east|-Y_in3) (Y_out2) {};
	\draw (Y_in1) to (Y1_in1);
	\draw (Y_in2) to (Y2_in2);
	\draw (Y_in3) to (Y_out2);
	\draw (Y1_out1) to (Y2_in1);
	\draw (Y2_out1) to (Y_out1);
	\node at ($(X.east)!.5!(Y.west)$) {\Huge$\cp$};
\end{tikzpicture}
\]
The result is:
\[
\begin{tikzpicture}[oriented WD, bbx=1.2cm, bb min width=.5cm, bb port length=2pt, bb port sep=.75, baseline=(X2.north)]
	\node[bb={1}{3}] (X1) {$a$};
	\node[bb={3}{3}, below right=-.5 and 1 of X1] (X2) {$b$};
	\node[bb={2}{1}, above right=-.5 and 1 of X2] (X3) {$c$};
	\node[bb={1}{1}, right= of X3] (Y1) {$d$};
	\node[bb={2}{1}, below right=-1.5 and 1 of Y1] (Y2) {$e$};
	\node[bb={0}{0}, fit={($(X1.north west)+(.3,0)$) (X2)  ($(X3.east)+(-.3,0)$) ($(Y2.south east)+(0,-2)$)}] (outer) {};
	\node[coordinate] at (outer.west|-X1_in1) (outer_in1) {};
	\node[coordinate] at (outer.west|-X2_in3) (outer_in2) {};
	\node[coordinate] at (outer.east|-Y2_out1) (outer_out1) {};
	\node[coordinate] at (outer.east|-X2_out3) (outer_out2) {};
	\draw (outer_in1) to (X1_in1);	
	\draw (outer_in2) to (X2_in3);
	\draw (X1_out1) to (X3_in1);
	\draw (X1_out2) to (X2_in2);
	\draw (X1_out3) to (X2_in1);
	\draw (X2_out1) to (X3_in2);
	\draw (X3_out1) to (Y1_in1);
	\draw (X2_out2) -- (X2_out2-|Y1.east) to (Y2_in2);
	\draw (X2_out3) to (outer_out2);
	\draw (Y1_out1) to (Y2_in1);
	\draw (Y2_out1) to (outer_out1);
\end{tikzpicture}
\]
}

\sol{exc.mon_prod_of_morphisms}{
Draw the monoidal product of the morphism shown in \cref{eqn.port_graph} with itself. It will be a $(4,6)$-port graph, i.e.\ a morphism $4\to 6$ in $\Cat{PG}$.
}
{
The monoidal product of two morphisms is drawn by stacking the corresponding port graphs. For this problem, we just stack the left-hand picture on top of itself to obtain the righthand picture:
\[
\begin{tikzpicture}[oriented WD, bb small, baseline=(Y.south)]
	\node[bb={1}{3}] (X1) {$a$};
	\node[bb={3}{3}, below right=of X1] (X2) {$b$};
	\node[bb={2}{1}, above right=of X2] (X3) {$c$};
	\node[bb={0}{0}, fit={($(X1.north west)+(.3,1.5)$) (X2)  ($(X3.east)+(-.3,0)$)}] (Y) {};
	\node[coordinate] at (Y.west|-X1_in1) (Y_in1) {};
	\node[coordinate] at (Y.west|-X2_in3) (Y_in2) {};
	\node[coordinate] at (X3_out1-|Y.east) (Y_out1) {};
	\node[coordinate] at (X2_out2-|Y.east) (Y_out2) {};
	\node[coordinate] at (X2_out3-|Y.east) (Y_out3) {};
	\draw (Y_in1) to (X1_in1);	
	\draw (Y_in2) to (X2_in3);
	\draw (X1_out1) to (X3_in1);
	\draw (X1_out2) to (X2_in2);
	\draw (X1_out3) to (X2_in1);
	\draw (X2_out1) to (X3_in2);
	\draw (X3_out1) to (Y_out1); 
	\draw (X2_out2) to (Y_out2);
	\draw (X2_out3) to (Y_out3);
\end{tikzpicture}
\hspace{1in}
\begin{tikzpicture}[oriented WD, bb small, baseline=(Y.south)]
	\node[bb={1}{3}] (X1) {$a$};
	\node[bb={3}{3}, below right=of X1] (X2) {$b$};
	\node[bb={2}{1}, above right=of X2] (X3) {$c\color{white}'$};
%
	\node[bb={1}{3}, below=6 of X1] (X1') {$a$};
	\node[bb={3}{3}, below right=of X1'] (X2') {$b'$};
	\node[bb={2}{1}, above right=of X2'] (X3') {$c'$};
	\node[bb={0}{0}, fit={($(X1.north west)+(.3,1.5)$) (X2')  ($(X3'.east)+(-.3,0)$)}] (Y) {};
%
	\node[coordinate] at (Y.west|-X1_in1) (Y_in1) {};
	\node[coordinate] at (Y.west|-X2_in3) (Y_in2) {};
	\node[coordinate] at (X3_out1-|Y.east) (Y_out1) {};
	\node[coordinate] at (X2_out2-|Y.east) (Y_out2) {};
	\node[coordinate] at (X2_out3-|Y.east) (Y_out3) {};
%
	\node[coordinate] at (Y.west|-X1'_in1) (Y_in1') {};
	\node[coordinate] at (Y.west|-X2'_in3) (Y_in2') {};
	\node[coordinate] at (X3'_out1-|Y.east) (Y_out1') {};
	\node[coordinate] at (X2'_out2-|Y.east) (Y_out2') {};
	\node[coordinate] at (X2'_out3-|Y.east) (Y_out3') {};
%
	\draw (Y_in1) to (X1_in1);	
	\draw (Y_in2) to (X2_in3);
	\draw (X1_out1) to (X3_in1);
	\draw (X1_out2) to (X2_in2);
	\draw (X1_out3) to (X2_in1);
	\draw (X2_out1) to (X3_in2);
	\draw (X3_out1) to (Y_out1); 
	\draw (X2_out2) to (Y_out2);
	\draw (X2_out3) to (Y_out3);
	\draw (Y_in1') to (X1'_in1);	
	\draw (Y_in2') to (X2'_in3);
	\draw (X1'_out1) to (X3'_in1);
	\draw (X1'_out2) to (X2'_in2);
	\draw (X1'_out3) to (X2'_in1);
	\draw (X2'_out1) to (X3'_in2);
	\draw (X3'_out1) to (Y_out1'); 
	\draw (X2'_out2) to (Y_out2');
	\draw (X2'_out3) to (Y_out3');
\end{tikzpicture}
\]
}

\sol{exc.free_preorder_check_1}{
Let $P$ be a set, let $R\ss P \times P$ a relation, let $(P,\leq_P)$ be the preorder
obtained by taking the reflexive, transitive closure of $R$, and let $(Q,\leq_Q)$ be an arbitrary preorder. Finally, let $f\colon P\to Q$ be a function, not assumed monotone.
\begin{enumerate}
	\item Suppose that for every $x,y\in P$, if $R(x,y)$ then $f(x)\leq f(y)$. Show that $f$ defines a monotone map $f\colon (P,\leq_P)\to (Q,\leq_Q)$.
	\item Suppose that $f$ defines a monotone map $f\colon (P,\leq_P)\to (Q,\leq_Q)$. Show that for every $x,y\in P$, if $R(x,y)$ then $f(x)\leq_Q f(y)$.
\end{enumerate}
We call this the \emph{universal property} of the free preorder.\index{universal
property}
}{
We have a relation $R\ss P\times P$ which generates a preorder $\leq_P$ on $P$, we have an arbitrary preorder $(Q,\leq_Q)$ and a function $f\colon P\to Q$, not necessarily monotonic.
\begin{enumerate}
	\item Assume that for every $x,y\in P$, if $R(x,y)$ then $f(x)\leq f(y)$; we want to show that $f$ is monotone, i.e.\ that for every $x\leq_Py$ we have $f(x)\leq_Qf(y)$. By definition of $P$ being the reflexive, transitive closure of $R$, we have $x\leq_Py$ iff there exists $n\in\nn$ and $x_0,\ldots,x_n$ in $P$ with $x_0=x$ and $x_n=y$ and $R(x_i,x_{i+1})$ for each $0\leq i\leq n-1$. (The case $n=0$ handles reflexivity.) But then by assumption, $R(x_i,x_{i+1})$ implies $f(x_i)\leq_Q f(x_{i+1})$ for each $i$. By induction on $i$ we show that $f(x_0)\leq_Q f(x_i)$ for all $0\leq i\leq n$, at which point we are done.
	\item Suppose now that $f$ is monotone, and take $x,y\in P$ for which $R(x,y)$ holds. Then $x\leq_Py$ because $\leq_P$ is the smallest preorder relation containing $R$. (Another way to see this based on the above description is with $n=1$, $x_0=x$, and $x_n=y$, which we said implies $x\leq_Py$.) Since $f$ is monotone, we indeed have $f(x)\leq_Qf(y)$.
\end{enumerate}
}

\sol{exc.free_preorder_check_2}{
Let $P$, $Q$, $R$, etc.\ be as in \cref{exc.free_preorder_check_1}. We want to see that the universal property is really about maps out of---and not maps in to---the reflexive, transitive closure $(P,\leq)$. So let $g\colon Q\to P$ be a function.
\begin{enumerate}
	\item Suppose that for every $a,b\in Q$, if $a\leq b$ then $(g(a),g(b))\in R$. Is it automatically true that $g$ defines a monotone map $g\colon(Q,\leq_Q)\to(P,\leq_P)$?
	\item Suppose that $g$ defines a monotone map $g\colon(Q,\leq_Q)\to(P,\leq_P)$. Is it automatically true that for every $a,b\in Q$, if $a\leq b$ then $(g(a),g(b))\in R$?
\end{enumerate}
The lesson is that maps between structured objects are defined to preserve
structure. This means the domain of a map must be more constrained than the
codomain. Thus having the fewest additional constraints coincides with having
the most maps out---every function that respects our generating constraints
should define a map.%
\footnote{A higher-level justification understands freeness
as a left adjoint---of an adjunction often called the syntax/semantics
adjunction---but we will not discuss that here.}
}{
Suppose that $P$, $Q$, and $R$ are as in \cref{exc.free_preorder_check_1} and we have a function $g\colon Q\to P$.
\begin{enumerate}
	\item If $R(g(a),g(b))$ holds for all $a\leq_Qb$ then $g$ is monotone, because $R(x,y)$ implies $x\leq_Py$.
	\item It is possible for $g\colon(Q,\leq_Q)\to(P,\leq_P)$ to be monotone and yet have some $a,b\in Q$ with $a\leq_Qb$ and $(g(a),g(b))\not\in R$. Indeed, take $Q\coloneqq\{1\}$ to be the free preorder on one element, and take $P\coloneqq\{1\}$ with $R=\varnothing$. Then the unique function $g\colon Q\to P$ is monotone (because $\leq_P$ is reflexive even though $R$ is empty), and yet $(g(1),g(1))\not\in R$.
\end{enumerate}
}

\sol{exc.free_cat_is_free}{
Let $G=(V,A,s,t)$ be a graph, and let $\cat{G}\coloneqq\Cat{Free}(G)$ be the free category on $G$. Let $\cat{C}$ be another category whose set of morphisms is denoted $\Mor(\cat{C})$. 
\begin{enumerate}
	\item Someone tells you that there are ``domain and codomain'' functions $\dom,\cod\colon\Mor(\cat{C})\to\Ob(\cat{C})$; interpret this statement.
	\item Show that the set of functors $\cat{G} \to
\cat{C}$ are in one-to-one correspondence with the set of pairs of functions
$(f,g)$, where $f\colon V \to \Ob(\cat{C})$ and $g\colon A\to\Mor(\cat{C})$ such that $\dom(g(a))=f(s(a))$ and $\cod(g(a))=f(t(a))$ for all $a\in A$.
	\item Is $(\Mor(\cat{C}),\Ob(\cat{C}),\dom,\cod)$ a graph? If so, see if you can use the word ``adjunction'' in a sentence that describes the statement in part 2. If not, explain why not.

\end{enumerate}
}{
Let $G=(V,A,s,t)$ be a graph, let $\cat{G}$ be the free category on $G$, and let $\cat{C}$ be another category, whose set of morphisms is denoted $\Mor(\cat{C})$.
\begin{enumerate}
	\item To give a function $\Mor(\cat{C})\to\Ob(\cat{C})$ means that for every element $\Mor(\cat{C})$ we need to give exactly one element of $\Ob(\cat{C})$. So for $\dom$ we take any $q\in\Mor(\cat{C})$, view it as a morphism $q\colon y\to z$, and send it to its domain $y$. Similarly for $\cod$: we put $\cod(q)\coloneqq z$.
	\item Suppose first that we are given a functor $F\colon\cat{G}\to\cat{C}$. On objects we have a function $\Ob(\cat{G})\to\Ob(\cat{C})$, and this defines $f$ since $\Ob(\cat{G})=V$. On morphisms, first note that the arrows of graph $G$ are exactly the length=1 paths in $G$, whereas $\Mor(\cat{G})$ is the set of all paths in $G$, so we have an inclusion $A\ss\Mor(\cat{G})$. The functor $F$ provides a function $\Mor(\cat{G})\to\Mor(\cat{C})$, which we can restrict to $A$ to obtain $g\colon A\to\Mor(\cat{C})$. All functors satisfy $\dom(F(r))=F(\dom(r))$ and $\cod(F(r))=F(\cod(r))$ for any $r\colon w\to x$. In particular when $r\in A$ is an arrow we have $\dom(r)=s(r)$ and $\cod(r)=t(r)$. Thus we have found $(f,g)$ with the required properties.\\	
	
	Suppose second that we are given a pair of functions $(f,g)$ where $f\colon V \to \Ob(\cat{C})$ and $g\colon A\to\Mor(\cat{C})$ such that $\dom(g(a))=f(s(a))$ and $\cod(g(a))=f(t(a))$ for all $a\in A$. Define $F\colon\cat{G}\to\cat{C}$ on objects by $f$. An arbitrary morphism in $\cat{G}$ is a path $p\coloneqq(v_0,a_1,a_2,\ldots,a_n)$ in $G$, where $v_0\in V$, $a_i\in A$, $v_0=s(a_1)$, and $t(a_i)=s(a_{i+1})$ for all $1\leq i\leq n-1$. Then $g(a_i)$ is a morphism in $\cat{C}$ whose domain is $f(v_0)$ and the morphisms $g(a_i)$ and $g(a_{i+1})$ are composable for every $1\leq i\leq n-1$. We then take $F(p)\coloneqq \id_{f(v_0)}\cp g(a_1)\cp\cdots\cp g(a_n)$ to be the composite. It is easy to check that this is indeed a functor (preserves identities and compositions).\\	

	Third, we want to see that the two operations we just gave are mutually inverse. On objects this is straightforward, and on morphisms it is straightforward to see that, given $(f,g)$, if we turn them into a functor $F\colon\cat{G}\to\cat{C}$ and then extract the new pair of functions $(f',g')$, then $f=f'$ and $g=g'$. Finally, given a functor $F\colon\cat{G}\to\cat{C}$, we extract the pair of functions $(f,g)$ as above and then turn them into a new functor $F'\colon\cat{G}\to\cat{C}$. It is clear that $F$ and $F'$ act the same on objects, so what about on morphisms. The formula says that $F'$ acts the same on morphisms of length $1$ in $\cat{G}$ (i.e.\ on the elements of $A$). But an arbitrary morphism in $\cat{G}$ is just a path, i.e.\ a sequence of composable arrows, and so by functoriality, both $F$ and $F'$ must act the same on arbitrary paths.
	\item $(\Mor(\cat{C}),\Ob(\cat{C}),\dom,\cod)$ is a graph; let's denote it $U(\cat{C})\in\Cat{Grph}$. We have functors $\free\colon\Cat{Grph}\leftrightarrows\smcat\cocolon U$, and $\free$ is left adjoint to $U$.
\end{enumerate}
}


\sol{exc.free_monoid}{
Recall that monoids are one-object categories. For any set $A$, there is a graph with one vertex and an arrow from the vertex to itself for each $a\in A$. Thus we can construct a free monoid from just the data of a set $A$.
\begin{enumerate}
	\item What are the elements of the free monoid on the set $A=\{a\}$?
	\item Can you find a well-known monoid that is isomorphic to the free monoid on $\{a\}$?
	\item What are the elements of the free monoid on the set $A=\{a,b\}$?
\end{enumerate}
}{
\begin{enumerate}
	\item The elements of the free monoid on the set $\{a\}$ are:
	\[a^0, a^1, a^2,a^3,\ldots, a^{2019},\ldots\]
	with monoid multiplication $*$ given by the usual natural number addition on the exponents, $a^i*a^j=a^{i+j}$.
	\item This is isomorphic to $\nn$, by sending $a^i\mapsto i$.
	\item The elements of the free monoid on the set $\{a,b\}$ are `words in $a$ and $b$,' each of which we will represent as a list whose entries are either $a$ or $b$. Here are some:
	\[[\;],\quad [a],\quad [b],\quad [a,a],\quad [a,b],\quad \ldots,\quad [b,a,b,b,a,b,a,a,a,a],\quad\ldots\]
\end{enumerate}
}

\sol{exc.free_prop_port_graph}{
Consider the following prop signature:
\[
  G\coloneqq\{\rho_{m,n}\colon m \to n \mid m, n \in \nn\},\qquad s(\rho_{m,n})\coloneqq m,\quad t(\rho_{m,n})\coloneqq n,
\]
i.e.\ having one generating morphism for each $(m,n)\in\nn^2$. Show that $\free(G)$ is the prop of port graphs.
}{
We have two props: the prop of port graphs and the free prop $\free(G,s,t)$ where
\[
  G\coloneqq\{\rho_{m,n}\colon m \to n \mid m, n \in \nn\},\qquad s(\rho_{m,n})\coloneqq m,\quad t(\rho_{m,n})\coloneqq n;
\]
we want to show they are the same prop. As categories they have the same set of objects (in both cases, $\nn$), so we need to show that for every $m,n\in\nn$, they have the same set of morphisms (and that their composition formulas and monoidal product formulas agree).

By \cref{def.free_prop}, a morphism $m\to n$ in $\free(G)$ is a $G$-labeled port graph, i.e.\ a pair $(\Gamma,\ell)$, where $\Gamma=(V,\pgin,\pgout,\iota)$ is an $(m,n)$-port graph and $\ell\colon V \to G$ is a function, such that the `arities agree.' What does this mean? Recall that every vertex $v\in V$ is drawn as a box with some left-hand ports and some right-hand ports---an arity---and $\ell(v)\in G$ is supposed to have the correct arity; precisely, $s(\ell(v))=\pgin(v)$ and $t(\ell(v)) = \pgout(v)$. But $G$ was chosen so that it has exactly one element with any given arity, so the function $\ell$ has only one choice, and thus contributes nothing: it neither increases nor decreases the freedom. In other words, a morphism in our particular $\free(G)$ can be identified with an $(m,n)$ port graph $\Gamma$, as desired.

Again by definition \cref{def.free_prop}, the `composition and the monoidal structure are just those for port graphs
  $\Cat{PG}$ (see \cref{eqn.PG_prop}); the labelings (the $\ell$'s) are just carried along.' So we are done.
}

\sol{exc.free_prop_pic}{
Consider again the free prop on generators $G=\{f\colon 1 \to 1, g\colon 2 \to 2, h\colon 2 \to
1\}$. Draw a picture of $(f+\id_1+\id_1)\cp(\sigma+\id_1)\cp(\id_1+h)\cp \sigma\cp g$.
}{
Here is a picture of $(f+\id_1+\id_1)\cp(\sigma+\id_1)\cp(\id_1+h)\cp \sigma\cp g$, in the free prop on generators $G=\{f\colon 1 \to 1, g\colon 2 \to 2, h\colon 2 \to
1\}$:
\[
\begin{tikzpicture}[oriented WD, bb port length=0pt, bbx=.7cm, bb port sep=1pt]
	\node[bb={1}{1}] (f) {$f$};
	\node[bb={2}{1}, below right=1 and 1.5 of f] (h) {$h$};
	\node[bb={2}{2}, above right=0 and 1.5 of h] (g) {$g$};
	\node[bb={0}{0}, fit=(f) (g) (h)] (outer) {};
	\coordinate (outer_in1) at (outer.west|-f_in1);
	\coordinate (outer_in2) at (outer.west|-h_in1);
	\coordinate (outer_in3) at (outer.west|-h_in2);
	\coordinate (outer_out1) at (outer.east|-g_out1);
	\coordinate (outer_out2) at (outer.east|-g_out2);
	\coordinate (A1) at ($(f_out1)+(0.5,0)$);
	\coordinate (A2) at (A1|-outer_in2);
	\coordinate (A3) at (A1|-outer_in3);
	\coordinate (B1) at ($(A1)+(0.5,0)$);
	\coordinate (B2) at (B1|-A2);
	\coordinate (B3) at (B1|-A3);
	\coordinate (C2) at ($(h_out1)+(0.5,0)$);
	\coordinate (C1) at (C2|-A1);
	\coordinate (D1) at ($(g_in1)-(0.5,0)$);
	\coordinate (D2) at ($(g_in2)-(0.5,0)$);
	\draw (outer_in1) -- (f_in1);
	\draw (f_out1) -- (A1) -- (B2) -- (h_in1);
	\draw (outer_in2) -- (A2) -- (B1) -- (C1) -- (D2) -- (g_in2);
	\draw (outer_in3) -- (A3) -- (B3) -- (h_in2);
	\draw (h_out1) -- (C2) -- (D1) -- (g_in1);
	\draw (g_out1) -- (outer_out1);
	\draw (g_out2) -- (outer_out2);
\end{tikzpicture}
\]
}


\sol{exc.same_free_prop}{
Is it the case that the free prop on generators $(G,s,t)$, defined in \cref{def.free_prop}, is the same thing as the prop presented by $(G,s,t,\varnothing)$, having no relations, as defined in \cref{rdef.presentation_prop}? Or is there a subtle difference somehow?
}{
The free prop on generators $(G,s,t)$, defined in \cref{def.free_prop}, is---for all intents and purposes---the same thing as the prop presented by $(G,s,t,\varnothing)$, having no relations. The only possible ``subtle difference'' we might have to admit is if someone said that a set $S$ is ``subtly different'' than its quotient by the trivial equivalence relation. In the latter, the elements are the singleton subsets of $S$. So for example the quotient of $S=\{1,2,3\}$ by the trivial equivalence relation is the set $\{\{1\}, \{2\}, \{3\}\}$. It is subtly different than $S$, but the two are naturally isomorphic, and category-theoretically, the difference will never make a difference.
}


\sol{exc.rigs_mats}{
\begin{enumerate}
	\item We said in \cref{ex.mat_rig} that for any rig $R$, the set $\Set{Mat}_n(R)$ forms a rig. What is its multiplicative identity $1\in\Set{Mat}_n(R)$?
	\item We also said that $\Set{Mat}_n(R)$ is generally not commutative. Pick an $n$ and show that that $\Set{Mat}_n(\NN)$ is not commutative, where $\NN$ is as in \cref{ex.rig_nat}.
\end{enumerate}
}{
\begin{enumerate}
	\item If $(R,0,+,1,*)$ is a rig, then the multiplicative identity $1\in\Set{Mat}_n(R)$ is the usual $n$-by-$n$ identity matrix: 1's on the diagonal and 0's everywhere else (where by `1' and `0', we mean those elements of $R$). So for $n=4$ it is:
  \[
  \left(
  \begin{tabular}{cccc}
  	1&0&0&0\\
  	0&1&0&0\\
  	0&0&1&0\\
  	0&0&0&1	
  \end{tabular}
  \right).
  \]
	\item We choose $n=2$ and hence need to find two elements $A,B\in\Set{Mat}_2(\nn)$ such that $A*B\neq B*A$.
	\[
  	A*B=\left(
    	\begin{array}{cc}
    		0&1\\
    		0&0
    	\end{array}
  	\right)
  	*
  	\left(
    	\begin{array}{cc}
    		0&1\\
    		1&0
    	\end{array}	
  	\right)
  	\neq
  	\left(
    	\begin{array}{cc}
    		0&1\\
    		1&0
    	\end{array}	
  	\right)
    *
  	\left(
    	\begin{array}{cc}
    		0&1\\
    		0&0
    	\end{array}
  	\right)
	=B*A
	\]
	One can calculate from the multiplication formula (recalled in \cref{ex.mat_rig}) says $(A*B)(1,1)=0*0+1*1=1$ and $(B*A)(1,1)=0*0+0*0=0$, which are not equal.
\end{enumerate}
}


\sol{exc.a_signal_flow_graph}{
  The flow graph  
\[
\begin{tikzpicture}[scale=.7]
	\begin{pgfonlayer}{nodelayer}
		\node [style=none] (0) at (-6, -0) {};
		\node [style=bdot] (1) at (-5, -0) {};
		\node [style=wamp] (2) at (-4, 0.5) {$\scriptstyle 3$};
		\node [style=none] (3) at (-4, -0.5) {};
		\node [style=wdot] (4) at (-3, -0) {};
		\node [style=none] (5) at (-6, 1.5) {};
		\node [style=bdot] (6) at (-2.5, 1.5) {};
		\node [style=none] (7) at (-1.5, 1) {};
		\node [style=wdot] (8) at (-0.5, 0.5) {};
		\node [style=none] (9) at (-1.5, 2) {};
		\node [style=wamp] (10) at (-0.75, 2) {$\scriptstyle 5$};
		\node [style=wdot] (11) at (2.25, 1.5) {};
		\node [style=none] (12) at (1.25, 1) {};
		\node [style=bdot] (13) at (0.25, 0.5) {};
		\node [style=none] (14) at (1.25, -0) {};
		\node [style=none] (15) at (3, -0) {};
		\node [style=wamp] (16) at (0.5, 2) {$\scriptstyle {3}$};
		\node [style=none] (17) at (1.25, 2) {};
		\node [style=none] (18) at (-1.5, -0) {};
		\node [style=none] (19) at (3, 1.5) {};
	\end{pgfonlayer}
	\begin{pgfonlayer}{edgelayer}
		\draw (0.center) to (1);
		\draw [bend left, looseness=1.00] (1) to (2.center);
		\draw [bend right, looseness=1.00] (1) to (3.center);
		\draw [bend left, looseness=1.00] (2.center) to (4);
		\draw [bend right, looseness=1.00] (3.center) to (4);
		\draw (4) to (18.center);
		\draw [bend right, looseness=1.00] (18.center) to (8);
		\draw [bend right, looseness=1.00] (8) to (7.center);
		\draw [bend left, looseness=1.00] (7.center) to (6);
		\draw (6) to (5);
		\draw [bend left, looseness=1.00] (6) to (9.center);
		\draw (9.center) to (10.center);
		\draw (10.center) to (16.center);
		\draw (16.center) to (17.center);
		\draw [bend left, looseness=1.00] (17.center) to (11);
		\draw [bend left, looseness=1.00] (11) to (12.center);
		\draw (11) to (19);
		\draw [bend right, looseness=1.00] (12.center) to (13);
		\draw (8) to (13);
		\draw [bend right, looseness=1.00] (13) to (14.center);
		\draw (14.center) to (15.center);
	\end{pgfonlayer}
\end{tikzpicture}
\]
  takes in two natural numbers, $x$ and $y$, and produces two
  output signals. What are they?
}{
  Semantically, if we apply the flow graph below to the input signal $(x,y)$ 
\[
\begin{tikzpicture}[scale=.7]
	\begin{pgfonlayer}{nodelayer}
		\node [style=none] (0) at (-6, -0) {};
		\node [style=bdot] (1) at (-5, -0) {};
		\node [style=wamp] (2) at (-4, 0.5) {$\scriptstyle 3$};
		\node [style=none] (3) at (-4, -0.5) {};
		\node [style=wdot] (4) at (-3, -0) {};
		\node [style=none] (5) at (-6, 1.5) {};
		\node [style=bdot] (6) at (-2.5, 1.5) {};
		\node [style=none] (7) at (-1.5, 1) {};
		\node [style=wdot] (8) at (-0.5, 0.5) {};
		\node [style=none] (9) at (-1.5, 2) {};
		\node [style=wamp] (10) at (-0.75, 2) {$\scriptstyle 5$};
		\node [style=wdot] (11) at (2.25, 1.5) {};
		\node [style=none] (12) at (1.25, 1) {};
		\node [style=bdot] (13) at (0.25, 0.5) {};
		\node [style=none] (14) at (1.25, -0) {};
		\node [style=none] (15) at (3, -0) {};
		\node [style=wamp] (16) at (0.5, 2) {$\scriptstyle {3}$};
		\node [style=none] (17) at (1.25, 2) {};
		\node [style=none] (18) at (-1.5, -0) {};
		\node [style=none] (19) at (3, 1.5) {};
	\end{pgfonlayer}
	\begin{pgfonlayer}{edgelayer}
		\draw (0.center) to (1);
		\draw [bend left, looseness=1.00] (1) to (2.center);
		\draw [bend right, looseness=1.00] (1) to (3.center);
		\draw [bend left, looseness=1.00] (2.center) to (4);
		\draw [bend right, looseness=1.00] (3.center) to (4);
		\draw (4) to (18.center);
		\draw [bend right, looseness=1.00] (18.center) to (8);
		\draw [bend right, looseness=1.00] (8) to (7.center);
		\draw [bend left, looseness=1.00] (7.center) to (6);
		\draw (6) to (5);
		\draw [bend left, looseness=1.00] (6) to (9.center);
		\draw (9.center) to (10.center);
		\draw (10.center) to (16.center);
		\draw (16.center) to (17.center);
		\draw [bend left, looseness=1.00] (17.center) to (11);
		\draw [bend left, looseness=1.00] (11) to (12.center);
		\draw (11) to (19);
		\draw [bend right, looseness=1.00] (12.center) to (13);
		\draw (8) to (13);
		\draw [bend right, looseness=1.00] (13) to (14.center);
		\draw (14.center) to (15.center);
	\end{pgfonlayer}
\end{tikzpicture}
\]
the resulting output signal is $(16x+4y, x+4y)$.
}


\sol{exc.monoidal_mat_prod}{
Let $A$ and $B$ be the following matrices with values in $\NN$:
\[
A=\left(
\begin{array}{ccc}
	3&3&1\\
	2&0&4
\end{array}
\right)
\hspace{1in}
B=\left(
\begin{array}{cccc}
	2&5&6&1\\
\end{array}
\right)
\]
What is the monoidal product matrix $A+B$?
}{
The monoidal product of 
$A=\left(
\begin{array}{ccc}
	3&3&1\\
	2&0&4
\end{array}
\right)
$
and 
$B=\left(
\begin{array}{cccc}
	2&5&6&1\\
\end{array}
\right)
$
is
\[
  A+B=\left(
  \begin{array}{ccccccc}
  	3&3&1&0&0&0&0\\
  	2&0&4&0&0&0&0\\
  	0&0&0&2&5&6&1
  \end{array}
  \right)
\]
}


\sol{exc.sig_flow_mats}{
\begin{enumerate}
	\item  What matrices does the signal flow graph
  \[
\begin{aligned}
\begin{tikzpicture}[spider diagram]
	\node[spider={1}{2}, fill=black] (a) {};
	\node[special spider={1}{2}{0}{\leglen}, fill=black, right=.1 of a_out1] (b) {};
	\draw (a_out1) to (b_in1);
	\draw (a_out2) to (b_out1|-a_out2);
\end{tikzpicture}
\end{aligned}
  \]
  represent?
  \item What about the signal flow graph
  \[
\begin{aligned}
\begin{tikzpicture}[spider diagram]
	\node[spider={1}{2}, fill=black] (a) {};
	\node[special spider={1}{2}{0}{\leglen}, fill=black, right=.1 of a_out2] (b) {};
	\draw (a_out2) to (b_in1);
	\draw (a_out1) to (b_out1|-a_out1);
\end{tikzpicture}
\end{aligned}
    ?
  \]
  \item Are they equal?
\end{enumerate}
}{
\begin{enumerate}
	\item  The signal flow graph on the left represents the matrix on the right:
  \[
    \parbox{1in}{
$
\begin{aligned}
\begin{tikzpicture}[spider diagram]
	\node[spider={1}{2}, fill=black] (a) {};
	\node[special spider={1}{2}{0}{\leglen}, fill=black, right=.1 of a_out1] (b) {};
	\draw (a_out1) to (b_in1);
	\draw (a_out2) to (b_out1|-a_out2);
\end{tikzpicture}
\end{aligned}
$
}
    \hspace{1in}
    \parbox{1in}{$\left(\begin{array}{ccc}1&1&1\end{array}\right)$}
  \]
	\item  The signal flow graph on the left represents the matrix on the right:
  \[
    \parbox{1in}{
    $
\begin{aligned}
\begin{tikzpicture}[spider diagram]
	\node[spider={1}{2}, fill=black] (a) {};
	\node[special spider={1}{2}{0}{\leglen}, fill=black, right=.1 of a_out2] (b) {};
	\draw (a_out2) to (b_in1);
	\draw (a_out1) to (b_out1|-a_out1);
\end{tikzpicture}
\end{aligned}$
}
    \hspace{1in}
    \parbox{1in}{$\left(\begin{array}{ccc}1&1&1\end{array}\right)$}
  \]
  \item They are equal.
\end{enumerate}
}

\sol{ex.drawsfgs}{
  Draw signal flow graphs that represent the following matrices:\\
  \begin{enumerate*}[itemjoin=\hspace{1in}]
  \item[] \hspace{-.2in}
    \item $
      \begin{pmatrix}
	0\\ 
	1 \\
	2
      \end{pmatrix}$
    \item $
      \begin{pmatrix}
	0 & 0 \\ 
	0 & 0
      \end{pmatrix}$
    \item $
      \begin{pmatrix}
	1 & 2 & 3\\ 
	4 & 5 & 6
      \end{pmatrix}$
  \end{enumerate*}
}{
\begin{enumerate}
\item
\[
      \begin{pmatrix}
	0 \\ 
	1 \\
	2
      \end{pmatrix}\qquad \sim \qquad
      \begin{aligned}
\begin{tikzpicture}[oriented WD, spider diagram, bbx=3ex, bby=2ex]
	\node[spider={2}{1}, fill=white] (add) {};
	\node[spider={1}{1}, wamp,left=.2 of add_in2] (scalar) {$2$};	
	\node[spider={1}{0}, fill=black, above =1 of add_in1] (unit) {};
	%
	\draw (add_in1-|scalar_in1) -- (add_in1);
	\draw (unit_in1-|scalar_in1) -- (unit_in1);
\end{tikzpicture}
\end{aligned}
\]
\item
\[
      \begin{pmatrix}
	0 & 0 \\ 
	0 & 0
      \end{pmatrix}\qquad \sim \qquad
      \begin{aligned}
\begin{tikzpicture}[oriented WD, spider diagram, bbx=3ex, bby=2ex]
	\node[spider={1}{0}, fill=black] (b1) {};
	\node[spider={1}{0}, fill=black, below=.5 of b1] (b2) {};
	\node[spider={0}{1}, fill=white, right=1 of b1] (w1) {};
	\node[spider={0}{1}, fill=white, below=.5 of w1] (w2) {};
\end{tikzpicture}
\end{aligned}
\]
\item
\[
      \begin{pmatrix}
	1 & 2 & 3\\ 
	4 & 5 & 6
      \end{pmatrix}\qquad \sim \qquad
      \begin{aligned}
\begin{tikzpicture}[oriented WD, spider diagram, bbx=3ex, bby=2ex]
        \coordinate (s1) at (0,0) ;
	\node[spider={1}{1}, wamp,below=.4 of s1] (s2) {$2$};	
	\node[spider={1}{1}, wamp,below=.4 of s2] (s3) {$3$};	
	\node[spider={1}{1}, wamp,below=.4 of s3] (s4) {$4$};	
	\node[spider={1}{1}, wamp,below=.4 of s4] (s5) {$5$};	
	\node[spider={1}{1}, wamp,below=.4 of s5] (s6) {$6$};	
	\node[spider={2}{1}, fill=white] (add1) at ($(s1)!.5!(s2)+(5,0)$) {};
	\node[spider={2}{1}, fill=white] (add2) at ($(s3)!.5!(s4)+(5,0)$) {};
	\node[spider={2}{1}, fill=white] (add3) at ($(s5)!.5!(s6)+(5,0)$) {};
	\node[spider={1}{2}, fill=black] (copy1) at ($(s2)-(5,0)$) {};
	\node[spider={1}{2}, fill=black] (copy2) at ($(copy1_out2)+(1,0)$) {};
	\node[spider={1}{2}, fill=black] (copy3) at ($(s5)-(5,0)$) {};
	\node[spider={1}{2}, fill=black] (copy4) at ($(copy3_out2)+(1,0)$) {};
	%
	\draw (copy1_out1) to (s1);
	\draw (copy1_out2) to (copy2_in1);
	\draw (copy2_out1) to (s2_in1);
	\draw (copy2_out2) to (s3_in1);
	\draw (copy3_out1) to (s4_in1);
	\draw (copy3_out2) to (copy4_in1);
	\draw (copy4_out1) to (s5_in1);
	\draw (copy4_out2) to (s6_in1);
	\draw (s1) to (add1_in1);
	\draw (s2_out1) to (add2_in1);
	\draw (s3_out1) to (add3_in1);
	\draw (s4_out1) to (add1_in2);
	\draw (s5_out1) to (add2_in2);
	\draw (s6_out1) to (add3_in2);
\end{tikzpicture}
\end{aligned}
\]
\end{enumerate}
}

\sol{exc.general_case_S_full}{
Write down a detailed proof of \cref{prop.Sfull}. Suppose $M$ is an $m \times
n$-matrix. Follow the idea of the $(2\times 2)$-case in \cref{eqn.two_by_two},
and construct the signal flow graph $g$---having $m$ inputs and $n$ outputs---as
the composite of four layers, respectively comprising (i) copy maps, (ii)
scalars, (iii) swaps and identities, (iv) addition maps.}
{
\begin{itemize}
  \item For the first layer $g_1$, take the monoidal product of $m$ copies of $c_n$,
    \[
      g_1\coloneq c_n +\dots+ c_n\colon m \to (m \times n),
    \]
    where $c_n$ is the signal flow diagram that makes $n$ copies of a single input:
    \[
      c_n\coloneq
      \comult{1em}\cp (1+\comult{1em})\cp (1+1+\comult{1em})\cp \dots\cp (1+\dots+1+\comult{1em})\colon
      1 \to n
    \]
  \item Next, define
    \begin{align*}
      g_2\coloneq &\quad s_{M(1,1)}+\cdots+s_{M(1,n)} \\
      & +s_{M(2,1)}+\cdots+s_{M(2,n)} \\
      &+\cdots \\
      &+s_{M(m,1)}+\cdots+s_{M(m,n)}\colon (m\times n) \to (m\times n),
    \end{align*}
    where $s_a\colon 1 \to 1$ is the signal flow
    graph generator ``scalar multiplication by $a$.'' This layer amplifies each copy of the input signal by the
    relevant rig element.
  \item The third layer rearranges wires. We will not write this down
    explicitly, but simply say it is the signal flow graph $g_3\colon m \times
    n \to m \times n$, that is the
    composite and monoidal product of swap and identity maps, such that the
    $(i-1)m+j$th input is sent to the $(j-1)n+i$th output, for all $1 \le i \le n$
    and $1 \le j \le m$.
  \item Finally, the fourth layer is similar to the first, but instead adds the
    amplified input signals. We define 
    \[
      g_4\coloneq a_m+\dots+a_m \colon (m \times n)\to n,
    \]
    where $a_m$ is the signal flow graph that adds $m$ inputs to produce a single output:
    \[
      a_m\coloneq
      (1+\dots+1+\add{1em})\cp \cdots\cp(1+1+\add{1em})\cp(1+\add{1em})\cp\add{1em}\colon
      m \to 1
    \]
\end{itemize}
Using \cref{prop.Simages}, it is a straightforward but tedious calculation to show that $g=g_1\cp g_2\cp g_3\cp g_4\colon m \to n$ has the property that $S(g)=M$.
}


\sol{exc.rep_mats}{
  \begin{enumerate}
  	\item For each matrix in \cref{ex.drawsfgs}, draw another signal flow graph that
  represents that matrix.
  	\item Using the above equations and the prop axioms, prove
  that the two signal flow graphs represent the same matrix.
  
  \end{enumerate}
}{
\begin{enumerate}
\item The matrices in \cref{ex.drawsfgs} may also be drawn as the following
signal flow graphs:
\begin{enumerate}
\item
  \[
      \begin{pmatrix}
	0 \\ 
	1 \\
	2
      \end{pmatrix}\qquad \sim \qquad
      \begin{aligned}
\begin{tikzpicture}[oriented WD, spider diagram, bbx=3ex, bby=2ex]
	\node[spider={2}{1}, fill=white] (add2) {};
	\node[special spider={2}{1}{\leglen}{0}, fill=white, left=.2 of add2_in2] (add) {};
	\node[spider={1}{2}, fill=black,left=2 of add] (copy) {};
	\node[spider={1}{0}, fill=black, above =2 of add_in1] (unit) {};
	%
	\draw (add2_in1-|copy_in1) -- (add2_in1);
	\draw (unit_in1-|copy_in1) -- (unit_in1);
	\draw (copy_out1) -- (add_in1);
	\draw (copy_out2) -- (add_in2);
	\draw (add.0) -- (add2_in2);
\end{tikzpicture}
\end{aligned}
\]
\item
\[
      \begin{pmatrix}
	0 & 0 \\ 
	0 & 0
      \end{pmatrix}\qquad \sim \qquad
      \begin{aligned}
\begin{tikzpicture}[oriented WD, spider diagram, bbx=3ex, bby=2ex]
	\node[spider={2}{1}, fill=white] (add) {};
	\node[special spider={1}{1}{.1}{0}, wamp, right=0 of add_out1]
	(scalar) {$0$};
	\node[spider={1}{2}, fill=black,right=.8 of scalar] (copy) {};
	%
	\draw (add_out1) -- (scalar.180);
	\draw (scalar.0) -- (copy_in1);
\end{tikzpicture}
\end{aligned}
\]
\item
\[
      \begin{pmatrix}
	1 & 2 & 3\\ 
	4 & 5 & 6
      \end{pmatrix}\qquad \sim \qquad
      \begin{aligned}
	\begin{tikzpicture}[oriented WD, spider diagram, bbx=3ex, bby=2ex]
        \coordinate (s1) at (0,0) ;
	\node[spider={1}{1}, wamp,below=.3 of s1] (s2) {$2$};	
	\node[spider={1}{1}, wamp,below=.6 of s2] (s3) {$3$};	
	\node[spider={1}{1}, wamp,below=1.2 of s3] (s5) {$5$};	
	\node[spider={1}{1}, wamp] (s4) at ($(s3)!.5!(s5)+(4,0)$) {$4$};	
	\node[spider={1}{1}, wamp,below left=.4 of s5] (s6) {$6$};	
	\node[special spider={2}{1}{\leglen}{.2}, fill=white] (add2) at ($(s1)!.5!(s2)+(6,-1)$) {};
	\node[special spider={2}{1}{\leglen}{.2}, fill=white] (add1) at ($(s2)!.5!(s5)+(6,-.38)$) {};
	\node[special spider={2}{1}{\leglen}{.2}, fill=white] (add3) at ($(s5)!.5!(s6)+(4,0)$) {};
	\node[spider={1}{2}, fill=black] (copy1) at ($(s2)-(5,0)$) {};
	\node[spider={1}{2}, fill=black] (copy2) at ($(copy1_out2)+(1,0)$) {};
	\node[spider={1}{2}, fill=black] (copy3) at ($(s5)-(5,0)$) {};
	\node[spider={1}{2}, fill=black] (copy4) at ($(copy3_out1)+(1,0)$) {};
        \coordinate (out1) at ($(add2)+(2,0)$) ;
        \coordinate (out2) at ($(add1)+(2,0)$) ;
        \coordinate (out3) at (add3-|out1) ;
	%
	\draw (copy1_out1) to (s1);
	\draw (copy1_out2) to (copy2_in1);
	\draw (copy2_out1) to (s2_in1);
	\draw (copy2_out2) to (s3_in1);
	\draw (copy3_out1) to (copy4_in1);
	\draw (copy4_out1) to (s4_in1);
	\draw (copy4_out2) to (s5_in1);
	\draw (copy3_out2) to (s6_in1);
	\draw (s1) to (add1_in1);
	\draw (s2_out1) to (add2_in1);
	\draw (s3_out1) to (add3_in1);
	\draw (s4_out1) to (add1_in2);
	\draw (s5_out1) to (add2_in2);
	\draw (s6_out1) to (add3_in2);
	\draw (add1_out1) to (out1);
	\draw (add2_out1) to (out2);
	\draw (add3_out1) to (out3);
\end{tikzpicture}
\end{aligned}
\]
\end{enumerate}
\item Here are graphical proofs that the representations we chose in our
solution to \cref{ex.drawsfgs} agree with those chosen in Part 1 above.
\begin{enumerate}
  \item
\[
      \begin{aligned}
\begin{tikzpicture}[oriented WD, spider diagram, bbx=3ex, bby=2ex]
	\node[spider={2}{1}, fill=white] (add2) {};
	\node[special spider={2}{1}{\leglen}{0}, fill=white, left=.2 of add2_in2] (add) {};
	\node[spider={1}{2}, fill=black,left=2 of add] (copy) {};
	\node[spider={1}{0}, fill=black, above =2 of add_in1] (unit) {};
	%
	\draw (add2_in1-|copy_in1) -- (add2_in1);
	\draw (unit_in1-|copy_in1) -- (unit_in1);
	\draw (copy_out1) -- (add_in1);
	\draw (copy_out2) -- (add_in2);
	\draw (add.0) -- (add2_in2);
\end{tikzpicture}
\end{aligned}
\quad=\quad
      \begin{aligned}
\begin{tikzpicture}[oriented WD, spider diagram, bbx=3ex, bby=2ex]
	\node[spider={2}{1}, fill=white] (add2) {};
	\node[special spider={2}{1}{\leglen}{0}, fill=white, left=.2 of add2_in2] (add) {};
	\node[special spider={1}{1}{.1}{0}, wamp,left=0 of add_in1] (scalar1) {$1$};	
	\node[special spider={1}{1}{.1}{0}, wamp,left=0 of add_in2] (scalar2) {$1$};	
	\node[spider={1}{2}, fill=black,left=3 of add] (copy) {};
	\node[spider={1}{0}, fill=black, above =2 of add_in1] (unit) {};
	%
	\draw (add2_in1-|copy_in1) -- (add2_in1);
	\draw (unit_in1-|copy_in1) -- (unit_in1);
	\draw (copy_out1) -- (scalar1.180);
	\draw (copy_out2) -- (scalar2.180);
	\draw (scalar1.0) -- (add_in1);
	\draw (scalar2.0) -- (add_in2);
	\draw (add.0) -- (add2_in2);
\end{tikzpicture}
\end{aligned}
\quad=\quad
      \begin{aligned}
\begin{tikzpicture}[oriented WD, spider diagram, bbx=3ex, bby=2ex]
	\node[spider={2}{1}, fill=white] (add) {};
	\node[spider={1}{1}, wamp,left=.2 of add_in2] (scalar) {$2$};	
	\node[spider={1}{0}, fill=black, above =1 of add_in1] (unit) {};
	%
	\draw (add_in1-|scalar_in1) -- (add_in1);
	\draw (unit_in1-|scalar_in1) -- (unit_in1);
\end{tikzpicture}
\end{aligned}
\]
\item
\[
      \begin{aligned}
\begin{tikzpicture}[oriented WD, spider diagram, bbx=3ex, bby=2ex]
	\node[spider={2}{1}, fill=white] (add) {};
	\node[special spider={1}{1}{.1}{0}, wamp, right=0 of add_out1]
	(scalar) {$0$};
	\node[spider={1}{2}, fill=black,right=.8 of scalar] (copy) {};
	%
	\draw (add_out1) -- (scalar.180);
	\draw (scalar.0) -- (copy_in1);
\end{tikzpicture}
\end{aligned}
\quad = \quad
      \begin{aligned}
\begin{tikzpicture}[oriented WD, spider diagram, bbx=3ex, bby=2ex]
	\node[spider={2}{1}, fill=white] (add) {};
	\node[special spider={1}{0}{.1}{0}, fill=black]	(counit) at (add_out1) {};
	\node[spider={1}{2}, fill=black,right=3 of add] (copy) {};
	\node[special spider={0}{1}{.1}{0}, fill=white]	(unit) at (copy_in1) {};
\end{tikzpicture}
\end{aligned}
\quad = \quad
      \begin{aligned}
\begin{tikzpicture}[oriented WD, spider diagram, bbx=3ex, bby=2ex]
	\node[spider={1}{0}, fill=black] (b1) {};
	\node[spider={1}{0}, fill=black, below=.5 of b1] (b2) {};
	\node[spider={0}{1}, fill=white, right=1 of b1] (w1) {};
	\node[spider={0}{1}, fill=white, below=.5 of w1] (w2) {};
\end{tikzpicture}
\end{aligned}
\]
\item
  \begin{align*}
    &
      \begin{aligned}
	\begin{tikzpicture}[oriented WD, spider diagram, bbx=3ex, bby=2ex,font=\small, scale=.8]
        \coordinate (s1) at (0,0) ;
	\node[spider={1}{1}, wamp,below=.3 of s1] (s2) {$2$};	
	\node[spider={1}{1}, wamp,below=.6 of s2] (s3) {$3$};	
	\node[spider={1}{1}, wamp,below=1.2 of s3] (s5) {$5$};	
	\node[spider={1}{1}, wamp] (s4) at ($(s3)!.5!(s5)+(4,0)$) {$4$};	
	\node[spider={1}{1}, wamp,below left=.4 of s5] (s6) {$6$};	
	\node[special spider={2}{1}{\leglen}{.2}, fill=white] (add2) at ($(s1)!.5!(s2)+(6,-1)$) {};
	\node[special spider={2}{1}{\leglen}{.2}, fill=white] (add1) at ($(s2)!.5!(s5)+(6,-.38)$) {};
	\node[special spider={2}{1}{\leglen}{.2}, fill=white] (add3) at ($(s5)!.5!(s6)+(4,0)$) {};
	\node[spider={1}{2}, fill=black] (copy1) at ($(s2)-(5,0)$) {};
	\node[spider={1}{2}, fill=black] (copy2) at ($(copy1_out2)+(1,0)$) {};
	\node[spider={1}{2}, fill=black] (copy3) at ($(s5)-(5,0)$) {};
	\node[spider={1}{2}, fill=black] (copy4) at ($(copy3_out1)+(1,0)$) {};
        \coordinate (out1) at ($(add2)+(2,0)$) ;
        \coordinate (out2) at ($(add1)+(2,0)$) ;
        \coordinate (out3) at (add3-|out1) ;
	%
	\draw (copy1_out1) to (s1);
	\draw (copy1_out2) to (copy2_in1);
	\draw (copy2_out1) to (s2_in1);
	\draw (copy2_out2) to (s3_in1);
	\draw (copy3_out1) to (copy4_in1);
	\draw (copy4_out1) to (s4_in1);
	\draw (copy4_out2) to (s5_in1);
	\draw (copy3_out2) to (s6_in1);
	\draw (s1) to (add1_in1);
	\draw (s2_out1) to (add2_in1);
	\draw (s3_out1) to (add3_in1);
	\draw (s4_out1) to (add1_in2);
	\draw (s5_out1) to (add2_in2);
	\draw (s6_out1) to (add3_in2);
	\draw (add1_out1) to (out1);
	\draw (add2_out1) to (out2);
	\draw (add3_out1) to (out3);
\end{tikzpicture}
\end{aligned}
= \quad
      \begin{aligned}
\begin{tikzpicture}[oriented WD, spider diagram, bbx=3ex, bby=2ex,font=\small, scale=.7]
        \coordinate (s1) at (0,0) ;
	\node[spider={1}{1}, wamp,below=.35 of s1] (s2) {$2$};	
	\node[spider={1}{1}, wamp,below=.35 of s2] (s3) {$3$};	
	\node[spider={1}{1}, wamp,below=.35 of s3] (s4) {$4$};	
	\node[spider={1}{1}, wamp,below=.35 of s4] (s5) {$5$};	
	\node[spider={1}{1}, wamp,below=.35 of s5] (s6) {$6$};	
	\node[special spider={2}{1}{\leglen}{.2}, fill=white] (add2) at ($(s1)!.5!(s2)+(5,0)$) {};
	\node[special spider={2}{1}{\leglen}{.2}, fill=white] (add1) at ($(s3)!.5!(s4)+(5,-.5)$) {};
	\node[special spider={2}{1}{\leglen}{.2}, fill=white] (add3) at ($(s5)!.5!(s6)+(5,0)$) {};
	\node[spider={1}{2}, fill=black] (copy1) at ($(s2)-(5,0)$) {};
	\node[spider={1}{2}, fill=black] (copy2) at ($(copy1_out2)+(1,0)$) {};
	\node[spider={1}{2}, fill=black] (copy3) at ($(s5)-(5,0)$) {};
	\node[spider={1}{2}, fill=black] (copy4) at ($(copy3_out1)+(1,0)$) {};
        \coordinate (out1) at ($(add2)+(2,0)$) ;
        \coordinate (out2) at ($(add1)+(2,0)$) ;
        \coordinate (out3) at ($(add3)+(2,0)$) ;
	%
	\draw (copy1_out1) to (s1);
	\draw (copy1_out2) to (copy2_in1);
	\draw (copy2_out1) to (s2_in1);
	\draw (copy2_out2) to (s3_in1);
	\draw (copy3_out1) to (copy4_in1);
	\draw (copy4_out1) to (s4_in1);
	\draw (copy4_out2) to (s5_in1);
	\draw (copy3_out2) to (s6_in1);
	\draw (s1) to (add1_in1);
	\draw (s2_out1) to (add2_in1);
	\draw (s3_out1) to (add3_in1);
	\draw (s4_out1) to (add1_in2);
	\draw (s5_out1) to (add2_in2);
	\draw (s6_out1) to (add3_in2);
	\draw (add1_out1) to (out1);
	\draw (add2_out1) to (out2);
	\draw (add3_out1) to (out3);
\end{tikzpicture}
\end{aligned}\\
\qquad&= \quad
      \begin{aligned}
\begin{tikzpicture}[oriented WD, spider diagram, bbx=3ex, bby=2ex,font=\small, scale=.7]
        \coordinate (s1) at (0,0) ;
	\node[spider={1}{1}, wamp,below=.3 of s1] (s2) {$2$};	
	\node[spider={1}{1}, wamp,below=.3 of s2] (s3) {$3$};	
	\node[spider={1}{1}, wamp,below=.3 of s3] (s4) {$4$};	
	\node[spider={1}{1}, wamp,below=.3 of s4] (s5) {$5$};	
	\node[spider={1}{1}, wamp,below=.3 of s5] (s6) {$6$};	
	\node[spider={2}{1}, fill=white] (add1) at ($(s1)!.5!(s2)+(5,0)$) {};
	\node[spider={2}{1}, fill=white] (add2) at ($(s3)!.5!(s4)+(5,0)$) {};
	\node[spider={2}{1}, fill=white] (add3) at ($(s5)!.5!(s6)+(5,0)$) {};
	\node[spider={1}{2}, fill=black] (copy1) at ($(s2)-(5,0)$) {};
	\node[spider={1}{2}, fill=black] (copy2) at ($(copy1_out2)+(1,0)$) {};
	\node[spider={1}{2}, fill=black] (copy3) at ($(s5)-(5,0)$) {};
	\node[spider={1}{2}, fill=black] (copy4) at ($(copy3_out1)+(1,0)$) {};
	%
	\draw (copy1_out1) to (s1);
	\draw (copy1_out2) to (copy2_in1);
	\draw (copy2_out1) to (s2_in1);
	\draw (copy2_out2) to (s3_in1);
	\draw (copy3_out1) to (copy4_in1);
	\draw (copy3_out2) to (s6_in1);
	\draw (copy4_out1) to (s4_in1);
	\draw (copy4_out2) to (s5_in1);
	\draw (s1) to (add1_in1);
	\draw (s2_out1) to (add2_in1);
	\draw (s3_out1) to (add3_in1);
	\draw (s4_out1) to (add1_in2);
	\draw (s5_out1) to (add2_in2);
	\draw (s6_out1) to (add3_in2);
\end{tikzpicture}
\end{aligned} = \quad
      \begin{aligned}
\begin{tikzpicture}[oriented WD, spider diagram, bbx=3ex, bby=2ex,font=\small, scale=.7]
        \coordinate (s1) at (0,0) ;
	\node[spider={1}{1}, wamp,below=.3 of s1] (s2) {$2$};	
	\node[spider={1}{1}, wamp,below=.3 of s2] (s3) {$3$};	
	\node[spider={1}{1}, wamp,below=.3 of s3] (s4) {$4$};	
	\node[spider={1}{1}, wamp,below=.3 of s4] (s5) {$5$};	
	\node[spider={1}{1}, wamp,below=.3 of s5] (s6) {$6$};	
	\node[spider={2}{1}, fill=white] (add1) at ($(s1)!.5!(s2)+(5,0)$) {};
	\node[spider={2}{1}, fill=white] (add2) at ($(s3)!.5!(s4)+(5,0)$) {};
	\node[spider={2}{1}, fill=white] (add3) at ($(s5)!.5!(s6)+(5,0)$) {};
	\node[spider={1}{2}, fill=black] (copy1) at ($(s2)-(5,0)$) {};
	\node[spider={1}{2}, fill=black] (copy2) at ($(copy1_out2)+(1,0)$) {};
	\node[spider={1}{2}, fill=black] (copy3) at ($(s5)-(5,0)$) {};
	\node[spider={1}{2}, fill=black] (copy4) at ($(copy3_out2)+(1,0)$) {};
	%
	\draw (copy1_out1) to (s1);
	\draw (copy1_out2) to (copy2_in1);
	\draw (copy2_out1) to (s2_in1);
	\draw (copy2_out2) to (s3_in1);
	\draw (copy3_out1) to (s4_in1);
	\draw (copy3_out2) to (copy4_in1);
	\draw (copy4_out1) to (s5_in1);
	\draw (copy4_out2) to (s6_in1);
	\draw (s1) to (add1_in1);
	\draw (s2_out1) to (add2_in1);
	\draw (s3_out1) to (add3_in1);
	\draw (s4_out1) to (add1_in2);
	\draw (s5_out1) to (add2_in2);
	\draw (s6_out1) to (add3_in2);
\end{tikzpicture}
\end{aligned}
\end{align*}
\end{enumerate}
\end{enumerate}
}

\sol{exc.proof_using_diag_lang}{
  Consider the signal flow graphs
  \begin{equation}\label{eqn.exc_two_sfgs}
  \begin{aligned}
    \begin{tikzpicture}[scale=.7]
	\begin{pgfonlayer}{nodelayer}
		\node [style=none] (0) at (4, -0) {};
		\node [style=none] (1) at (0, 1) {};
		\node [style=none] (2) at (0, -0) {};
		\node [style=none] (3) at (4, 1) {};
		\node [style=bdot] (4) at (1.5, 1) {};
		\node [style=wdot] (5) at (2.5, 1) {};
		\node [style=bdot] (6) at (1.5, -0) {};
		\node [style=wdot] (7) at (2.5, -0) {};
	\end{pgfonlayer}
	\begin{pgfonlayer}{edgelayer}
		\draw (1.center) to (4);
		\draw (5) to (3.center);
		\draw (2.center) to (6);
		\draw (7) to (0.center);
	\end{pgfonlayer}
\end{tikzpicture}
  \end{aligned}
  \quad \mbox{and} \quad
\begin{aligned}
  \begin{tikzpicture}[scale=.7]
	\begin{pgfonlayer}{nodelayer}
		\node [style=none] (3) at (-9, 2) {};
		\node [style=none] (22) at (-9, -0) {};
		\node [style=bdot] (21) at (-8, -0) {};
		\node [style=none] (20) at (-7, -0.5) {};
		\node [style=wamp] (23) at (-7, 0.625) {$\scriptstyle 3$};
		\node [style=wamp] (19) at (-6.5, 2) {$\scriptstyle 5$};
		\node [style=bdot] (26) at (-6.25, 0.625) {};
		\node [style=wamp] (18) at (-5.25, 2) {$\scriptstyle {3}$};
		\node [style=wamp] (24) at (-5.25, 1) {$\scriptstyle 3$};
		\node [style=none] (25) at (-5.25, 0.25) {};
		\node [style=none] (28) at (-4.25, 1) {};
		\node [style=none] (29) at (-4.25, 2) {};
		\node [style=wamp] (0) at (-4, 0.25) {$\scriptstyle 3$};
		\node [style=none] (1) at (-4, -0.5) {};
		\node [style=wdot] (27) at (-3.25, 1.5) {};
		\node [style=wdot] (2) at (-3, -0.125) {};
		\node [style=bdot] (4) at (-2.5, 1.5) {};
		\node [style=none] (5) at (-1.5, 1) {};
		\node [style=none] (7) at (-1.5, 2) {};
		\node [style=none] (16) at (-1.5, -0.125) {};
		\node [style=wamp] (8) at (-0.75, 2) {$\scriptstyle 5$};
		\node [style=wdot] (6) at (-0.5, 0.5) {};
		\node [style=bdot] (11) at (0.25, 0.5) {};
		\node [style=wamp] (14) at (0.5, 2) {$\scriptstyle {3}$};
		\node [style=none] (10) at (1.25, 1) {};
		\node [style=none] (12) at (1.25, -0) {};
		\node [style=none] (15) at (1.25, 2) {};
		\node [style=wdot] (9) at (2.25, 1.5) {};
		\node [style=none] (13) at (3, -0) {};
		\node [style=none] (17) at (3, 1.5) {};
	\end{pgfonlayer}
	\begin{pgfonlayer}{edgelayer}
		\draw [bend left, looseness=1.00] (0.center) to (2);
		\draw [bend right, looseness=0.9] (1.center) to (2);
		\draw (2) to (16.center);
		\draw [bend right, looseness=1.00] (16.center) to (6);
		\draw [bend right, looseness=1.00] (6) to (5.center);
		\draw [bend left, looseness=1.00] (5.center) to (4);
		\draw [bend left, looseness=1.00] (4) to (7.center);
		\draw (7.center) to (8.center);
		\draw (8.center) to (14.center);
		\draw (14.center) to (15.center);
		\draw [bend left, looseness=1.00] (15.center) to (9);
		\draw [bend left, looseness=1.00] (9) to (10.center);
		\draw (9) to (17.center);
		\draw [bend right, looseness=1.00] (10.center) to (11);
		\draw (6) to (11);
		\draw [bend right, looseness=1.00] (11) to (12.center);
		\draw (12.center) to (13.center);
		\draw (19) to (18);
		\draw (3.center) to (19);
		\draw (19) to (18);
		\draw (22.center) to (21);
		\draw [bend left, looseness=1.00] (21) to (23);
		\draw [bend right, looseness=1.00] (21) to (20.center);
		\draw [bend left, looseness=1.00] (26) to (24);
		\draw [bend right, looseness=0.9] (26) to (25.center);
		\draw (23) to (26);
		\draw (20.center) to (1.center);
		\draw [bend left, looseness=1.00] (29.center) to (27);
		\draw [bend left, looseness=1.00] (27) to (28.center);
		\draw (27) to (4);
		\draw (18) to (29.center);
		\draw (24) to (28.center);
		\draw (25.center) to (0);
	\end{pgfonlayer}
\end{tikzpicture}
\end{aligned}
  \end{equation}
  \begin{enumerate}
  	\item Let $R=(\NN,0,+,1,*)$. By examining the presentation of $\mat(R)$ in \cref{thm.presentation_mat}, and without computing the
  matrices that they represent, conjecture whether they represent the same matrix. How could you prove your conjecture?
  	\item Now suppose the rig is $R=\NN/3$; if you do not know what this means, just replace all 3's with 0's in the right-hand diagram of \cref{eqn.exc_two_sfgs}. Find what you would call a minimal representation of this diagram, using the presentation in \cref{thm.presentation_mat}.
\end{enumerate}
}{
\begin{enumerate}
	\item The signal flow graphs 
  \[
    \begin{aligned}
      \begin{tikzpicture}[scale=.7]
  	\begin{pgfonlayer}{nodelayer}
  		\node [style=none] (0) at (4, -0) {};
  		\node [style=none] (1) at (0, 1) {};
  		\node [style=none] (2) at (0, -0) {};
  		\node [style=none] (3) at (4, 1) {};
  		\node [style=bdot] (4) at (1.5, 1) {};
  		\node [style=wdot] (5) at (2.5, 1) {};
  		\node [style=bdot] (6) at (1.5, -0) {};
  		\node [style=wdot] (7) at (2.5, -0) {};
  	\end{pgfonlayer}
  	\begin{pgfonlayer}{edgelayer}
  		\draw (1.center) to (4);
  		\draw (5) to (3.center);
  		\draw (2.center) to (6);
  		\draw (7) to (0.center);
  	\end{pgfonlayer}
  \end{tikzpicture}
    \end{aligned}
    \quad \mbox{and} \quad
  \begin{aligned}
    \begin{tikzpicture}[scale=.7]
  	\begin{pgfonlayer}{nodelayer}
  		\node [style=none] (3) at (-9, 2) {};
  		\node [style=none] (22) at (-9, -0) {};
  		\node [style=bdot] (21) at (-8, -0) {};
  		\node [style=none] (20) at (-7, -0.5) {};
  		\node [style=wamp] (23) at (-7, 0.625) {$\scriptstyle 3$};
  		\node [style=wamp] (19) at (-6.5, 2) {$\scriptstyle 5$};
  		\node [style=bdot] (26) at (-6.25, 0.625) {};
  		\node [style=wamp] (18) at (-5.25, 2) {$\scriptstyle {3}$};
  		\node [style=wamp] (24) at (-5.25, 1) {$\scriptstyle 3$};
  		\node [style=none] (25) at (-5.25, 0.25) {};
  		\node [style=none] (28) at (-4.25, 1) {};
  		\node [style=none] (29) at (-4.25, 2) {};
  		\node [style=wamp] (0) at (-4, 0.25) {$\scriptstyle 3$};
  		\node [style=none] (1) at (-4, -0.5) {};
  		\node [style=wdot] (27) at (-3.25, 1.5) {};
  		\node [style=wdot] (2) at (-3, -0.125) {};
  		\node [style=bdot] (4) at (-2.5, 1.5) {};
  		\node [style=none] (5) at (-1.5, 1) {};
  		\node [style=none] (7) at (-1.5, 2) {};
  		\node [style=none] (16) at (-1.5, -0.125) {};
  		\node [style=wamp] (8) at (-0.75, 2) {$\scriptstyle 5$};
  		\node [style=wdot] (6) at (-0.5, 0.5) {};
  		\node [style=bdot] (11) at (0.25, 0.5) {};
  		\node [style=wamp] (14) at (0.5, 2) {$\scriptstyle {3}$};
  		\node [style=none] (10) at (1.25, 1) {};
  		\node [style=none] (12) at (1.25, -0) {};
  		\node [style=none] (15) at (1.25, 2) {};
  		\node [style=wdot] (9) at (2.25, 1.5) {};
  		\node [style=none] (13) at (3, -0) {};
  		\node [style=none] (17) at (3, 1.5) {};
  	\end{pgfonlayer}
  	\begin{pgfonlayer}{edgelayer}
  		\draw [bend left, looseness=1.00] (0.center) to (2);
  		\draw [bend right, looseness=0.9] (1.center) to (2);
  		\draw (2) to (16.center);
  		\draw [bend right, looseness=1.00] (16.center) to (6);
  		\draw [bend right, looseness=1.00] (6) to (5.center);
  		\draw [bend left, looseness=1.00] (5.center) to (4);
  		\draw [bend left, looseness=1.00] (4) to (7.center);
  		\draw (7.center) to (8.center);
  		\draw (8.center) to (14.center);
  		\draw (14.center) to (15.center);
  		\draw [bend left, looseness=1.00] (15.center) to (9);
  		\draw [bend left, looseness=1.00] (9) to (10.center);
  		\draw (9) to (17.center);
  		\draw [bend right, looseness=1.00] (10.center) to (11);
  		\draw (6) to (11);
  		\draw [bend right, looseness=1.00] (11) to (12.center);
  		\draw (12.center) to (13.center);
  		\draw (19) to (18);
  		\draw (3.center) to (19);
  		\draw (19) to (18);
  		\draw (22.center) to (21);
  		\draw [bend left, looseness=1.00] (21) to (23);
  		\draw [bend right, looseness=1.00] (21) to (20.center);
  		\draw [bend left, looseness=1.00] (26) to (24);
  		\draw [bend right, looseness=0.9] (26) to (25.center);
  		\draw (23) to (26);
  		\draw (20.center) to (1.center);
  		\draw [bend left, looseness=1.00] (29.center) to (27);
  		\draw [bend left, looseness=1.00] (27) to (28.center);
  		\draw (27) to (4);
  		\draw (18) to (29.center);
  		\draw (24) to (28.center);
  		\draw (25.center) to (0);
  	\end{pgfonlayer}
  \end{tikzpicture}
  \end{aligned}
  \]
cannot represent the same morphism because one has a path from a vertex on the left to one on the right, and the other does not. To prove this, observe that the only graphical equation in \cref{thm.presentation_mat} that breaks a path from left to right is the equation
\[
\begin{aligned}
\begin{tikzpicture}[spider diagram, leg length=20pt]
	\node[spider={1}{1}, wamp] (a) {$\scriptstyle \mathsf 0$};
\end{tikzpicture}
\end{aligned}
=
\begin{aligned}
\begin{tikzpicture}[spider diagram]
	\node[spider={1}{0}, fill=black] (a) {};
	\node[spider={0}{1}, fill=white, right=.3 of a] (b) {};
\end{tikzpicture}
\end{aligned}
\]
So a $0$ scalar must within a path from left to right before we could rewrite the diagram to break that path. No such $0$ scalar can appear, however, because the diagram does not contain any, and the sum and product of any two nonzero natural numbers is always nonzero.
\item Replacing each of the $3$s with $0$ allows us to rewrite the diagram to
\[
\begin{tikzpicture}[spider diagram, leg length=20pt]
  \node[spider={1}{0}, fill=black] (a) {};
  \node[spider={1}{2}, fill=black, below=.25 of a] (b) {};
  \draw (a_in1|-b_in1) to (b_in1);
\end{tikzpicture}
\]
\end{enumerate}
}

\sol{exc.two_monoids_on_reals}{
Consider the set $\RR$ of real numbers.
	\begin{enumerate}
		\item Show that if $\mu\colon\RR\times\RR\to\RR$ is defined by $\mu(a,b)=a*b$ and if $\eta\in\RR$ is defined to be $\eta=1$, then $(\RR,*,1)$ satisfies all three conditions of \cref{def.monoid_object}.
		\item Show that if $\mu\colon\RR\times\RR\to\RR$ is defined by $\mu(a,b)=a+b$ and if $\eta\in\RR$ is defined to be $\eta=0$, then $(\RR,+,0)$ satisfies all three conditions of \cref{def.monoid_object}.
\end{enumerate}
}{
The three conditions of \cref{def.monoid_object} are
\begin{enumerate}[label=(\alph*)]
  \item $(\mu \otimes \id)\cp\mu = (\id \otimes \mu)\cp\mu$,
  \item $(\eta \otimes \id)\cp\mu = \id = (\id \otimes \eta)\cp\mu$, and
  \item $\sigma_{M,M}\cp\mu = \mu$.
\end{enumerate}
where $\sigma_{M,M}$ is the swap map on $M$ in $\cat{C}$.
\begin{enumerate}
	\item Suppose $\mu\colon\RR\times\RR\to\RR$ is defined by $\mu(a,b)=a*b$ and  $\eta\in\RR$ is defined to be $\eta=1$. The conditions, written diagrammatically, say that starting in the upper left of each diagram below, the result in the lower right is the same regardless of which path you take:
	\[
	\begin{tikzcd}[ampersand replacement=\&, column sep=30pt]
		(a,b,c)\ar[r, |->, "(\mu\otimes\id)"]\ar[d, |->, "(\id\otimes\mu)"']\&
		(a*b, c)\ar[d, |->, "\mu"]\\
		(a,b*c)\ar[r, |->, "\mu"']\&
		a*b*c
	\end{tikzcd}
	\qquad
	\begin{tikzcd}[ampersand replacement=\&, column sep=30pt]
		a\ar[r, |->, "(\eta\otimes\id)"]\ar[d, |->, "(\id\otimes\eta)"']\ar[dr, |->, "\id_a" description]\&
		(1, a)\ar[d, |->, "\mu"]\\
		(a, 1)\ar[r, |->, "\mu"']\&
		a
	\end{tikzcd}	
	\qquad
	\begin{tikzcd}[ampersand replacement=\&]
		(a,b)\ar[r, |->, "\sigma"]\ar[dr, |->, "\mu"']\&
		(b, a)\ar[d, |->, "\mu"]\\\&
		a*b
	\end{tikzcd}	
	\]
	and this is true for $(\rr,*,1)$.
	\item The same reasoning works for $(\rr,+,0)$, shown below:
	\[
	\begin{tikzcd}[ampersand replacement=\&, column sep=30pt]
		(a,b,c)\ar[r, |->, "(\mu\otimes\id)"]\ar[d, |->, "(\id\otimes\mu)"']\&
		(a{+}b, c)\ar[d, |->, "\mu"]\\
		(a,b{+}c)\ar[r, |->, "\mu"']\&
		a{+}b{+}c
	\end{tikzcd}
	\qquad
	\begin{tikzcd}[ampersand replacement=\&, column sep=30pt]
		a\ar[r, |->, "(\eta\otimes\id)"]\ar[d, |->, "(\id\otimes\eta)"']\ar[dr, |->, "\id_a" description]\&
		(0, a)\ar[d, |->, "\mu"]\\
		(a, 0)\ar[r, |->, "\mu"']\&
		a
	\end{tikzcd}	
	\qquad
	\begin{tikzcd}[ampersand replacement=\&]
		(a,b)\ar[r, |->, "\sigma"]\ar[dr, |->, "\mu"']\&
		(b, a)\ar[d, |->, "\mu"]\\\&
		a{+}b
	\end{tikzcd}	
	\]
	\end{enumerate}
}

\sol{exc.check_monoid_obj}{
Recall that in $\mat(R)$, the monoidal unit is $0$ and the monoidal product is $+$, because it is a prop. Recall also that in (the usual monoidal structure on) $\smset$, the monoidal unit is $\{1\}$, a set with one element, and the monoidal product is $\times$ (see \cref{ex.set_as_mon_cat}).

\begin{enumerate}
	\item Check that the functor $U\colon\mat(R)\to\smset$, defined above, preserves the monoidal unit and the monoidal product.
	\item Find a function $p\colon R^n\times R^n\to R^n$ such that $p(0,v)=v$ and $p(v,0)=v$ for any $v\in R^n$.
	\item Use your $p$ to show that if $(M,\mu,\eta)$ is a monoid object in $\mat(R)$ then $(U(M),U(\mu),U(\eta))$ is a monoid object in $\smset$. (This works for any monoidal functor---which we will define in \cref{roughdef.monoidal_functor}---not just for $U$ in particular.)
	\item In \cref{ex.diagrammatic_monoid_obj}, we said that the triple
  $(1,\add{1em},\zero{1em})$ is a commutative monoid object in the prop $\mat(R)$. If $R=\RR$ is the rig of real numbers, this means that we have a monoid structure on the set $\RR$. But in \cref{exc.two_monoids_on_reals} we gave two such monoid structures. Which one is it?
\end{enumerate}
}{
The functor $U\colon\mat(R)\to\smset$ is given on objects by sending $n$ to the set $R^n$, and on morphisms by matrix-vector multiplication. Here $R^n$ means the set of $n$-tuples or $n$-dimensional vectors in $R$. In particular, $R^0=\{()\}$ consists of a single vector of dimension 0.
\begin{enumerate}
	\item $U$ preserves the monoidal unit because $0$ is the monoidal unit of any prop ($\mat(R)$ is a prop), $\{1\}$ is the monoidal unit of $\smset$, and $R^0$ is canonically isomorphic to $\{1\}$. $U$ also preserves the monoidal product because there is a canonical isomorphism $R^m\times R^n\cong R^{m+n}$.
	\item A monoid object in $\mat(R)$ is a tuple $(m,\mu,\eta)$ where $m\in\nn$, $\mu\colon m+m\to m$, and $\eta\colon 0\to m$ satisfy the properties $\mu(\eta,x)=x=\mu(x,\eta)$ and $\mu(x,\mu(y,z))=\mu(\mu(x,y),z)$. Note that there is only one morphism $0\to m$ in $\mat(R)$ for any $m$. It is not hard to show that for any $m\in\nn$ there is only one monoid structure. For example, when $m=2$, $\mu$ must be the following matrix
	\[
	\mu\coloneqq
  	\left(
  	\begin{array}{cc}
  		1&0\\
  		0&1\\
  		1&0\\
  		0&1
  	\end{array}
  	\right)
	\]	
	Anyway, for any monoid $(m,\mu,\eta)$, the morphism $U(\eta)\colon R^0\to R^m$ is given by $U(\eta)(1)\coloneqq (0,\ldots,0)$, and the morphism $U(\mu)\colon R^m\times R^m\to R^m$ is given by
	\[
		U(\mu)((a_1,\ldots,a_m),(b_1,\ldots,b_m))\coloneqq (a_1+b_1,\ldots,a_m+b_m).
	\]
	These give $R^m$ the structure of a monoid.
	\item The triple $(1,\add{1em},\zero{1em})$ corresponds to the additive monoid structure on $\rr$, e.g.\ with $(5,3)\mapsto 8$.
\end{enumerate}
}

\sol{exc.understand_reversed_icons}{
\begin{enumerate}
	\item What is the behavior $\beh(\coadd{1.3em})$ of the reversed addition icon $\coadd{1.3em}\colon 1 \to 2$?
	\item What is the behavior $\beh(\mult{1.3em})$ of the reversed copy icon, $\mult{1.3em}\colon 2\to 1$?
\end{enumerate}
}{
\begin{enumerate}
	\item The behavior $\beh(\coadd{1.3em})$ of the reversed addition icon $\coadd{1.3em}\colon 1 \to 2$ is the relation $\{(x,y,z)\in R^3\mid x=y+z\}$.
	\item The behavior $\beh(\mult{1.3em})$ of the reversed copy icon, $\mult{1.3em}\colon 2\to 1$ is the relation $\{(x,y,z)\in R^3\mid x=y=z\}$.
\end{enumerate}
}


\sol{exc.monoidal_prod_+}{
In \cref{def.rel_prop} we went quickly through monoidal products $+$ in the prop $\rel_R$. If $B\ss R^m\times R^n$ and $C\ss R^p\times R^q$ are morphisms in $\rel_R$, write down $B+C$ in set-notation.}
{
If $B\ss R^m\times R^n$ and $C\ss R^p\times R^q$ are morphisms in $\rel_R$, then take $B+C\ss R^{m+p}\times R^{n+q}$ to be the set
\[
	B+C\coloneqq\{(w,y,x,z)\in R^{m+p}\times R^{n+q}\mid (w,x)\in B\text{ and }(y,z)\in C\}.
\]
}


\sol{exc.SFG_composite_behavior}{
Let $g\colon m \to n$, $h\colon \ell \to n$ be signal flow graphs. Note that
$h\op\colon n \to \ell$ is a signal flow graph, and we can form the
composite $g\cp h\op$:
\[
  \begin{tikzpicture}[oriented WD, bb port length=0pt, bb port sep=1pt]
    \node[bb={4}{4}, minimum width = 2cm] (X) 
    {$\begin{array}{c} \longrightarrow \\ g \end{array}$};
	\node[bb={4}{4}, right= 1 of X, minimum width = 2cm] (Y)
    {$\begin{array}{c} \longleftarrow \\ h\op \end{array}$};
	\draw ($(X_in1)-(.2,0)$) to (X_in1);
	\draw ($(X_in2)-(.2,0)$) to (X_in2);
	\draw ($(X_in4)-(.2,0)$) to (X_in4);
	\draw (X_out1) to (Y_in1);
	\draw (X_out2) to (Y_in2);
	\draw (X_out4) to (Y_in4);
	\draw (Y_out1) to ($(Y_out1)+(.2,0)$);
	\draw (Y_out2) to ($(Y_out2)+(.2,0)$);
	\draw (Y_out4) to ($(Y_out4)+(.2,0)$);
	\draw[label]
		node at ($.5*(X_out3)+.5*(Y_in3)$) {$\vdots$}
		node[left=3pt of X_in3] {$\vdots$}
		node[right=3pt of Y_out3] {$\vdots$}
	;	
\end{tikzpicture}
\]
Show that the behavior of $g\cp(h\op)$ is equal to 
\[
  \beh(g\cp(h\op))=\{(x,y)\,\mid\, S(g)(x)=S(h)(y)\}\ss R^m\times R^\ell.
\]
}{
The behavior of $g\colon m\to n$ and $h\op\colon n\to\ell$ are respectively
\begin{align*}
	\beh(g)&=\{(x,z)\in R^m\times R^n\,\mid\, S(g)(x)=z\}\\
	\beh(h\op)&=\{(z,y)\in R^n\times R^\ell\,\mid\,z=S(h)(y)\}
\end{align*}
and by \cref{eqn.comp_rule_rels}, the composite $\beh(g\cp (h\op))=\beh(g)\cp\beh(h\op)$ is:
\[
  \{(x,y)\mid\text{ there exists } z\in R^n\text{ such that } S(g)(x)=z\text{ and }z=S(h)(y)\}.
\]
Since $S(g)$ and $S(h)$ are functions, the above immediately reduces to the desired formula:
\[  \beh(g\cp(h\op))=\{(x,y)\,\mid\, S(g)(x)=S(h)(y)\}.\]
}

\sol{exc.sfg_behavior_again}{
Let $g\colon m \to n$, $h\colon m \to p$ be signal flow graphs. Note that
$g\op\colon n \to m$ is a signal flow graph, and we can form the
composite $g\op\cp h$
\[
  \begin{tikzpicture}[oriented WD, bb port length=0pt, bb port sep=1pt]
    \node[bb={4}{4}, minimum width = 2cm] (X) 
    {$\begin{array}{c} \longleftarrow \\ g\op \end{array}$};
	\node[bb={4}{4}, right= 1 of X, minimum width = 2cm] (Y)
    {$\begin{array}{c} \longrightarrow \\ h \end{array}$};
	\draw ($(X_in1)-(.2,0)$) to (X_in1);
	\draw ($(X_in2)-(.2,0)$) to (X_in2);
	\draw ($(X_in4)-(.2,0)$) to (X_in4);
	\draw (X_out1) to (Y_in1);
	\draw (X_out2) to (Y_in2);
	\draw (X_out4) to (Y_in4);
	\draw (Y_out1) to ($(Y_out1)+(.5,0)$);
	\draw (Y_out2) to ($(Y_out2)+(.5,0)$);
	\draw (Y_out4) to ($(Y_out4)+(.5,0)$);
	\draw[label]
		node at ($.5*(X_out3)+.5*(Y_in3)$) {$\scriptstyle\vdots$}
		node[above left=-4pt and 3pt of X_in3] {$\vdots$}
		node[above right=-4pt and 3pt of Y_out3] {$\vdots$}
	;	
\end{tikzpicture}
\]
Show that the behavior of $g\op\cp h$ is equal to 
\[
  \{(S(g)(x),S(h)(x))\,\mid\, x \in R^m\}.
\]
}{
The behavior of $g\op\colon n\to m$ and $h\colon m\to p$ are respectively
\begin{align*}
	\beh(g\op)&=\{(y,x)\in R^n\times R^m\,\mid\, y=S(g)(x)\}\\
	\beh(h)&=\{(x,z)\in R^m\times R^p\,\mid\,S(h)(x)=z\}
\end{align*}
and by \cref{eqn.comp_rule_rels}, the composite $\beh((g\op)\cp h)=\beh(g\op)\cp\beh(h)$ is:
\[
  \{(y,z)\mid\text{ there exists } x\in R^m\text{ such that } y=S(g)(x)\text{ and }S(h)(x)=z\}.
\]
This immediately reduces to the desired formula:
\[
  \beh((g\op)\cp h)=\{(S(g)(x),S(h)(x))\,\mid\, x \in R^m\}.
\]
}


\sol{exc.linear_relations}{
Here is an exercise for those that know linear algebra. Let $R$ be a field, and $g\colon
m \to n$ a signal flow graph and let $S(g)\in\mat(R)$ be the associated $(m\times n)$-matrix (see \cref{thm.sfg_to_mat}).
\begin{enumerate}
  \item Show that the composite of $g$ with $0$-reverses, shown here
  \[
    \begin{tikzpicture}[oriented WD, bb port length=0pt, bb port sep=1pt]
      \node[bb={4}{4}, minimum width = 2cm] (X) 
      {$\begin{array}{c} \longrightarrow \\ g \end{array}$};
  	\draw ($(X_in1)-(.5,0)$) to (X_in1);
  	\draw ($(X_in2)-(.5,0)$) to (X_in2);
  	\draw ($(X_in4)-(.5,0)$) to (X_in4);
  	\draw (X_out1) to ($(X_out1)+(.5,0)$);
  	\draw (X_out2) to ($(X_out2)+(.5,0)$);
  	\draw (X_out4) to ($(X_out4)+(.5,0)$);
  	\draw[label]
  		node[above left=-4pt and 3pt of X_in3] {$\vdots$}
  		node[above right=-4pt and 3pt of X_out3] {$\vdots$}
  		node[wdot] at ($(X_out1)+(.5,0)$) {}
  		node[wdot] at ($(X_out2)+(.5,0)$) {}
  		node[wdot] at ($(X_out4)+(.5,0)$) {}
  	;	
  \end{tikzpicture}
  \]
  is equal to the kernel of the matrix $S(g)$.
  \item Show that the composite of discard-reverses with $g$, shown here 
  \[
    \begin{tikzpicture}[oriented WD, bb port length=0pt, bb port sep=1pt]
      \node[bb={4}{4}, minimum width = 2cm] (X) 
      {$\begin{array}{c} \longrightarrow \\ g \end{array}$};
  	\draw ($(X_in1)-(.5,0)$) to (X_in1);
  	\draw ($(X_in2)-(.5,0)$) to (X_in2);
  	\draw ($(X_in4)-(.5,0)$) to (X_in4);
  	\draw (X_out1) to ($(X_out1)+(.2,0)$);
  	\draw (X_out2) to ($(X_out2)+(.2,0)$);
  	\draw (X_out4) to ($(X_out4)+(.2,0)$);
  	\draw[label]
  		node[above left=-4pt and 3pt of X_in3] {$\vdots$}
  		node[above left=-4pt and 3pt of X_out3] {$\vdots$}
  		node[bdot] at ($(X_in1)-(.5,0)$) {}
  		node[bdot] at ($(X_in2)-(.5,0)$) {}
  		node[bdot] at ($(X_in4)-(.5,0)$) {}
  	;	
  \end{tikzpicture}
  \]
  is equal to the image of the matrix $S(g)$.
  \item Show that for any signal flow graph $g$, the subset
  $\beh(g)\subseteq R^m \times R^n$ is a linear subspace. That is, if $b_1,b_2\in \beh(g)$ then so are $b_1+b_2$ and $r*b_1$, for any $r\in R$.
\end{enumerate}
}{
\begin{enumerate}
	\item The behavior of the 0-reverse $\cozero{1.5em}$ is the subset $\{y\in R\mid y=0\}$, and its $n$-fold tensor is similarly $\{y\in R^n\mid y=0\}$. Composing this relation with $S(g)\ss R^m\times R^n$ gives $\{x\in R^m\mid S(g)=0\}$, which is the kernel of $S(g)$.
	\item The behavior of the discard-inverse $\unit{1.5em}$ is the subset $\{x\in R\}$, i.e.\ the largest subset of $R$, and similarly its $m$-fold tensor is $R^n\ss R^n$. Composing this relation with $S(g)\ss R^m\times R^n$ gives $\{y\in R^n\mid \text{ there exists }x\in R^m\text{ such that }S(g)(x)=y\}$, which is exactly the image of $S(g)$.
	\item For any $g\colon m\to n$, we first claim that the behavior $\beh(g)=\{(x,y)\mid S(g)(x)=y\}$ is linear, i.e.\ it is closed under addition and scalar multiplication. Indeed, $S(g)$ is multiplication by a matrix, so if $S(g)(x)=y$ then $S(g)(rx)=ry$ and $S(g)(x_1+x_2)=S(g)(x_1)+S(g)(x_2)$. Thus we conclude that $(x,y)\in\beh(g)$ implies $(rx,ry)\in\beh(g)$, so it's closed under scalar multiplication, and $(x_1,y_1),(x_2,y_2)\in\beh(g)$ implies $(x_1+x_2,y_1+y_2)\in\beh(g)$ so it's closed under addition. Similarly, the behavior $\beh(g\op)$ is also linear; the proof is similar.\\
	
	Finally, we need to show that the composite of any two linear relations is linear. Suppose that $B\ss R^m\times R^n$ and $C\ss R^n\times R^p$ are linear. Take $(x_1,z_1),(x_2,z_2)\in B\cp C$ and take $r\in R$. By definition, there exist $y_1,y_2\in R^n$ such that $(x_1,y_1),(x_2,y_2)\in B$ and $(y_1,z_1),(y_2,z_2)\in C$. Since $B$ and $C$ are linear, $(rx_1,ry_1)\in B$ and $(ry_1,rz_1)\in C$, and also $(x_1+x_2,y_1+y_2)\in B$ and $(y_1+y_2,z_1+z_2)\in C$. Hence $(rx_1,rz_1)\in (B\cp C)$ and $(x_1+x_2,z_1+z_2)\in (B\cp C)$, as desired.
\end{enumerate}
}

\sol{exc.linrel_prop}{
One might want to show that linear relations on $R$ form a prop, $\Cat{LinRel}_R$. That is, one might want to show that there is a sub-prop of the prop $\rel_R$ from \cref{def.rel_prop}, where the morphisms $m\to n$ are the subsets $B\ss R^m\times R^n$ such that $B$ is linear. In other words, where for any $(x,y)\in B$ and $r\in R$, the element $(r*x,r*y)\in R^m\times R^n$ is in $B$, and for any $(x',y')\in B$, the element $(x+x',y+y')$ is in $B$.

This is certainly doable, but for this exercise, we only ask that you prove that the composite of two linear relations is linear.
}
{
Suppose that $B\ss R^m\times R^n$ and $C\ss R^n\times R^p$ are linear. Their composite is the relation $(B\cp C)\ss R^m\times R^p$ consisting of all $(x,z)$ for which there exists $y\in R^n$ with $(x,y)\in B$ and $(y,z)\in C$. We want to show that the set $(B\cp C)$ is linear, i.e.\ closed under scalar multiplication and addition.

For scalar multiplication, take an $(x,z)\in (B\cp C)$ and any $r\in R$. Since $B$ is linear, we have $(r*x,r*y)\in B$ and since $C$ is linear we have $(r*y,r*z)\in C$, so then $(r*x,r*z)\in (B\cp C)$. For addition, if we also have $(x',z')\in(B\cp C)$ then there is some $y'\in R^n$ with $(x',y')\in B$ and $(y',z')\in C$, so since $B$ and $C$ are linear we have $(x+x',y+y')\in B$ and $(y+y',z+z')\in C$, hence $(x+x',z+z')\in (B\cp C)$.
}
\finishSolutionChapter
%======== Section =========%
\section[Solutions for Chapter 6]{Solutions for \cref{chap.hypergraph_cats}.}

\sol{exc.initial_ob_practice}{
Consider the set $A=\{a,b\}$. Find a preorder relation $\leq$ on $A$ such that
\begin{enumerate}
	\item $(A,\leq)$ has no initial object.
	\item $(A,\leq)$ has exactly one initial object.
	\item $(A,\leq)$ has two initial objects.
\end{enumerate}
}
{
Let $A=\{a,b\}$, and consider the preorders shown here: \fbox{$\LMO{a}\quad\LMO{b}$}\,, \quad \fbox{$\LMO{a}\to\LMO{b}$}\,, \quad \fbox{$\LMO{a}\leftrightarrows\LMO{b}$}\,.
\begin{enumerate}
	\item The left-most (the discrete preorder on $A$) has no initial object, because $a\not\leq b$ and $b\not\leq a$.\index{preorder!discrete}
	\item The middle one has one initial object, namely $a$.
	\item The right-most (the co-discrete preorder on $A$) has two initial objects.\index{preorder!codiscrete}
\end{enumerate}
}

\sol{exc.initial_object}{
For each of the graphs below, consider the free category on that graph, and say whether it has an initial object.\\
\begin{enumerate*}[itemjoin=\hspace{.8in}]
	\item \fbox{$\LMO{a}$}
	\item \fbox{$\LMO{a}\to\LMO{b}\to\LMO{c}$}
	\item \fbox{$\LMO{a}\qquad\LMO{b}$}
	\item \fbox{$\begin{tikzcd}\LMO{a}\ar[loop right]\end{tikzcd}$}	
\end{enumerate*}
}{
Recall that the objects of a free category on a graph are the vertices of the
graph, and the morphisms are paths. Thus the free category on a graph $G$ has an
initial object if there exists a vertex $v$ that has a unique path to every
object. In 1.\ and 2., the vertex $a$ has this property, so the free categories
on graphs 1.\ and 2.\ have initial objects. In graph 3., neither $a$ nor $b$
have a path to each other, and so there is no initial object. In graph 4., the
vertex $a$ has many paths to itself, and hence its free category does not have
an initial object either.
}

\sol{exc.rig_initial_obj}{
Recall the notion of rig from \cref{chap.SFGs}. A \emph{rig homomorphism} from $(R,0_R,+_R,1_R,*_R)$ to $(S,0_S,+_S,,1_S, *_S)$ is a function $f\colon R\to S$ such that $f(0_R)=0_S$, $f(r_1+_Rr_2)=f(r_1)+_Sf(r_2)$, etc.
\begin{enumerate}
	\item We said ``etc.'' Guess the remaining conditions for $f$ to be a rig homomorphism.
	\item Let $\Cat{Rig}$ denote the category whose objects are rigs and whose morphisms are rig homomorphisms. We claim $\Cat{Rig}$ has an initial object. What is it?
\end{enumerate}	
}{
\begin{enumerate}
\item The remaining conditions are that $f(1_R)=1_S$, and that $f(r_1\ast_R r_2)
= f(r_1) \ast_S f(r_2)$.
\item The initial object in the category $\Cat{Rig}$ is the natural numbers rig
$(\nn,0,+,1,\ast)$. The fact that is initial means that for any other rig
$R=(R,0_R,+_R,1_R,\ast_R)$, there is a unique rig homomorphism $f\colon \nn \to
R$. 

What is this homomorphism? Well, to be a rig homomorphism, $f$ must send
$0$ to $0_R$, $1$ to $1_R$. Furthermore, we must also have $f(m + n) =
f(m) +_Rf(n)$, and hence 
\[
f(m) = f(\underbrace{1+1+\dots+1}_{m \text{ summands}}) =
\underbrace{f(1)+f(1)+\dots+f(1)}_{m \text{ summands}} = \underbrace{1_R+1_R+
\dots+1_R}_{m \text{ summands}}.
\]
So if there is a rig homomorphism $f\colon\nn\to R$, it must be given by the above formula. But does this formula work correctly for multiplication?

It remains to check $f(m \ast n) = f(m) \ast_R f(n)$, and this will follow from
distributivity. Noting that $f(m\ast n)$ is equal to the sum of $mn$ copies of
$1_R$, we have
\begin{align*}
f(m) \ast_R f(n) &= (\underbrace{1_R+\dots+1_R}_{m \text{ summands}}) \ast_R
(\underbrace{1_R+\dots+1_R}_{n \text{ summands}}) \\
&= \underbrace{1_R \ast (\underbrace{1_R+\dots+1_R}_{n \text{ summands}}) + \dots
+1_R\ast (\underbrace{1_R+\dots+1_R}_{n \text{ summands}})}_{m \text{ summands}} \\
&= \underbrace{1_R+\dots+1_R}_{mn \text{ summands}} = f(m\ast n).
\end{align*}
Thus $(\nn,0,+,1,\ast)$ is the initial object in $\Cat{Rig}$.
\end{enumerate}
}

\sol{exc.universality}{
Explain the statement ``the hallmark of universality is the existence of a
unique map to any other comparable object,'' in the context of
\cref{def.initial_obj}. In particular, what is universal? What is the
``comparable object''?
}{
In \cref{def.initial_obj}, it is the initial object $\varnothing\in\cat{C}$ that is
universal. In this case, all objects $c\in\cat{C}$ are `comparable objects'.
So the universal property of the initial object is that to any \emph{object} $c\in\cat{C}$,
there is a unique map $\varnothing\to c$ coming from the initial object.
}

\sol{exc.initials_are_isomorphic}{
Let $\cat{C}$ be a category, and suppose that $c_1$ and $c_2$ are initial objects. Find an isomorphism between them, using the universal property from \cref{def.initial_obj}.
}
{
If $c_1$ is initial then by the universal property, for any $c$ there is a unique morphism $c_1\to c$; in particular, there is a unique morphism $c_1\to c_2$, call it $f$. Similarly, if $c_2$ is initial then there is a unique morphism $c_2\to c_1$, call it $g$. But how do we know that $f$ and $g$ are mutually inverse? Well since $c_1$ is initial there is a unique morphism $c_1\to c_1$. But we can think of two: $\id_{c_1}$ and $f\cp g$. Thus they must be equal. Similarly for $c_2$, so we have $f\cp g=\id_{c_1}$ and $g\cp f=\id_{c_2}$, which is the definition of $f$ and $g$ being mutually inverse.
}

\sol{exc.join_as_coproduct}{
  Explain why coproducts in a preorder are the same as joins.
}{
Let $(P,\le)$ be a preorder, and $p,q \in P$. Recall that a preorder is a
category with at most one morphism, denoted $\le$, between any two objects. 
Also recall that all diagrams in a preorder commute, since this means any two
morphisms with the same domain and codomain are equal. 

Translating \cref{def.coproduct} to this case, a coproduct $p+q$ is $P$ is an
element of $P$ such that $p \le p+q$ and $q\le p+q$, and such that for all elements $x
\in P$ with maps $p \le x$ and $q \le x$, we have $p+q \le x$. But this says
exactly that $p+q$ is a join: it is a least element above both $p$ and
$q$. Thus coproducts in preorders are exactly the same as joins.
}

\sol{exc.copairing}{
	Suppose $T=\{a,b,c,\ldots,z\}$ is the set of letters in the alphabet, and let $A$ and $B$ be the sets from \cref{eqn.apples_oranges_again}. Consider the function $f\colon A\to T$ sending each element of $A$ to the first letter of its label, e.g.\ $f(\mathrm{apple})=a$. Let $g\colon B\to T$ be the function sending each element of $B$ to the last letter of its label, e.g.\ $g(\mathrm{apple})=e$. Write down the function $\copair{f,g}(x)$ for all elements of $A\sqcup B$.
}{
The function $\copair{f,g}$ is defined by
\begin{align*}
\copair{f,g}\colon A \sqcup B &\longrightarrow T \\
\mathrm{apple1} &\longmapsto a \\
\mathrm{banana1} &\longmapsto b \\
\mathrm{pear1} &\longmapsto p \\
\mathrm{cherry1} &\longmapsto c \\
\mathrm{orange1} &\longmapsto o \\
\mathrm{apple2} &\longmapsto e \\
\mathrm{tomato2} &\longmapsto o \\
\mathrm{mango2} &\longmapsto o.
\end{align*}
}

\sol{exc.coprod_properties}{
  Let $f \colon A \to C$, $g \colon B \to C$, and $h\colon C \to D$ be morphisms
  in a category $\cat{C}$ with coproducts. Show that
  \begin{enumerate}
  \item $\iota_A \cp \copair{f,g} = f$.
  \item $\iota_B \cp \copair{f,g} = g$.
  \item $\copair{f,g}\cp h = \copair{f \cp h,g\cp h}$. 
  \item $\copair{\iota_A,\iota_B} = \id_{A+B}$.
  \end{enumerate}
}{
  \begin{enumerate}
  \item The equation $\iota_A \cp \copair{f,g} = f$ is the commutativity of the
  left hand triangle in the commutative diagram
  \eqref{eqn.universal_prop_coprod} defining $\copair{f,g}$.
  \item The equation $\iota_B \cp \copair{f,g} = g$ is the commutativity of the
  right hand triangle in the commutative diagram
  \eqref{eqn.universal_prop_coprod} defining $\copair{f,g}$.
  \item The equation $\copair{f,g}\cp h = \copair{f \cp h,g\cp h}$ follows from
  the universal property of the coproduct. Indeed, the diagram
  \[
    \begin{tikzcd}[ampersand replacement=\&, column sep=large, cramped]
      A \ar[r,"\iota_A"] \ar[dr,"f"' near end]  \ar[ddr, "f\cp h"']
      \& A+B \ar[d, "\copair{f,g}" description] 
      \& B \ar[l,"\iota_B"'] \ar[dl, "g" near end] \ar[ddl, "g\cp h"] \\[10pt]
      \& C \ar[d, "h" description] \\[10pt]
      \& D
    \end{tikzcd}
  \]
  commutes, and the universal property says there is a unique map $\copair{f\cp
  h, g \cp h}\colon A+B \to D$ for which this occurs. Hence we must have $[f,g]
  \cp h = \copair{f \cp h, g \cp h}$. 
  \item Similarly, to show $\copair{\iota_A,\iota_B} = \id_{A+B}$, observe that
  the diagram 
  \[
    \begin{tikzcd}[ampersand replacement=\&, cramped]
      A \ar[r,"\iota_A"] \ar[dr,"\iota_A"'] \& A+B \ar[d,equal, "\id_{A+B}" description] \& B
      \ar[l,"\iota_B"'] \ar[dl, "\iota_B"] \\[15pt]
      \& A+B
    \end{tikzcd}
  \]
  trivially commutes. Hence by the uniqueness in
  \eqref{eqn.universal_prop_coprod}, $\copair{\iota_{A}, \iota_{B}}=\id_{A+B}$.
  \end{enumerate}
}

\sol{exc.coproducts_give_monoidal_structure}{
Suppose a category $\cat C$ has coproducts, denoted $+$, and an initial object,
denoted $\varnothing$. Then we can show $(\cat C,+,\varnothing)$ is a symmetric
monoidal category (recall \cref{rdef.sym_mon_cat}). In this exercise we develop
the data.
\begin{enumerate}
\item Show that $+$ extends to a functor $\cat C \times \cat C \to \cat C$. In particular, how does it act on morphisms in $\cat{C}\times\cat{C}$?
\item Using the universal properties of the initial object and coproduct, show
that there are isomorphisms $A+\varnothing \to A$ and $\varnothing +A \to A$.
\item Using the universal property of the coproduct, write down morphisms
\begin{enumerate}
\item $(A+B)+C \to A+(B+C)$.
\item $A+B \to B+A$.
\end{enumerate}
If you like, check that these are isomorphisms.
\end{enumerate}
It can then be checked that this data obeys the axioms of a symmetric monoidal
category, but we'll end the exercise here.
}{
This exercise is about showing that coproducts and an initial object give a
symmetric monoidal category. Since all we have are coproducts and an initial
object, and since these are defined by their universal properties, the solution
is to use these universal properties over and over, to prove that all the data
of \cref{rdef.sym_mon_cat} can be constructed.
\begin{enumerate}
\item To define a functor $+\colon \cat{C} \times \cat{C} \to \cat{C}$ we must
define its action on objects and morphisms. In both cases, we just take the
coproduct. If $(A,B)$ is an object of $\cat{C} \times \cat{C}$, its image $A+B$
is, as usual, the coproduct of the two objects of $\cat{C}$. If $(f,g) \colon
(A,B)\to (C,D)$ is a morphism, then we can form a morphism $f+g = \copair{f \cp
\iota_C, g \cp \iota_D}\colon A+B \to C+D$, where $\iota_C \colon C \to C+D$ and
$\iota_D\colon D \to C+D$ are the canonical morphisms given by the definition of
the coproduct $A+B$.

Note that this construction sends identity morphisms to identity morphisms,
since by \cref{exc.coprod_properties} 4 we have 
\[
\id_A + \id_B = \copair{\id_A\cp \iota_{A}, \id_B \cp \iota_{B}} =\copair{\iota_{A}, \iota_{B}}=\id_{A+B}.
\]

To show that $+$ is a functor, we need to also show it preserves composition.
Suppose we also have a morphism $(h,k)\colon (C,D) \to (E,F)$ in $\cat{C} \times
\cat{C}$. We need to show that $(f+g) \cp (h+k) = (f \cp h) + (g \cp k)$. This
is a slightly more complicated version of the argument in
\cref{exc.coprod_properties} 3. It follows from the fact the diagram below
commutes:
\[
    \begin{tikzcd}[ampersand replacement=\&]
      A \ar[rr,"\iota_A"] \ar[dr,"f"'] \& \& A+B \ar[d, "f+g"] \&\& B
      \ar[ll,"\iota_B"'] \ar[dl, "g"] \\[5pt]
      \& C \ar[r, "\iota_C"] \ar[dr,"h\cp \iota_E"'] \& C+D \ar[d,"h+k" near
      start] \& D
      \ar[l,"\iota_D"'] \ar[dl, "k \cp \iota_F"] \\[5pt]
      \&\& E+F
    \end{tikzcd}
\]
Indeed, we again use the uniqueness of the copairing in
\eqref{eqn.universal_prop_coprod}, this time to show that $(f \cp h) + (g \cp
k)=\copair{f\cp h \cp \iota_E, g\cp k \cp \iota_F} = (f+g) \cp (h+k)$, as
required.

\item Recall the universal property of the initial object gives a unique map
$!_A\colon \varnothing \to A$. Then the copairing $\copair{\id_A,!_A}$ is a map
$A+\varnothing \to A$. Moreover, it is an isomorphism with inverse $\iota_A
\colon A \to A +\varnothing$. Indeed, using the properties in
\cref{exc.coprod_properties} and the universal property of the initial object,
we have $\iota_A \cp \copair{\id_A,!_A} = \id_A$, and 
\[
\copair{\id_A,!_A} \cp \iota_A = \copair{\id_A \cp \iota_A, !_A \cp \iota_A} =
\copair{\iota_A,!_{A+\varnothing}} = \copair{\iota_A,\iota_\varnothing} =
\id_{A+\varnothing}.
\]

An analogous argument shows $\copair{!_A,\id_A} \colon \varnothing +A \to A$ is
an isomorphism.

\item We'll just write down the maps and their inverses; we leave it to you, if
you like, to check that they indeed are inverses.
\begin{enumerate}
\item The map $\copair{\id_A +\iota_B,\iota_C} = \copair{\copair{\iota_A,\iota_B \cp
\iota_{B+C}},\iota_C\cp\iota_{B+C}} \colon (A+B)+C \to A+(B+C)$ is an isomorphism,
with inverse $\copair{\iota_A, \iota_B+\id_C} \colon A+(B+C) \to (A+B)+C$.
\item The map $\copair{\iota_A,\iota_B}\colon A+B \to B+A$ is an isomorphism.
Note our notation here is slightly confusing: there are two maps named
$\iota_A$, (i) $\iota_A \colon A \to A+B$, and (ii) $\iota_A\colon A \to B+A$, and
similarly for $\iota_B$. In the above we mean the map (ii). It has inverse
$\copair{\iota_A,\iota_B}\colon B+A \to A+B$, where in this case we mean the
map (i).
\end{enumerate}
\end{enumerate}
}

\sol{exc.disc_cats_have_pushouts}{
For any set $S$, we have the discrete category $\Cat{Disc}_S$, with $S$ as objects and only identity morphisms.
\begin{enumerate}
	\item Show that all pushouts exist in $\Cat{Disc}_S$, for any set $S$.
	\item For what sets $S$ does $\Cat{Disc}_S$ have an initial object?
\end{enumerate}
}
{
\begin{enumerate}
	\item Suppose given an arbitrary diagram of the form $B\from A\to C$ in $\Cat{Disc}_S$; we need to show that it has a pushout. The only morphisms in $\Cat{Disc}_S$ are identities, so in particular $A=B=C$, and the square consisting of all identities is its pushout.
	\item Suppose $\Cat{Disc}_S$ has an initial object $s$. Then $S$ cannot be empty! But it also cannot have more than one object, because if $s'$ is another object then there is a morphism $s\to s'$, but the only morphisms in $S$ are identities so $s=s'$. Hence the set $S$ must consist of exactly one element.
\end{enumerate}
}
\sol{exc.pushout}{
  What is the pushout of the functions $f\colon \ord{4} \to \ord{5}$ and $g\colon \ord{4} \to
  \ord{3}$ pictured below?
  \[ 
    \begin{tikzpicture}[x=.6cm,y=.4cm, baseline=(3a)]
      \node at (2,-1.5) {$f\colon \ord{4} \to \ord{5}$};
      \node[contact] at (0,3) (1) {};
      \node[contact] at (0,2) (2) {};
      \node[contact] at (0,1) (3) {};
      \node[contact] at (0,0) (4) {};
      \node[contact] at (4,3.5) (1a) {};
      \node[contact] at (4,2.5) (2a) {};
      \node[contact] at (4,1.5) (3a) {};
      \node[contact] at (4,0.5) (4a) {};
      \node[contact] at (4,-0.5) (5a) {};
      \node[draw, rounded corners, inner xsep=7pt, inner ysep=4pt, fit=(1) (4)] (B) {};
      \node[draw, rounded corners, inner xsep=7pt, inner ysep=4pt, fit=(1a) (5a)] (B) {};
      \begin{scope}[function]
      	\draw (1) to (1a);
      	\draw (2) to (1a);
      	\draw (3) to (3a);
      	\draw (4) to (5a);
      \end{scope}
    \end{tikzpicture}
    \hspace{.25\textwidth}
    \begin{tikzpicture}[x=.6cm,y=.4cm, baseline=(2a)]
      \node at (2,-1.35) {$g\colon \ord{4} \to \ord{3}$};
      \node[contact] at (0,3) (1) {};
      \node[contact] at (0,2) (2) {};
      \node[contact] at (0,1) (3) {};
      \node[contact] at (0,0) (4) {};
      \node[contact] at (4,2.5) (1a) {};
      \node[contact] at (4,1.5) (2a) {};
      \node[contact] at (4,0.5) (3a) {};
      \node[draw, rounded corners, inner xsep=7pt, inner ysep=4pt, fit=(1) (4)] (B) {};
      \node[draw, rounded corners, inner xsep=7pt, inner ysep=4pt, fit=(1a) (3a)] (B) {};
  %    
      \begin{scope}[function]
      	\draw (1) to (1a);
      	\draw (2) to (2a);
      	\draw (3) to (3a);
      	\draw (4) to (3a);
      \end{scope}
    \end{tikzpicture}
  \]
 Check your answer using the abstract description from \cref{ex.pushouts}.
}
{
The pushout is the set $\ord{4}$, as depicted in the top right in the diagram
below, equipped also with the depicted functions:
  \begin{equation}\label{eqn.pushout_4534}
    \begin{tikzpicture}[x=.5cm,y=.5cm, node distance=.5cm]
      \node[contact] at (0,0) (1) {};
      \node[contact, below left of=1] (2) {};
      \node[contact, below left of=2]  (3) {};
      \node[contact, below left of=3] (4) {};
      \node[draw, rotate fit=45, rounded corners, inner xsep=7pt, inner
      ysep=7pt, fit=(1) (4)] (I) {};
      \node at (2,-3.2) {$f$};
      %
      \node[contact,] at (4,0) (1a) {};
      \node[contact, below right of =1a] (2a) {};
      \node[contact, below right of =2a] (3a) {};
      \node[contact, below right of =3a] (4a) {};
      \node[contact, below right of =4a] (5a) {};
      \node[draw, rotate fit=45, rounded corners, inner xsep=7pt, inner
      ysep=7pt, fit=(1a) (5a)] (A) {};
      %
      \node at (-2.8,2) {$g$};
      \node[contact] at (0,4) (1b) {};
      \node[contact, above left of=1b] (2b) {};
      \node[contact, above left of=2b] (3b) {};
      \node[draw, rotate fit=45, rounded corners, inner xsep=7pt, inner
      ysep=7pt, fit=(1b) (3b)] (B) {};
      %
      \node[contact] at (4,4) (1p) {};
      \node[contact, above right of=1p] (2p) {};
      \node[contact, above right of=2p]  (3p) {};
      \node[contact, above right of=3p] (4p) {};
      \node[draw, rotate fit=45, rounded corners, inner xsep=7pt, inner
      ysep=7pt, fit=(1p) (4p)] (P) {};
      \node at (8,5) {\small pushout};
      %
      \begin{scope}[function]
      	\draw (1) to (1a);
      	\draw (2) to (1a);
      	\draw (3) to (3a);
      	\draw (4) to (5a);
      \end{scope}
      \begin{scope}[function]
      	\draw (1) to (1b);
      	\draw (2) to (2b);
      	\draw (3) to (3b);
      	\draw (4) to (3b);
      \end{scope}
      \begin{scope}[function]
      	\draw (1a) to (1p);
      	\draw (2a) to (2p);
      	\draw (3a) to (3p);
      	\draw (4a) to (4p);
      	\draw (5a) to (3p);
      \end{scope}
      \begin{scope}[function]
      	\draw (1b) to (1p);
      	\draw (2b) to (1p);
      	\draw (3b) to (3p);
      \end{scope}
    \end{tikzpicture}
    \end{equation}

We want to see that this checks out with the description from \cref{ex.pushouts}, i.e.\ that it is the set of equivalence classes in $\ord{5}\sqcup\ord{3}$ generated by the relation $\{f(a)\sim g(a)\mid a\in\ord{4}\}$. If we denote elements of $\ord{5}$ as $\{1,\ldots,5\}$ and those of $\ord{3}$ as $\{1',2',3'\}$, we can redraw the functions $f,g$:
\[
\begin{tikzcd}[ampersand replacement=\&, row sep=0]
	1\& \bullet\ar[l, -]\ar[r, -]\&1'\\
	2\& \bullet\ar[dl, -]\ar[ur, -]\&2'\\
	3\& \bullet\ar[ddl, -]\ar[ur,-]\&3'\\
	4\& \bullet\ar[dl, -]\ar[ur, -]\\
	5
\end{tikzcd}
\]
which says we take the equivalence relation on $\ord{5}\sqcup\ord{3}$ generated by: $1\sim 1'$, , $3\sim 1'$, $5\sim 2'$, and $5\sim 3'$. The equivalence classes are $\{1,1',3\}$, $\{2\}$, $\{4\}$, and $\{5,2',3'\}$. These four are exactly the four elements in the set labeled `pushout' in \cref{eqn.pushout_4534}.
}

\sol{exc.pushout_initial}{
In \cref{ex.pushout_initial} we asked ``why?'' three times.
\begin{enumerate}
	\item Give a justification for ``why?$^1$.''
	\item Give a justification for ``why?$^2$.''
	\item Give a justification for ``why?$^3$.''	
\end{enumerate}
}{
\begin{enumerate}
	\item The diagram to the left commutes because $\varnothing$ is initial,
	and so has a unique map $\varnothing \to X + Y$. This implies we
	must have $f\cp \iota_X = g \cp \iota_Y$.
	\item There is a unique map $X + Y \to T$ making the diagram in
	\eqref{eqn.univ_prop_pushout} commute simply by the universal property
	of the coproduct \eqref{eqn.universal_prop_coprod} applied to the maps
	$x\colon X \to T$ and $y\colon Y \to T$.
	\item Suppose $X+_\varnothing Y$ exists. By the universal property of
	$\varnothing$, given any pair of arrows $x\colon X \to T$ and $y\colon Y
	\to T$, the diagram	
	\[
	\begin{tikzcd}[ampersand replacement =\&]
	  \varnothing \ar[r,"f"] \ar[d,"g"']\&X\ar[d,"x"]\\
	  Y\ar[r,"y"'] \& T
	\end{tikzcd}
	\]
	commutes. This means, by the universal property of the pushout
	$X+_\varnothing Y$, there exists a unique map $t\colon X+_\varnothing Y \to T$
	such that $\iota_X \cp t = x$ and $\iota_Y \cp t =y$. Thus
	$X+_\varnothing Y$ is the coproduct $X+Y$.
\end{enumerate}
}

\sol{exc.W_colim}{
Check that the pushout of pushouts from \cref{ex.W_colim} satisfies the universal property of the colimit for the original diagram, \cref{eqn.W_colim}.
}
{
We have to check that the colimit of the diagram shown left really is given by taking three pushouts as shown right:
\[
\begin{tikzcd}[ampersand replacement=\&]
	\&
	B\ar[r]\ar[d]\&
	Z\\
	A\ar[r]\ar[d]\&
	C\\
	D
\end{tikzcd}
\hspace{1in}
\begin{tikzcd}[ampersand replacement=\&]
	\&
	B\ar[r]\ar[d]\&
	Z\ar[d]\\
	A\ar[r]\ar[d]\&
	Y\ar[r]\ar[d]\&
	R\ar[ul, phantom, very near start, "\ulcorner"]\ar[d]\\
	X\ar[r]\&
	Q\ar[ul, phantom, very near start, "\ulcorner"]\ar[r]\&
	S\ar[ul, phantom, very near start, "\ulcorner"]
\end{tikzcd}
\]
That is, we need to show that $S$, together with the maps from $A$, $B$, $X$, $Y$, and $Z$, has the required universal property. So suppose given an object $T$ with two commuting diagrams as shown:
\[
\begin{tikzcd}[ampersand replacement=\&]
	\&
	B\ar[r]\ar[d]\&
	Z\ar[dd]\\
	A\ar[r]\ar[d]\&
	Y\ar[rd]\&
	\\
	X\ar[rr]\&
	\&
	T
\end{tikzcd}
\]
We need to show there is a unique map $S\to T$ making everything commute. Since $Q$ is a pushout of $X\from A\to Y$, there is a unique map $Q\to T$ making a commutative triangle with $Y$, and since $R$ is the pushout of $Y\from B\to Z$, there is a unique map $R\to T$ making a commutative triangle with $Y$. This implies that there is a commuting $(Y,Q,R,T)$ square, and hence a unique map $S\to T$ from its pushout making everything commute. This is what we wanted to show.
}

\sol{exc.pushout_formula}{
Use the formula to show that pushouts---colimits on a diagram $X \xleftarrow{f}
N \xrightarrow{g} Y$---agree with the description we gave in \cref{ex.pushouts}.
}{
The formula in \cref{thm.colims_in_set} says that the pushout $X+_NY$ is given
by the set of equivalence classes of $X \sqcup N \sqcup Y$ under the equivalence
relation generated by $x \sim n$ if $x=f(n)$, and $y \sim n$ if $y = g(n)$,
where $x \in X$, $y\in Y$, $n \in N$.  Since for every $n \in N$ there exists an
$x \in X$ such that $x=f(n)$, this set is equal to the set of equivalence
classes of $X \sqcup Y$ under the equivalence relation generated by $x \sim y$
if there exists $n$ such that $x=f(n)$ and $y = g(n)$. This is exactly the
description of \cref{ex.pushouts}.
}

\sol{exc.cospan_tensor}{
In \cref{eq.cospan_comp} we showed morphisms $A\to B$ and $B\to C$ in
$\cospan{\finset}$. Draw their monoidal product as a morphism $A+B\to
B+C$ in $\cospan{\finset}$.
}{
The monoidal product is 
\[
      \begin{tikzpicture}[x=1.25cm]
	\begin{pgfonlayer}{nodelayer}
	%A
	  \node [contact, outer sep=5pt] (6) at (-2, 1) {};
	  \node [contact, outer sep=5pt] (7) at (-2, -0.5) {};
	  \node [contact, outer sep=5pt] (8) at (-2, 0.5) {};
	  \node [contact, outer sep=5pt] (9) at (-2, -0) {};
	  \node [contact, outer sep=5pt] (10) at (-2, -1) {};
	  %B2
	  \node [contact, outer sep=5pt] (-2a) at (-2, -1.5) {};
	  \node [contact, outer sep=5pt] (-1a) at (-2, -2) {};
	  \node [contact, outer sep=5pt] (0a) at (-2, -2.5) {};
	  \node [contact, outer sep=5pt] (1a) at (-2, -3) {};
	  \node [contact, outer sep=5pt] (2a) at (-2, -3.5) {};
	  \node [contact, outer sep=5pt] (3a) at (-2, -4) {};
	  %N
	  \node [contact, outer sep=5pt] (15) at (-0.5, 0.875) {};
	  \node [contact, outer sep=5pt] (28) at (-0.5, 0.25) {};
	  \node [contact, outer sep=5pt] (16) at (-0.5, -0.125) {};
	  \node [contact, outer sep=5pt] (29) at (-0.5, -0.5) {};
	  \node [contact, outer sep=5pt] (17) at (-0.5, -1) {};
	  %P
	  \node [contact, outer sep=5pt] (21) at (-0.5, -1.75) {};
	  \node [contact, outer sep=5pt] (23) at (-0.5, -2.25) {};
	  \node [contact, outer sep=5pt] (24) at (-0.5, -2.75) {};
	  \node [contact, outer sep=5pt] (22) at (-0.5, -3.25) {};
	  \node [contact, outer sep=5pt] (18) at (-0.5, -4) {};
	  %B1
	  \node [contact, outer sep=5pt] (-2) at (1, 1) {};
	  \node [contact, outer sep=5pt] (-1) at (1, 0.5) {};
	  \node [contact, outer sep=5pt] (0) at (1, 0) {};
	  \node [contact, outer sep=5pt] (1) at (1, -0.5) {};
	  \node [contact, outer sep=5pt] (2) at (1, -1) {};
	  \node [contact, outer sep=5pt] (3) at (1, -1.5) {};
	  %C
	  \node [contact, outer sep=5pt] (19) at (1, -2) {};
	  \node [contact, outer sep=5pt] (14) at (1, -2.5) {};
	  \node [contact, outer sep=5pt] (11) at (1, -3) {};
	  \node [contact, outer sep=5pt] (13) at (1, -3.5) {};
	  \node [contact, outer sep=5pt] (12) at (1, -4) {};
	  \node [style=none] (a) at (-2, -4.75) {$A+B$};
	  \node [style=none] (n) at (-.5, -4.75) {$N+P$};
	  \node [style=none] (b) at (1, -4.75) {$B+C$};
	\end{pgfonlayer}
	\begin{pgfonlayer}{edgelayer}
	  \begin{scope}[->,shorten <=10pt,shorten >=10pt]
	    \draw (6.center) to (15.center);
	    \draw (8.center) to (15.center);
	    \draw (-2.center) to (15.center);
	    \draw (-1.center) to (15.center);
	    \draw (9.center) to (16.center);
	    \draw (7.center) to (16.center);
	    \draw (10.center) to (17.center);
	    \draw (2.center) to (17.center);
	    \draw (3.center) to (17.center);
	    \draw (0.center) to (28.center);
	    \draw (1.center) to (29.center);
	    \draw (3a.center) to (18.center);
	    \draw (-2a.center) to (21.center);
	    \draw (-1a.center) to (21.center);
	    \draw (1a.center) to (22.center);
	    \draw (2a.center) to (22.center);
	    \draw (0a.center) to (24.center);
	    \draw (11.center) to (22.center);
	    \draw (12.center) to (18.center);
	    \draw (13.center) to (22.center);
	    \draw (14.center) to (23.center);
	    \draw (19.center) to (21.center);
	  \end{scope}
	\end{pgfonlayer}
      \end{tikzpicture}
\]
}

\sol{exc.cospan_comp}{
  Depicting the composite of cospans in \cref{eq.cospan_comp} with the wire
  notation gives
  \[
    \begin{aligned}
      \begin{tikzpicture}
	\begin{pgfonlayer}{nodelayer}
	  \node [contact, outer sep=5pt] (6) at (-2, 1) {};
	  \node [contact, outer sep=5pt] (7) at (-2, -0.5) {};
	  \node [contact, outer sep=5pt] (8) at (-2, 0.5) {};
	  \node [contact, outer sep=5pt] (9) at (-2, -0) {};
	  \node [contact, outer sep=5pt] (10) at (-2, -1) {};
	  \node [outer sep=5pt] (15) at (-0.5, 0.875) {};
	  \node [outer sep=5pt] (28) at (-0.5, 0.25) {};
	  \node [outer sep=5pt] (16) at (-0.5, -0.125) {};
	  \node [outer sep=5pt] (29) at (-0.5, -0.5) {};
	  \node [outer sep=5pt] (17) at (-0.5, -1) {};
	  \node [contact, outer sep=5pt] (-2) at (1, 1.25) {};
	  \node [contact, outer sep=5pt] (-1) at (1, 0.75) {};
	  \node [contact, outer sep=5pt] (0) at (1, 0.25) {};
	  \node [contact, outer sep=5pt] (1) at (1, -0.25) {};
	  \node [contact, outer sep=5pt] (2) at (1, -0.75) {};
	  \node [contact, outer sep=5pt] (3) at (1, -1.25) {};
	  \node [outer sep=5pt] (18) at (2.5, -1.125) {};
	  \node [outer sep=5pt] (21) at (2.5, 1) {};
	  \node [outer sep=5pt] (22) at (2.5, -0.375) {};
	  \node [outer sep=5pt] (23) at (2.5, 0.475) {};
	  \node [outer sep=5pt] (24) at (2.5, 0.25) {};
	  \node [contact, outer sep=5pt] (19) at (4, 1) {};
	  \node [contact, outer sep=5pt] (14) at (4, 0.5) {};
	  \node [contact, outer sep=5pt] (11) at (4, -0) {};
	  \node [contact, outer sep=5pt] (13) at (4, -0.5) {};
	  \node [contact, outer sep=5pt] (12) at (4, -1) {};
	\end{pgfonlayer}
	\begin{pgfonlayer}{edgelayer}
	  \begin{scope}[very thick]
	    \draw (6.center) to (15.center);
	    \draw (8.center) to (15.center);
	    \draw (-2.center) to (15.center);
	    \draw (-1.center) to (15.center);
	    \draw (9.center) to (16.center);
	    \draw (7.center) to (16.center);
	    \draw (10.center) to (17.center);
	    \draw (2.center) to (17.center);
	    \draw (3.center) to (17.center);
	    \draw (0.center) to (28.center);
	    \draw (1.center) to (29.center);
	    \draw (3.center) to (18.center);
	    \draw (-2.center) to (21.center);
	    \draw (-1.center) to (21.center);
	    \draw (1.center) to (22.center);
	    \draw (2.center) to (22.center);
	    \draw (0.center) to (24.center);
	    \draw (11.center) to (22.center);
	    \draw (12.center) to (18.center);
	    \draw (13.center) to (22.center);
	    \draw (14.center) to (23.center);
	    \draw (19.center) to (21.center);
	  \end{scope}
	\end{pgfonlayer}
      \end{tikzpicture}
    \end{aligned}
    \quad=\quad
    \begin{aligned}
      \begin{tikzpicture}	
	\begin{pgfonlayer}{nodelayer}
	  \node [contact, outer sep=5pt] (2) at (-2, 1) {};
	  \node [contact, outer sep=5pt] (3) at (-2, -0.5) {};
	  \node [contact, outer sep=5pt] (4) at (-2, 0.5) {};
	  \node [contact, outer sep=5pt] (5) at (-2, -0) {};
	  \node [contact, outer sep=5pt] (6) at (-2, -1) {};
	  \node [outer sep=5pt] (11) at (-0.5, 0.875) {};
	  \node [outer sep=5pt] (12) at (-0.5, 0.35) {};
	  \node [draw, fill=black, circle, inner sep=1.2pt] (extra) at (-0.5, 0.05) {};
	  \node [outer sep=5pt] (14) at (-0.5, -0.2) {};
	  \node [outer sep=5pt] (15) at (-0.5, -0.7) {};
	  \node [contact, outer sep=5pt] (7) at (1, -0) {};
	  \node [contact, outer sep=5pt] (8) at (1, -1) {};
	  \node [contact, outer sep=5pt] (9) at (1, -0.5) {};
	  \node [contact, outer sep=5pt] (10) at (1, 0.5) {};
	  \node [contact, outer sep=5pt] (13) at (1, 1) {};
	\end{pgfonlayer}
	\begin{pgfonlayer}{edgelayer}
	  \begin{scope}[very thick]
	    \draw (2.center) to (11.center);
	    \draw (4.center) to (11.center);
	    \draw (5.center) to (14.center);
	    \draw (3.center) to (14.center);
	    \draw (6.center) to (15.center);
	    \draw (7.center) to (15.center);
	    \draw (8.center) to (15.center);
	    \draw (9.center) to (15.center);
	    \draw (10.center) to (12.center);
	    \draw (13.center) to (11.center);
	  \end{scope}
	\end{pgfonlayer}
      \end{tikzpicture}
    \end{aligned}
  \]
  Comparing \cref{eq.cospan_comp} and \cref{eq.cospan_comp_wires}, describe the
  composition rule in $\cospan\finset$ in terms of wires and connected
  components.
}{
Let $x$ and $y$ be composable cospans in $\cospan\finset$. In terms of wires and
connected components, the composition rule in $\cospan\finset$ says that (i) the
composite cospan has a unique element in the apex for every connected component
of the concatenation of the wire diagrams $x$ and $y$, and (ii) in the wire
diagram for $x\cp y$, each element of the feet is connected by a wire to the
element representing the connected component to which it belongs.
}

\sol{exc.spider}{
  Which morphisms in the following list are equal?
  \begin{enumerate}
    \item 
      \[
\begin{tikzpicture}[spider diagram]
  \node[spider={2}{3}, fill=black] (a) {};
\end{tikzpicture}
\]
    \item 
      \[
\begin{tikzpicture}[spider diagram]
  \node[spider={2}{0}, fill=black] (a) {};
  \node[spider={0}{3}, fill=black, right=.5 of a] (b) {};
\end{tikzpicture}
\]
    \item
      \[
\begin{tikzpicture}[spider diagram]
  \node[spider={1}{2}, fill=black] (a) {};
  \node[spider={2}{2}, fill=black, right=1 of a] (b) {};
  \node[spider={1}{1}, fill=black, below=.5 of a] (c) {};
  \node[spider={1}{1}, fill=black, right=1 of c] (d) {};
  \begin{scope}
    \draw (a_out1) to (b_in1);
    \draw (a_out2) to (b_in2);
    \draw (c_out1) to (d_in1);
  \end{scope}
\end{tikzpicture}
\]
    \item
      \[
\begin{tikzpicture}[spider diagram]
  \node[spider={2}{2}, fill=black] (a) {};
  \node[spider={2}{3}, fill=black, right=1 of a] (b) {};
  \begin{scope}
    \draw (a_out1) to (b_in1);
    \draw (a_out2) to (b_in2);
  \end{scope}
\end{tikzpicture}
\]
    \item
      \[
\begin{tikzpicture}[spider diagram]
  \node[spider={1}{2}, fill=black] (a) {};
  \node[spider={2}{2}, fill=black, right=1 of a] (b) {};
  \node[spider={1}{2}, fill=black, below=.5 of a] (c) {};
  \node[spider={2}{1}, fill=black, right=1 of c] (d) {};
  \begin{scope}
    \draw (a_out1) to (b_in1);
    \draw (a_out2) to (b_in2);
    \draw (c_out1) to (d_in1);
    \draw (c_out2) to (d_in2);
  \end{scope}
\end{tikzpicture}
\]
    \item
      \[
\begin{tikzpicture}[spider diagram]
  \node[spider={1}{2}, fill=black] (a) {};
  \node[spider={1}{2}, fill=black, below=.8 of a] (b) {};
  \node[spider={2}{2}, fill=black, below right=.36 and .9 of a] (c) {};
  \node[spider={2}{1}, fill=black, right=1.9 of a] (d) {};
  \begin{scope}
    \draw (a_out1) to (d_in1);
    \draw (a_out2) to (c_in1);
    \draw (b_out1) to (c_in2);
  \draw (b_out2) to (d_out1|-b_out2);
    \draw (c_out1) to (d_in2);
    \draw (c_out2) to (d_out1|-c_out2);
  \end{scope}
\end{tikzpicture}
\]  
\end{enumerate}
}{
Morphisms 1, 4, and 6 are equal, and morphisms 3 and 5 are equal. Morphism 3 is
not equal to any other depicted morphism. This is an immediate consequence of
\cref{thm.spider}.
}

\sol{exc.suppressed_labels}{
\begin{enumerate}
	\item What label should be on the input to $h$?
	\item What label should be on the output of $g$?
	\item What label should be on the fourth output wire of the composite?
\end{enumerate}
}{
\begin{enumerate}
	\item The input to $h$ should be labelled $B$.
	\item The output of $g$ should be labelled $D$, since we know from the
	labels in the top right that $h$ is a morphism $B \to D \otimes D$.
	\item The fourth output wire of the composite should be labelled $D$
	too!
\end{enumerate}
}

\sol{exc.frob_cospan}{
  By \cref{ex.cospan_hypergraph}, the category $\cospan\finset$ is a hypergraph
  category. (In fact it is a hypergraph prop.) Draw the Frobenius maps for the
  object $\ord{1}$ in $\cospan\finset$ using both the function and wiring
  depictions as in \cref{ex.cospan_finset}.
}{
We draw the function depictions above, and the wiring depictions below. Note
that we depict the empty set with blank space.
  \[
    \begin{aligned}
      \begin{tikzpicture}
	\begin{pgfonlayer}{nodelayer}
	  \node [contact, outer sep=5pt] (i1) at (-1, 0.25) {};
	  \node [contact, outer sep=5pt] (i2) at (-1, -.25) {};
	  \node [contact, outer sep=5pt] (a) at (-0, 0) {};
	  \node [contact, outer sep=5pt] (o) at (1, 0) {};
	  %
	  \node [contact, outer sep=5pt] (i1w) at (-1, -0.75) {};
	  \node [contact, outer sep=5pt] (i2w) at (-1, -1.25) {};
	  \node [outer sep=5pt] (aw) at (-0, -1) {};
	  \node [contact, outer sep=5pt] (ow) at (1, -1) {};
	  %
	  \node [style=none] (4) at (-1, -1.75) {$1+1$};
	  \node [style=none] (20) at (0, -1.75) {$1$};
	  \node [style=none] (30) at (1, -1.75) {$1$};
	  %
	  \node [style=none] (30) at (0, -2.5) {multiplication $\mu_1$};
	\end{pgfonlayer}
	\begin{pgfonlayer}{edgelayer}
	  \begin{scope}[->,shorten <=10pt,shorten >=10pt]
	    \draw (i1.center) to (a.center);
	    \draw (i2.center) to (a.center);
	    \draw (o.center) to (a.center);
	  \end{scope}
	  \begin{scope}[very thick]
	    \draw (i1w.center) to (aw.center);
	    \draw (i2w.center) to (aw.center);
	    \draw (ow.center) to (aw.center);
	  \end{scope}
	\end{pgfonlayer}
      \end{tikzpicture}
    \end{aligned}
    \hspace{1cm}
    \begin{aligned}
      \begin{tikzpicture}
	\begin{pgfonlayer}{nodelayer}
	  \node (i1) at (-1, 0.25) {};
	  \node [contact, outer sep=5pt] (a) at (-0, 0) {};
	  \node [contact, outer sep=5pt] (o) at (1, 0) {};
	  %
	  \node [outer sep=5pt] (aw) at (-0, -1) {};
	  \node [contact, outer sep=5pt] (ow) at (1, -1) {};
	  %
	  \node [style=none] (4) at (-1, -1.75) {$\varnothing$};
	  \node [style=none] (20) at (0, -1.75) {$1$};
	  \node [style=none] (30) at (1, -1.75) {$1$};
	  %
	  \node [style=none] (30) at (0, -2.5) {unit $\eta_1$};
	\end{pgfonlayer}
	\begin{pgfonlayer}{edgelayer}
	  \begin{scope}[->,shorten <=10pt,shorten >=10pt]
	    \draw (o.center) to (a.center);
	  \end{scope}
	  \begin{scope}[very thick]
	    \draw (ow.center) to (aw.center);
	  \end{scope}
	\end{pgfonlayer}
      \end{tikzpicture}
    \end{aligned}
    \hspace{1cm}
    \begin{aligned}
      \begin{tikzpicture}
	\begin{pgfonlayer}{nodelayer}
	  \node [contact, outer sep=5pt] (i) at (-1, 0) {};
	  \node [contact, outer sep=5pt] (a) at (-0, 0) {};
	  \node [contact, outer sep=5pt] (o1) at (1, 0.25) {};
	  \node [contact, outer sep=5pt] (o2) at (1, -.25) {};
	  %
	  \node [contact, outer sep=5pt] (iw) at (-1, -1) {};
	  \node [outer sep=5pt] (aw) at (-0, -1) {};
	  \node [contact, outer sep=5pt] (o1w) at (1, -0.75) {};
	  \node [contact, outer sep=5pt] (o2w) at (1, -1.25) {};
	  %
	  \node [style=none] (4) at (-1, -1.75) {$1$};
	  \node [style=none] (20) at (0, -1.75) {$1$};
	  \node [style=none] (30) at (1, -1.75) {$1+1$};
	  %
	  \node [style=none] (30) at (0, -2.5) {comultiplication $\delta_1$};
	\end{pgfonlayer}
	\begin{pgfonlayer}{edgelayer}
	  \begin{scope}[->,shorten <=10pt,shorten >=10pt]
	    \draw (i.center) to (a.center);
	    \draw (o1.center) to (a.center);
	    \draw (o2.center) to (a.center);
	  \end{scope}
	  \begin{scope}[very thick]
	    \draw (iw.center) to (aw.center);
	    \draw (aw.center) to (o1w.center);
	    \draw (aw.center) to (o2w.center);
	  \end{scope}
	\end{pgfonlayer}
      \end{tikzpicture}
    \end{aligned}
    \hspace{1cm}
    \begin{aligned}
      \begin{tikzpicture}
	\begin{pgfonlayer}{nodelayer}
	  \node [contact, outer sep=5pt] (i) at (-1, 0) {};
	  \node [contact, outer sep=5pt] (a) at (-0, 0) {};
	  %
	  \node [contact, outer sep=5pt] (iw) at (-1, -1) {};
	  \node [outer sep=5pt] (aw) at (-0, -1) {};
	  %
	  \node [style=none] (4) at (-1, -1.75) {$1$};
	  \node [style=none] (20) at (0, -1.75) {$1$};
	  \node [style=none] (30) at (1, -1.75) {$\varnothing$};
	  %
	  \node [style=none] (30) at (0, -2.5) {counit $\epsilon_1$};
	\end{pgfonlayer}
	\begin{pgfonlayer}{edgelayer}
	  \begin{scope}[->,shorten <=10pt,shorten >=10pt]
	    \draw (i.center) to (a.center);
	  \end{scope}
	  \begin{scope}[very thick]
	    \draw (iw.center) to (aw.center);
	  \end{scope}
	\end{pgfonlayer}
      \end{tikzpicture}
    \end{aligned}
  \]
}

\sol{exc.frob_cospan2}{
  Using your knowledge of colimits, show that the maps defined in
  \cref{ex.cospan_hypergraph} do indeed obey the special law (see
  \cref{def.spec_comm_frob_mon}).
%  \begin{enumerate}
%  \item the unitality law (see \cref{eqn.ass_un_comm}), and
%  \item 
  %\item the Frobenius law.
%  \end{enumerate}
}{
%\begin{enumerate}
%  \item \cref{exc.coproducts_give_monoidal_structure}.
%
%  The unitality law says that the composite of cospans
%  \[
%  \unitl{.1\textwidth} = X \xrightarrow{\iota_1} X +X \xleftarrow{\id+\id} X+X
%  \xrightarrow{\copair{\id,\id}} X \xleftarrow{\id} X
%  \]
%  is the identity cospan. We first have to take the pushout in the middle. By
%  \cref{ex.pushout_along_identity}, the pushout of $\copair{\id,\id}$ along the
%  identity map $\id+\id$ is again $\copair{\id,\id}$, so the above expression is
%  equal to
%  \[
%  X \xrightarrow{\iota_1} X+X \xrightarrow{\copair{\id,\id}} X
%  \xleftarrow{\id} X \xleftarrow{\id} X.
%  \]
%  Next, \cref{exc.coprod_properties} show that $\iota_1 \cp\copair{\id,\id} =
%  \id$, so this composes to the identity cospan.

  The special law says that the composite of cospans
  \[
  \spec{.1\textwidth} = X \xrightarrow{\id} X \xleftarrow{\copair{\id,\id}} X+X
  \xrightarrow{\copair{\id,\id}} X \xleftarrow{\id} X
  \]
  is the identity. This comes down to checking that the square
  \begin{equation} \label{eqn.pushout_questionmark}
  \begin{tikzcd}[ampersand replacement=\&]
  X+X \ar[d,"\copair{\id,\id}"'] \ar[r,"\copair{\id,\id}"] \&
  X\ar[d,"\id"]  \\
  X \ar[r,"\id"'] \& X 
  \end{tikzcd}
  \end{equation}
  is a pushout square. It is trivial to see that the square commutes. Suppose
  now that we have maps $f\colon X \to Y$ and $g\colon X \to Y$ such that 
  \[
  \begin{tikzcd}[ampersand replacement=\&]
  X+X \ar[d,"\copair{\id,\id}"'] \ar[r,"\copair{\id,\id}"] \&
  X\ar[d,"f"]  \\
  X \ar[r,"g"'] \& T
  \end{tikzcd}
  \]
  Write $\iota_1\colon X \to X+X$ for the map into the first copy of $X$ in
  $X+X$, given by the definition of coproduct. Then, using the fact that
  $\iota_1 \cp \copair{\id,\id} = \id$ from \cref{exc.coprod_properties} 1, and
  the commutativity of the above square, we have $f=
  \iota_1\cp\copair{\id,\id}\cp f =\iota_1\cp\copair{\id,\id}\cp g = g$. This
  means that $f\colon X \to T$ is the unique map such that
  \[
    \begin{tikzcd}[ampersand replacement=\&]
      X \ar[r,"\copair{\id,\id}"] \ar[d,"\copair{\id,\id}"'] \& X \ar[d,"\id"]
      \ar[ddr,"f", bend left] \\
      X \ar[r,"\id"'] \ar[drr,"g=f"', bend right=20pt] \& X \ar[dr,dashed, "f"] \\[-7pt]
      \&\&[-15pt]T
    \end{tikzcd}
  \]
  commutes, and so \eqref{eqn.pushout_questionmark} is a pushout square.

%  \item 
%  The Frobenius law equates the composites of cospans
%  \[
%  \frobx{.1\textwidth} = X+X \xrightarrow{\copair{\id,\id}} X \xleftarrow{\id} X
%  \xrightarrow{\id} X \xleftarrow{\copair{\id,\id}} X+X
%  \]
%  and
%  \[
%  \frobs{.1\textwidth} = X+X \xrightarrow{\id+\id} X+X \xleftarrow{\copair{\id,\id}+\id}
%  X+X+X
%  \xrightarrow{\id+\copair{\id,\id}} X+X \xleftarrow{\id+\id} X+X
%  \]
%  This comes down to checking that the square
%  \[
%  \begin{tikzcd}[ampersand replacement=\&]
%  X+X \ar[r,"\copair{\id,\id}"] \& X \\
%  X+X+X \ar[u,"\copair{\id,\id}+\id"] \ar[r,"\id+\copair{\id,\id}"] \&
%  X+X\ar[u,"\copair{\id,\id}"'] 
%  \end{tikzcd}
%  \]
%  is a pushout square.
%\end{enumerate}
}

%\sol[print]{exc.frob_corel}{
%  Recall the monoidal category $(\Cat{Corel},\varnothing, \sqcup)$
%  (\cref{ex.corel}), whose objects are finite sets and whose morphisms are
%  corelations. Given a finite set $X$, define the corelation $\mu_X\colon
%  X\sqcup X \to X$ such that two elements of $X \sqcup X \sqcup X$ are
%  equivalent if and only if they come from the same underlying element of $X$.
%  Define $\delta_X\colon X \to X\sqcup X$ in the same way, and define
%  $\eta_X\colon \varnothing \to X$ and $\epsilon_X\colon X \to \varnothing$ such
%  that no two elements of $X = \varnothing \sqcup X = X \sqcup \varnothing$ are
%  equivalent.
%
%  Using the corelation diagrams of \cref{ex.corel}, convince yourself that these
%  maps define a special commutative Frobenius monoid
%  $(X,\mu_X,\eta_X,\delta_X,\epsilon_X)$.  Conclude that $\Cat{Corel}$ is a
%  hypergraph category.
%}{
%  
%}

\sol{exc.fill_in_diagram}{
Fill in the missing diagram in the proof of \cref{prop.hyp_cat_comp_closed}.
}{
The missing diagram is
\[
  \begin{aligned}
    \begin{tikzpicture}[yscale=.7]
      \begin{pgfonlayer}{nodelayer}
	\node [style=none] (left) at (-1.5, 1) {};
	\node [style=none] (helpl) at (-0.5, 1) {};
	\node [style=none] (unith) at (-0.5, 0) {};
	\node [style=bdot] (unit) at (-0.75, 0) {};
	\node [style=bdot] (mult) at (0, .5) {};
	\node [style=bdot] (comult) at (.5, .5) {};
	\node [style=none] (counith) at (1, 1) {};
	\node [style=bdot] (counit) at (1.25, 1) {};
	\node [style=none] (helpr) at (1, 0) {};
	\node [style=none] (right) at (2, 0) {};
      \end{pgfonlayer}
      \begin{pgfonlayer}{edgelayer}
	\draw (left.center) to (helpl.center);
	\draw (unit.center) to (unith.center);
	\draw [bend left, looseness=1.00] (helpl.center) to (mult.center);
	\draw [bend right, looseness=1.00] (unith.center) to (mult.center);
	\draw (mult) to (comult);
	\draw [bend left, looseness=1.00] (comult.center) to (counith.center);
	\draw [bend right, looseness=1.00] (comult.center) to (helpr.center);
	\draw (counith.center) to (counit.center);
	\draw (helpr.center) to (right.center);
      \end{pgfonlayer}
    \end{tikzpicture}
  \end{aligned}
\]
}

\sol{exc.powset_mon_coherence}{
  Check that the maps $\varphi_{S,T}$ defined in \cref{ex.powset} are natural in
  $S$ and $T$. In other words, given $f\colon S\to S'$ and $g\colon T\to T'$, show that the diagram below commutes:
  \[
  \begin{tikzcd}[column sep=large,ampersand replacement=\&]
  	\powset(S)\times\powset(T)\ar[r,"\varphi_{S,T}"] \ar[d,"\im_f\times \im_g"']\&
		\powset(S\times T)\ar[d, "\im_{f\times g}"]\\
		\powset(S')\times\powset(T')\ar[r, "\varphi_{S',T'}"']\&
		\powset(S'\times T')
  \end{tikzcd}  
  \]
}{
Let $A \subseteq S$ and $B \subseteq T$. Then 
\begin{align*}
\varphi_{S',T'}\left((\im_f\times \im_g)(A \times B)\right) 
&= \varphi_{S',T'}(\{f(a) \mid a \in A\} \times \{g(b) \mid b \in B\}) \\
&= \{(f(a),g(b)) \mid a \in A, \, b \in B\} \\
&= \im_{f\times g}(A \times B) \\
&= \im_{f\times g}(\varphi_{S,T}(A,B)).
\end{align*}
Thus the required square commutes.
}

\sol{exc.cospan_as_fcospan}{
	Suppose you're worried that the notation $\cospan{\cat{C}}$ looks like the notation $\cospan{F}$, even though they're very different. An expert tells you ``they're not so different; one is a special case of the other. Just use the constant functor $F(c)\coloneqq\{*\}$.'' What do they mean?
}{
They mean that every category $\cospan{\cat{C}}$ is equal to a category
$\cospan{F}$, for some well-chosen $F$. They also tell you how to choose this
$F$: take the functor $F\colon \cat{C} \to \smset$ that sends every object of
$\cat{C}$ to the set $\{\ast\}$, and every morphism of $\cat{C}$ to the identity
function on $\{\ast\}$. Of course, you will have to check this functor is a lax
symmetric monoidal functor, but in fact this is not hard to do.

To check that $\cospan{\cat{C}}$ is equal to $\cospan{F}$, first observe that
they have the same objects: the objects of $\cat{C}$. Next, observe that a
morphism in $\cospan{F}$ is a cospan $X \leftarrow N \rightarrow Y$ in $\cat{C}$
together with an element of $FN = \{\ast\}$. But $FN$ also has a unique element,
$\ast$! So there's no choice here, and we can consider morphisms of $\cospan{F}$
just to be cospans in $\cat{C}$. Moreover, composition of morphisms in
$\cospan{F}$ is simply the usual composition of cospans via pushout, so
$\cospan{F} =\cospan{\cat{C}}$.

(More technically, we might say that $\cospan{\cat{C}}$ and $\cospan{F}$ are
isomorphic, where the isomorphism is the identity-on-objects functor
$\cospan{\cat{C}} \to \cospan{F}$ that simply decorates each cospan with $\ast$,
and its inverse is the one that forgets this $\ast$. But this is close enough to
equal that many category theorists, us included, don't mind saying equal in this
case.)
}

\sol{exc.circuit_tuple}{
  Write a tuple $(V,A,s,t,\ell)$ that represents the circuit in
  \cref{eq.circuit}.
}{
We can represent the circuit in \cref{eq.circuit} by the tuple
$(V,A,s,t,\ell)$ where $V=\{\textrm{ul},\textrm{ur},\textrm{dl},\textrm{dr}\}$,
$A=\{\textrm{r1},\textrm{r2},\textrm{r3},\textrm{c1},\textrm{i1}\}$, and $s$, $t$, and $\ell$ are
defined by the table
\[
\begin{tabular}{c | c c c c c}
    & r1 & r2 & r3 & c1 & i1 \\ \hline
$s(-)$ & dl & ul & ur & ul & dl \\ 
$t(-)$ & ul & ur & dr & ur & dr \\
$\ell(-)$ & $1\Omega$ & $2\Omega$ & $1\Omega$ & $3F$ & $1H$ 
\end{tabular}
\]
}

\sol{exc.pushforwardcircuit}{
  To understand this functor better, let $c\in F(\ord{4})$ be the circuit
\[
  \begin{tikzpicture}[circuit ee IEC, set resistor graphic=var resistor IEC graphic]
    \node[contact]         (A) at (0,0) {};
    \node[contact]         (B) at (2,0) {};
    \node[contact]         (C) at (3,0) {};
    \node[contact]         (D) at (5,0) {};
    \path (A) edge  [capacitor] node[label={[label distance=1pt]90:{$3F$}}]
    {} (B);
    \path (C) edge  [resistor] node[label={[label distance=1pt]90:{$3\Omega$}}]
    {} (D);
    \node[below=0 of A, font=\scriptsize] {$1$};
    \node[below=0 of B, font=\scriptsize] {$2$};
    \node[below=0 of C, font=\scriptsize] {$3$};
    \node[below=0 of D, font=\scriptsize] {$4$};
  \end{tikzpicture}
\]
and let $f\colon \ord{4} \to \ord{3}$ be the function
\[
  \begin{tikzpicture}[y=.5cm]
    \node[contact]         (A) at (0,0) {};
    \node[contact]         (B) at (2,0) {};
    \node[contact]         (C) at (3,0) {};
    \node[contact]         (D) at (5,0) {};
    \node[contact]         (1) at (1,-2) {};
    \node[contact]         (2) at (2.5,-2) {};
    \node[contact]         (3) at (4,-2) {};
    \begin{scope}[font=\scriptsize]
      \node[above=0 of A] {$1$};
      \node[above=0 of B] {$2$};
      \node[above=0 of C] {$3$};
      \node[above=0 of D] {$4$};
      \node[below=0 of 1] {$1$};
      \node[below=0 of 2] {$2$};
      \node[below=0 of 3] {$3$};
		\end{scope}
    \begin{scope}[function]
      \draw (A) to (1);
      \draw (B) to (2);
      \draw (C) to (2);
      \draw (D) to (3);
    \end{scope}
  \end{tikzpicture}
\]
What is the circuit $\elec(f)(c)$?
}{
The circuit $\elec(f)(c)$ is
\[
  \begin{tikzpicture}[circuit ee IEC, set resistor graphic=var resistor IEC graphic]
    \node[contact]         (A) at (0,0) {};
    \node[contact]         (B) at (2,0) {};
    \node[contact]         (D) at (4,0) {};
    \path (A) edge  [bulb] (B);
    \path (B) edge  [resistor] node["$3\Omega$"]
    {} (D);
    \node[below=0 of A, font=\scriptsize] {$1$};
    \node[below=0 of B, font=\scriptsize] {$2\sim 3$};
    \node[below=0 of D, font=\scriptsize] {$4$};
  \end{tikzpicture}
\]
}

\sol{exc.parallelcirc}{
  Suppose we have circuits  
\[
\begin{aligned}
  \begin{tikzpicture}[circuit ee IEC, set make contact graphic=var make contact IEC graphic]
    \node at (-.5,0) {$b\coloneq$};
    \node[contact]         (A) at (0,0) {};
    \node[contact]         (B) at (2,0) {};
    \path (A) edge  [battery] (B);
  \end{tikzpicture}
\end{aligned}
  \qquad
  \mbox{and}
  \qquad
\begin{aligned}
  \begin{tikzpicture}[circuit ee IEC, set resistor graphic=var resistor IEC graphic]
    \node at (-.5,0) {$s\coloneq$};
    \node[contact]         (A) at (0,0) {};
    \node[contact]         (B) at (2,0) {};
    \path (A) edge  [make contact] (B);
  \end{tikzpicture}
\end{aligned}
\]
in $\elec(\ord{2})$. Use the definition of $\psi_{V,V'}$ from
\eqref{eqn.Flaxator} to figure out what $4$-vertex circuit $\psi_{\ord{2},\ord{2}}(b,s) \in
\elec(\ord{2} + \ord{2}) = \elec(\ord{4})$ should be.
}
{
The circuit $\psi_{\ord{2},\ord{2}}(b,s)$ is the disjoint union of the two
labelled graphs $b$ and $s$:
\[
\begin{aligned}
  \begin{tikzpicture}[circuit ee IEC, set make contact graphic=var make contact IEC graphic]
    \node[contact]         (A) at (0,0) {};
    \node[contact]         (B) at (2,0) {};
    \node[contact]         (C) at (4,0) {};
    \node[contact]         (D) at (6,0) {};
    \path (A) edge  [battery] (B);
    \path (C) edge  [make contact] (D);
  \end{tikzpicture}
\end{aligned}
\]
}

\sol{exc.namethedecoration}{
Morphisms of $\cospan\elec$ are $\elec$-decorated cospans, as defined in
\cref{def.decorated_cospan}. This means \eqref{eqn.decorated_cospan} depicts a
cospan together with a \emph{decoration}, which is some $C$-circuit
$(V,A,s,t,\ell) \in \elec(\ord{2})$. What is it?
}{
The cospan is the cospan $\ord{1} \To{f} \ord{2} \From{g} \ord{1}$, where
$f(1)=1$ and $g(1)=2$. The decoration is the $C$-circuit
$(\ord{2},\{a\},s,t,\ell)$, where $s(a)=1$, $t(a)=2$ and $\ell(a)=\battery$. 
}

\sol{exc.composefcospans}{
Refer back to the example at the beginning of \cref{sec.deccospans}. In
particular, consider the composition of circuits in
\cref{eq.circuitcomposition}. Express the two circuits in this diagram as
morphisms in $\cospan\elec$, and compute their composite. Does it match the
picture in \cref{eqn.circuitcomposed}?
}{
Recall the circuit $C\coloneqq (V,A,s,t,\ell)$ from the solution to
\cref{exc.circuit_tuple}. Then the first decorated cospan is given by the cospan
$\ord{1} \To{f} V \From{g} \ord{2}$, $f(1) =\textrm{ul}$, $g(1) =\textrm{ur}$,
and $g(2) =\textrm{ur}$, decorated by circuit $C$.
The second decorated cospan is given by the cospan $\ord{2} \To{f'} V' \From{g'}
\ord{2}$ and the circuit $C'\coloneqq(V',A',s',t',\ell')$, where $V'=\{l,r,d\}$,
$A'=\{\textrm{r}1',\textrm{r}2'\}$, and the functions are given by the tables
\[
\begin{tabular}{c | c c}
    & 1 & 2 \\ \hline
$f'(-)$ &  l & d\\ 
$g'(-)$ & r & r\\
\end{tabular}
\hspace{2cm}
\begin{tabular}{c | c c}
    & r1' & r2' \\ \hline
$s(-)$ &  l & r\\ 
$t(-)$ & r & d\\
$\ell(-)$ & $5\Omega$ & $8\Omega$
\end{tabular}
\]

To compose these, we first take the pushout of $V \From{g} \ord{2} \To{f'} V'$.
This gives the a new apex $V''
=\{\textrm{ul},\textrm{dl},\textrm{dr},\textrm{m},\textrm{r}\}$ with five
elements, and composite cospan $\ord{1} \To{h} V'' \From{k} \ord{2}$ given by
$h(1) = \textrm{ul}$, $k(1) = \textrm{r}$ and $k(2) =\textrm{m}$. The new
circuit is given by $(V,''A+A',s,''t,''\ell'')$ where the functions are given by
\[
\begin{tabular}{c | c c c c c c c}
    & r1 & r2 & r3 & c1 & i1 & r1' & r2' \\ \hline
$s''(-)$ & dl & ul & m & ul & dl & m & r \\ 
$t''(-)$ & ul & m & dr & m & dr & r & m \\
$\ell''(-)$ & $1\Omega$ & $2\Omega$ & $1\Omega$ & $3F$ & $1H$ & $5\Omega$ &
$8\Omega$ 
\end{tabular}
\]
This is exactly what is depicted in \cref{eqn.circuitcomposed}.
}

\sol{exc.buildcircuit}{
Write $x$ for the open circuit in \eqref{eqn.circuit}. Also define cospans $\eta\colon 0\to 2$ and $\eta\colon 2\to 0$ as follows: 
\[
\eta \coloneqq
\quad
\begin{aligned}
\begin{tikzpicture}[circuit ee IEC, set resistor graphic=var resistor IEC
graphic, set make contact graphic=var make contact IEC graphic]
  \node (i) at (-1,0) {$\varnothing$};
  \node [contact] (m) at (0,0) {};
  \node [draw, rounded corners, gray, inner ysep=10pt, inner xsep=10pt, fit=(m)] {};
  \node [draw, inner sep=1.5pt,circle] (a) at (1,.5) {};
  \node [draw, inner sep=1.5pt,circle] (b) at (1,-.5) {};
  \begin{scope}[mapsto]
    \draw (a) to (m);
    \draw (b) to (m);
  \end{scope}
\end{tikzpicture}
\end{aligned}
\hspace{1in}
\begin{aligned}
\begin{tikzpicture}[circuit ee IEC, set resistor graphic=var resistor IEC
graphic, set make contact graphic=var make contact IEC graphic]
  \node (i) at (1,0) {$\varnothing$};
  \node [contact] (m) at (0,0) {};
  \node [draw, rounded corners, gray, inner ysep=10pt, inner xsep=10pt, fit=(m)] {};
  \node [draw, inner sep=1.5pt,circle] (a) at (-1,.5) {};
  \node [draw, inner sep=1.5pt,circle] (b) at (-1,-.5) {};
  \begin{scope}[mapsto]
    \draw (a) to (m);
    \draw (b) to (m);
  \end{scope}
\end{tikzpicture}
\end{aligned}
\quad
=\colon\epsilon
\]
where each of these are decorated by the empty circuit
$(\ord{1},\varnothing,!,!,!) \in \elec(\ord{1})$.%
\tablefootnote{As usual $!$ denotes the unique function, in this case from the empty set to the relevant codomain.}

Compute the composite $\eta \cp x \cp \epsilon$ in $\cospan\elec$. This is a
morphism $\ord{0} \to \ord{0}$; we call such things \emph{closed circuits}.
}{
Composing $\eta$ and $x$ we have
\begin{align*}
\eta \cp x 
&=
\begin{aligned}
\begin{tikzpicture}[circuit ee IEC, set resistor graphic=var resistor IEC
graphic, set make contact graphic=var make contact IEC graphic]
  \node (i) at (-1,.5) {$\varnothing$};
  \node [contact] (n) at (0,.5) {};
  \node [draw, inner sep=1.5pt,circle] (a2) at (1,1) {};
  \node [draw, inner sep=1.5pt,circle] (a) at (1,0) {};
  \node [contact] (l2) at (2,1) {};
  \node [contact] (m2) at (4,1) {};
  \node [contact] (r2) at (6,1) {};
  \node [contact] (l) at (2,0) {};
  \node [contact] (m) at (4,0) {};
  \node [contact] (r) at (6,0) {};
  \node [draw, inner sep=1.5pt,circle] (b) at (7,0) {};
  \node [draw, inner sep=1.5pt,circle] (b2) at (7,1) {};
  \node [draw, rounded corners, gray, inner ysep=10pt, inner xsep=10pt, fit=(n)] {};
  \node [draw, rounded corners, gray, inner ysep=10pt, inner xsep=10pt, fit=(l2) (r)] {};
  \draw (l2) to [battery] (m2);
  \draw (m2) to [make contact] (r2);
  \draw (l) to [bulb] (m);
  \draw (m) to [resistor] (r);
  \begin{scope}[mapsto]
    \draw (a) to (n);
    \draw (a2) to (n);
    \draw (a) to (l);
    \draw (a2) to (l2);
    \draw (b) to (r);    
    \draw (b2) to (r2);
  \end{scope}
\end{tikzpicture}
\end{aligned}
\\
&=
\begin{aligned}
\begin{tikzpicture}[circuit ee IEC, set resistor graphic=var resistor IEC
graphic, set make contact graphic=var make contact IEC graphic]
  \node (i) at (1,.5) {$\varnothing$};
  \node [contact] (m2) at (4,1) {};
  \node [contact] (r2) at (6,1) {};
  \node [contact] (l) at (2,0.5) {};
  \node [contact] (m) at (4,0) {};
  \node [contact] (r) at (6,0) {};
  \node [draw, inner sep=1.5pt,circle] (b) at (7,0) {};
  \node [draw, inner sep=1.5pt,circle] (b2) at (7,1) {};
  \node [draw, rounded corners, gray, inner ysep=10pt, inner xsep=10pt, fit=(l)
  (r) (r2)] {};
  \draw (l) to [battery] (m2);
  \draw (m2) to [make contact] (r2);
  \draw (l) to [bulb] (m);
  \draw (m) to [resistor] (r);
  \begin{scope}[mapsto]
    \draw (b) to (r);    
    \draw (b2) to (r2);
  \end{scope}
\end{tikzpicture}
\end{aligned}
\end{align*}
and composing the result of $\epsilon$ gives
\begin{align*}
\eta \cp x \cp \epsilon
&=
\begin{aligned}
\begin{tikzpicture}[circuit ee IEC, set resistor graphic=var resistor IEC
graphic, set make contact graphic=var make contact IEC graphic]
  \node (i) at (1,.5) {$\varnothing$};
  \node [contact] (m2) at (4,1) {};
  \node [contact] (r2) at (6,1) {};
  \node [contact] (l) at (2,0.5) {};
  \node [contact] (m) at (4,0) {};
  \node [contact] (r) at (6,0) {};
  \node [draw, inner sep=1.5pt,circle] (b) at (7,0) {};
  \node [draw, inner sep=1.5pt,circle] (b2) at (7,1) {};
  \node [contact] (n) at (8,0.5) {};
  \node (i) at (9,0.5) {$\varnothing$};
  \node [draw, rounded corners, gray, inner ysep=10pt, inner xsep=10pt, fit=(l)
  (r) (r2)] {};
  \node [draw, rounded corners, gray, inner ysep=10pt, inner xsep=10pt, fit=(n)] {};
  \draw (l) to [battery] (m2);
  \draw (m2) to [make contact] (r2);
  \draw (l) to [bulb] (m);
  \draw (m) to [resistor] (r);
  \begin{scope}[mapsto]
    \draw (b) to (r);    
    \draw (b2) to (r2);
    \draw (b) to (n);
    \draw (b2) to (n);  
  \end{scope}
\end{tikzpicture}
\end{aligned}
\\
&=
\begin{aligned}
\begin{tikzpicture}[circuit ee IEC, set resistor graphic=var resistor IEC
graphic, set make contact graphic=var make contact IEC graphic]
  \node (i) at (1,.5) {$\varnothing$};
  \node [contact] (m2) at (4,1) {};
  \node [contact] (l) at (2,0.5) {};
  \node [contact] (m) at (4,0) {};
  \node [contact] (r) at (6,0.5) {};
  \node [draw, rounded corners, gray, inner ysep=10pt, inner xsep=10pt, fit=(l)
  (r) (m) (m2)] {};
  \draw (l) to [battery] (m2);
  \draw (m2) to [make contact] (r);
  \draw (l) to [bulb] (m);
  \draw (m) to [resistor] (r);
  \node (i) at (7,.5) {$\varnothing$};
\end{tikzpicture}
\end{aligned}
\end{align*}
}

\sol{exc.wd_drawing_practice}{
\begin{enumerate}
	\item Consider the following cospan $f\in\oprdcospan(2,2; 2)$:
	\[
	\begin{tikzpicture}[x=.25cm, y=.75cm]
		\begin{scope}[every node/.style={draw, inner sep=1.5pt, fill, circle}]
  		\node (a1) {};
  		\node[right=1 of a1] (a2) {};
  		\node[right=2 of a2] (a3) {};
  		\node[right=1 of a3] (a4) {};
		%
  		\node at ($(a2)!.5!(a3)+(0,-1)$) (b2) {};
  		\node[left=1 of b2] (b1) {};
  		\node[right=1 of b2] (b3) {};
		%
			\node at ($(b1)!.5!(b2)+(0,-1)$) (c1) {};
			\node[right=1 of c1] (c2) {};
		\end{scope}
		\begin{scope}[every node/.style={draw, rounded corners, gray, inner ysep=5pt, inner xsep=5pt}]
  	  \node [fit=(a1) (a2)] {};
  	  \node [fit=(a3) (a4)] {};
  	  \node [fit=(b1) (b3)] {};
  	  \node [fit=(b1) (b3)] {};
  	  \node [fit=(c1) (c2)] {};
		\end{scope}
		\begin{scope}[mapsto]
			\draw (a1) -- (b1);
			\draw (a2) -- (b2);
			\draw (a3) -- (b2);
			\draw (a4) -- (b3);
			\draw (c1) -- (b1);
			\draw (c2) -- (b3);
		\end{scope}
	\end{tikzpicture}
	\]
	Draw it as a wiring diagram with two inner circles, each with two ports, and one outer circle with two ports.
	\item Draw the wiring diagram corresponding to the following cospan $g\in\oprdcospan(2,2,2;0)$:
	\[
	\begin{tikzpicture}[x=.25cm, y=.75cm]
		\begin{scope}[every node/.style={draw, inner sep=1.5pt, fill, circle}]
			\node (a11) {};
			\node[right=1 of a11] (a12) {};
			\node[right=2 of a12] (a21) {};
			\node[right=1 of a21] (a22) {};
			\node[right=2 of a22] (a31) {};
			\node[right=1 of a31] (a32) {};
		%
			\node at ($(a21)!.5!(a22)+(0,-2)$) (b2) {};
			\node[left=1 of b2] (b1) {};
			\node[right=1 of b2] (b3) {};
		%
		\end{scope}
		\node at ($(b2)+(0,-1)$) {$\varnothing$};
		\begin{scope}[every node/.style={draw, rounded corners, gray, inner ysep=7pt, inner xsep=5pt}]
  	  \node [fit=(a11) (a12)] {};
  	  \node [fit=(a21) (a22)] {};
  	  \node [fit=(a31) (a32)] {};
			\node [fit=(b1) (b3)] {};
		\end{scope}
		\begin{scope}[mapsto]
			\draw (a11) -- (b2);
			\draw (a12) -- (b1);
			\draw (a21) -- (b1);
			\draw (a22) -- (b3);
			\draw (a31) -- (b3);
			\draw (a32) -- (b2);
		\end{scope}
	\end{tikzpicture}
	\]	
	\item Compute the cospan $g\circ_1 f$. What is its arity?
	\item Draw the cospan $g\circ_1 f$. Do you see it as substitution?
\end{enumerate}
}
{
\begin{enumerate}
	\item The cospan shown left corresponds to the wiring diagram shown right:
	\[
	\begin{tikzpicture}[x=.25cm, y=.75cm]
		\begin{scope}[every node/.style={draw, inner sep=1.5pt, fill, circle}]
  		\node (a1) {};
  		\node[right=1 of a1] (a2) {};
  		\node[right=2 of a2] (a3) {};
  		\node[right=1 of a3] (a4) {};
		%
  		\node at ($(a2)!.5!(a3)+(0,-1)$) (b2) {};
  		\node[left=1 of b2] (b1) {};
  		\node[right=1 of b2] (b3) {};
		%
			\node at ($(b1)!.5!(b2)+(0,-1)$) (c1) {};
			\node[right=1 of c1] (c2) {};
		\end{scope}
		\begin{scope}[every node/.style={draw, rounded corners, gray, inner ysep=5pt, inner xsep=5pt}]
  	  \node [fit=(a1) (a2)] (inner1) {};
  	  \node [fit=(a3) (a4)] (inner2) {};
  	  \node [fit=(b1) (b3)] (links) {};
  	  \node [fit=(c1) (c2)] (outer){};
		\end{scope}
		\node[left=1 of inner1, gray] (lab) {inner circles' ports};
		\node[gray, left] at (lab.east|-links) {links};
		\node[gray, left] at (lab.east|-outer) {outer circle's ports};
		\begin{scope}[mapsto]
			\draw (a1) -- (b1);
			\draw (a2) -- (b2);
			\draw (a3) -- (b2);
			\draw (a4) -- (b3);
			\draw (c1) -- (b1);
			\draw (c2) -- (b3);
		\end{scope}
	\end{tikzpicture}
	\hspace{.6in}
	\begin{tikzpicture}[unoriented WD, pack size=25pt, pack inside color=white, pack
outside color=black, link size=2pt, font=\footnotesize, spacing=30pt]
  	\node[pack] at (0,0) (f) {};
  	\node[pack, right=.6 of f] (g) {};
  	\node[outer pack, inner xsep=3pt, inner ysep=3pt, fit=(f) (g) ] (fg) {};
  	\node[link,  left=.2 of f.west] (link1) {};
  	\node[link] at ($(f)!.5!(g)$) (link2) {};
  	\node[link,  right=.2 of g.east] (link3) {};
  	\draw (fg.west) -- (link1);
  	\draw (f.west) -- (link1);
  	\draw (f.east) -- (link2);
  	\draw (g.west) -- (link2);
  	\draw (g.east) -- (link3);
  	\draw (fg.east) -- (link3);
\end{tikzpicture}
	\]
  It has two inner circles, each with two ports. One port of the first is wired to a port of the second. One port of the first is wired to the outside circle, and one port of the second is wired to the outside circle. This is exactly what the cospan says to do.
	\item The cospan shown left corresponds to the wiring diagram shown right:
		\[
	\begin{tikzpicture}[x=.25cm, y=.75cm]
		\begin{scope}[every node/.style={draw, inner sep=1.5pt, fill, circle}]
			\node (a11) {};
			\node[right=1 of a11] (a12) {};
			\node[right=2 of a12] (a21) {};
			\node[right=1 of a21] (a22) {};
			\node[right=2 of a22] (a31) {};
			\node[right=1 of a31] (a32) {};
		%
			\node at ($(a21)!.5!(a22)+(0,-1)$) (b2) {};
			\node[left=1 of b2] (b1) {};
			\node[right=1 of b2] (b3) {};
		%
		\end{scope}
		\coordinate (helper) at ($(b2)+(0,-1)$) {};
		\begin{scope}[every node/.style={draw, rounded corners, gray, inner ysep=5pt, inner xsep=5pt}]
  	  \node [fit=(a11) (a12)] (inner1) {};
  	  \node [fit=(a21) (a22)] (inner2) {};
  	  \node [fit=(a31) (a32)] (inner3) {};
			\node [fit=(b1) (b3)] (links) {};
  	  \node [fit=(helper)] (outer){};
		\end{scope}
		\node[left=1 of inner1, gray] (lab) {inner circles' ports};
		\node[gray, left] at (lab.east|-links) {links};
		\node[gray, left] at (lab.east|-outer) {outer circle's ports};
		\begin{scope}[mapsto]
			\draw (a11) -- (b2);
			\draw (a12) -- (b1);
			\draw (a21) -- (b1);
			\draw (a22) -- (b3);
			\draw (a31) -- (b3);
			\draw (a32) -- (b2);
		\end{scope}
	\end{tikzpicture}
	\hspace{.6in}
	\begin{tikzpicture}[unoriented WD, pack size=15pt, pack inside color=white, pack
outside color=black, link size=2pt, font=\footnotesize, spacing=20pt]
  	\node[pack] (f) {};
  	\node[pack, right=1 of f] (g) {};
		\node[pack] at ($(f)!.5!(g)+(0,-1.5)$) (h) {};
  	\node[outer pack, inner xsep=3pt, inner ysep=0pt, fit=(f) (g) (h)] (fgh) {};
  	\node[link] at ($(f)!.5!(g)$) (linkfg) {};
  	\node[link] at ($(f)!.5!(h)$) (linkfh) {};
  	\node[link] at ($(g)!.5!(h)$) (linkgh) {};
  	\draw (f) -- (linkfg);
  	\draw (g) -- (linkfg);
  	\draw (f) -- (linkfh);
  	\draw (h) -- (linkfh);
  	\draw (g) -- (linkgh);
  	\draw (h) -- (linkgh);
\end{tikzpicture}
	\]	
	\item The composite $g\circ_1 f$ has arity $(2,2,2,2;0)$; there is a depiction on the left:
			\[
	\begin{tikzpicture}[x=.25cm, y=.75cm]
		\begin{scope}[every node/.style={draw, inner sep=1.5pt, fill, circle}]
			\node (a01) {};
			\node[right=1 of a01] (a02) {};
			\node[right=2 of a02] (a11) {};
			\node[right=1 of a11] (a12) {};
			\node[right=2 of a12] (a21) {};
			\node[right=1 of a21] (a22) {};
			\node[right=2 of a22] (a31) {};
			\node[right=1 of a31] (a32) {};
		%
			\node at ($(a12)!.5!(a21)+(0,-1)$) (b2) {};
			\node[left=1 of b2] (b1) {};
			\node[right=1 of b2] (b3) {};
			\node[below=.3 of b2] (b0) {};
		%
		\end{scope}
		\coordinate (helper) at ($(b0)+(0,-1)$);
		\begin{scope}[every node/.style={draw, rounded corners, gray, inner ysep=5pt, inner xsep=5pt}]
  	  \node [fit=(a01) (a02)] (inner 1) {};
  	  \node [fit=(a11) (a12)] {};
  	  \node [fit=(a21) (a22)] {};
  	  \node [fit=(a31) (a32)] {};
			\node [fit=(b0) (b1) (b3)] (links) {};
			\node [fit=(helper)] (outer) {};
		\end{scope}
		\node[left=1 of inner1, gray] (lab) {inner circles' ports};
		\node[gray, left] at (lab.east|-links) {links};
		\node[gray, left] at (lab.east|-outer) {outer circle's ports};
		\begin{scope}[mapsto]
			\draw (a01) to[bend right] (b0);
			\draw (a32) to[bend left] (b0);
			\draw (a02) -- (b1);
			\draw (a11) -- (b1);
			\draw (a12) -- (b2);
			\draw (a21) -- (b2);
			\draw (a22) -- (b3);
			\draw (a31) -- (b3);
		\end{scope}
	\end{tikzpicture}
	\hspace{.6in}
		\begin{tikzpicture}[unoriented WD, pack size=15pt, pack inside color=white, pack
outside color=black, link size=2pt, font=\footnotesize, spacing=20pt]
  	\node[pack] (f) {};
  	\node[pack, right=1 of f] (g) {};
		\node[pack, below=1 of f] (h) {};
		\node[pack, below=1 of g] (i) {};
  	\node[outer pack, inner xsep=3pt, inner ysep=0pt, fit=(f) (g) (h) (i)] {};
  	\node[link] at ($(f)!.5!(g)$) (linkfg) {};
  	\node[link] at ($(f)!.5!(h)$) (linkfh) {};
  	\node[link] at ($(g)!.5!(i)$) (linkgi) {};
  	\node[link] at ($(h)!.5!(i)$) (linkhi) {};
  	\draw (f) -- (linkfg);
  	\draw (g) -- (linkfg);
  	\draw (f) -- (linkfh);
  	\draw (h) -- (linkfh);
  	\draw (g) -- (linkgi);
  	\draw (i) -- (linkgi);
  	\draw (h) -- (linkhi);
  	\draw (i) -- (linkhi);
\end{tikzpicture}
	\]
	\item The associated wiring diagram is shown on the right above. One can see that one diagram has been substituted in to a circle of the other.
\end{enumerate}
}

\finishSolutionChapter
%======== Section =========%
\section[Solutions for Chapter 7]{Solutions for \cref{chap.temporal_topos}.}

\sol{exc.pullback_pasting}{
Prove \cref{prop.pullback_pasting} using the definition of limit from \cref{subsec.adjoints_lims_colims}.
}{
In the commutative diagram below, suppose the $(B,C,B',C')$ square is a pullback:
\[
\begin{tikzcd}[ampersand replacement=\&]
	A\ar[r, "f"]\ar[d, "h_1"]\&B\ar[r, "g"]\ar[d, "h_2"]\&C\ar[d, "h_3"]\\
	A'\ar[r, "f'"']\&B'\ar[r, "g'"']\&C'\ar[ul, phantom, very near end, "\lrcorner"]
\end{tikzcd}
\]
We need to show that the $(A,B,A',B')$ square is a pullback iff the $(A,C,A',C')$ rectangle is a pullback.

Suppose first that $(A,B,A',B')$ is a pullback, and take any $(X,p,q)$ as in the following diagram:
\[
\begin{tikzcd}[ampersand replacement=\&]
	X\ar[rrrd, bend left=15pt, "p"]\ar[rdd, bend right=15pt, "q"']\&[-15pt]\\[-5pt]
	\&A\ar[r, "f"]\ar[d, "h_1"]\&B\ar[r, pos=.4, "g"]\ar[d, "h_2"']\&C\ar[d, "h_3"']\\
	\&A'\ar[r, "f'"']\&B'\ar[r, "g'"']\&C'\ar[ul, phantom, very near end, "\lrcorner"]
\end{tikzcd}
\]
where $q\cp f'\cp g'=p\cp h_3$. Then by the universal property of the $(B,C,B',C')$ pullback, we get a unique dotted arrow $r$ making the left-hand diagram below commute:
\[
\begin{tikzcd}[ampersand replacement=\&]
	X\ar[rrrd, bend left=15pt, "p"]\ar[rrdd, bend right=15pt, "q\cp f'"']\ar[rrd, dashed, "r"']\&[-15pt]\\[-5pt]
	\&\&B\ar[r, pos=.4, "g"]\ar[d, "h_2"']\&C\ar[d, "h_3"']\\
	\&\&B'\ar[r, "g'"']\&C'\ar[ul, phantom, very near end, "\lrcorner"]
\end{tikzcd}
\hspace{1in}
\begin{tikzcd}[ampersand replacement=\&]
	X\ar[rrd, bend left=15pt, "r"]\ar[rdd, bend right=15pt, "q"']\ar[rd, dashed, "r'"]\&[-15pt]\\[-5pt]
	\&A\ar[r, "f"]\ar[d, "h_1"]\&B\ar[r, pos=.4, "g"]\ar[d, "h_2"']\&C\ar[d, "h_3"']\\
	\&A'\ar[r, "f'"']\&B'\ar[r, "g'"']\&C'\ar[ul, phantom, very near end, "\lrcorner"]
\end{tikzcd}
\]
In other words $r\cp h_2=g\cp f'$ and $r\cp g=p$. Then by the universal property of the $(A,B,A',B')$ pullback, we get a unique dotted arrow $r'\colon X\to A$ making the right-hand diagram commute, i.e.\ $r'\cp f=r$ and $r'\cp h_1=q$. This gives the existence of an $r$ with the required property, $r'\cp f=r$ and $r'\cp f\cp g=r\cp g=p$. To see uniqueness, suppose given another morphisms $r_0$ such that $r_0\cp f\cp g=p$ and $r_0\cp h_1=q$:
\[
\begin{tikzcd}[ampersand replacement=\&]
	X\ar[dr, "r_0"]\ar[rrrd, bend left=15pt, "p"]\ar[rdd, bend right=20pt, "q"']\&[-15pt]\\[-5pt]
	\&A\ar[r, "f"]\ar[d, "h_1"]\&B\ar[r, pos=.4, "g"]\ar[d, "h_2"']\&C\ar[d, "h_3"']\\
	\&A'\ar[r, "f'"']\&B'\ar[r, "g'"']\&C'\ar[ul, phantom, very near end, "\lrcorner"]
\end{tikzcd}
\]
Then by the uniqueness of $r$, we must have $r_0\cp f=r$, and then by the uniqueness of $r'$, we must have $r_0=r'$. This proves the first result.\\

The second is similar. Suppose that $(A,C,A',C')$ and $(B,C,B',C')$ are pullbacks and suppose given a commutative diagram of the following form:
\[
\begin{tikzcd}[ampersand replacement=\&]
	X\ar[rrd, bend left=15pt, "r"]\ar[rdd, bend right=15pt, "q"']\&[-15pt]\\[-5pt]
	\&A\ar[r, "f"]\ar[d, "h_1"]\&B\ar[r, pos=.4, "g"]\ar[d, "h_2"']\&C\ar[d, "h_3"']\\
	\&A'\ar[r, "f'"']\&B'\ar[r, "g'"']\&C'\ar[ul, phantom, very near end, "\lrcorner"]
\end{tikzcd}
\]
i.e.\ where $r\cp h_2=q\cp f'$. Then letting $p\coloneqq r\cp g$, we have
\[p\cp h_3=r\cp g\cp h_3=r\cp h_2\cp g'=q\cp f'\cp g'\]
so by the universal property of the $(A,C,A',C')$ pullback, there is a unique morphism $r'\colon X\to A$ such that $r'\cp f\cp g=p$ and $r_0\cp h_1=q$, as shown:
\[
\begin{tikzcd}[ampersand replacement=\&]
	X\ar[dr, "r'"]\ar[rrd, gray, bend left=15pt, "r" description]\ar[rrrd, bend left=15pt, "p"]\ar[rdd, bend right=20pt, "q"']\&[-15pt]\\[-5pt]
	\&A\ar[r, "f"]\ar[d, "h_1"]\&B\ar[r, pos=.4, "g"]\ar[d, "h_2"']\&C\ar[d, "h_3"']\\
	\&A'\ar[r, "f'"']\&B'\ar[r, "g'"']\&C'\ar[ul, phantom, very near end, "\lrcorner"]
\end{tikzcd}
\]
But now let $r_0\coloneqq r'\cp f$. It satisfies $r_0\cp g=p$ and $r_0\cp h_2=q\cp f'$, and $r$ satisfies the same equations: $r\cp g=p$ and $r\cp h_2=q\cp f'$. Hence by the universal property of the $(B,C,B',C')$ pullback $r_0=r'$. It follows that $r'$ is a pullback of the $(A,B,A',B')$ square, as desired.
}

\sol{exc.mono_inj}{
Show that in $\smset$, monomorphisms are just injections:
\begin{enumerate}
	\item Show that if $f$ is a monomorphism then it is injective.
	\item Show that if $f\colon A\to B$ is injective then it is a monomorphism.
\end{enumerate}
}{
A function $f\colon A\to B$ is injective iff for all $a_1,a_2\in A$, if $f(a_1)=f(a_2)$ then $a_1=a_2$. It is a monomorphism iff for all sets $X$ and functions $g_1,g_2\colon X\to A$, if $g_1\cp f=g_2\cp f$ then $g_1=g_2$. Indeed, this comes directly from the universal property of the pullback from \cref{def.mono_epi},
\[
\begin{tikzcd}[ampersand replacement=\&]
	X\ar[ddr, bend right=20pt, "g_1"']\ar[drr, bend left=20pt, "g_2"]\ar[dr, dashed]
	\&[-10pt]\\[-10pt]
	\&A\ar[r, "\id_A"]\ar[d, "\id_A"']\&A\ar[d, "f"]\\
	\&A\ar[r, "f"']\&B\ar[ul, phantom, very near end, "\lrcorner"]
\end{tikzcd}
\]
because the dashed arrow is forced to equal both $g_1$ and $g_2$, thus forcing $g_1=g_2$.
\begin{enumerate}
	\item Suppose $f$ is a monomorphism, let $a_1,a_2\in A$ be elements, and suppose $f(a_1)=f(a_2)$. Let $X=\{*\}$ be a one element set, and let $g_1,g_2\colon X\to A$ be given by $g_1(*)\coloneqq a_1$ and $g_2(*)\coloneqq a_2$. Then $g_1\cp f=g_2\cp f$, so $g_1=g_2$, so $a_1=a_2$.
	\item Suppose that $f$ is an injection, let $X$ be any set, and let $g_1,g_2\colon X\to A$ be such that $g_1\cp f=g_2\cp f$. We will have $g_1=g_2$ if we can show that $g_1(x)=g_2(x)$ for every $x\in X$. So take any $x\in X$; since $f(g_1(x))=f(g_2(x))$ and $f$ is injective, we have $g_1(x)=g_2(x)$ as desired.
\end{enumerate}
}

\sol{exc.pullback_iso_iso}{
\begin{enumerate}
	\item	Show that the pullback of an isomorphism along any morphism is an isomorphism. That is, suppose that $i\colon B'\to B$ is an isomorphism and $f\colon A\to B$ is any morphism. Show that $i'$ is an isomorphism, in the following diagram:
	\[
	\begin{tikzcd}[ampersand replacement=\&]
		A'\ar[r, "f'"]\ar[d, pos=.6, "i'"', "\cong"]\&B'\ar[d, pos=.6, "i", "\cong"']\\
		A\ar[r, "f"']\&B\ar[ul, phantom, very near end, "\lrcorner"]
	\end{tikzcd}
	\]	
	\item Show that for any map $f\colon A\to B$, the square shown is a pullback:
	\[
	\begin{tikzcd}
		A\ar[r, "f"]\ar[d, equal]&
		A\ar[d, equal]\\
		B\ar[r, "f"']&
		B\ar[ul, phantom, very near end, "\lrcorner"]
	\end{tikzcd}
	\]
\end{enumerate}
}{
\begin{enumerate}
	\item Suppose we have a pullback as shown, where $i$ is an isomorphism:
  	\[
  	\begin{tikzcd}[ampersand replacement=\&]
  		A'\ar[r, "f'"]\ar[d, "i'"']\&B'\ar[d, "i", "\cong"']\\
  		A\ar[r, "f"']\&B\ar[ul, phantom, very near end, "\lrcorner"]
  	\end{tikzcd}
  	\]
  	Let $j\coloneqq i\inv$ be the inverse of $i$, and consider $g\coloneqq (f\cp j)\colon A\to B'$. Then $g\cp i=f$, so by the existence part of the universal property, there is a map $j'\colon A\to A'$ such that $j'\cp i'=\id_A$ and $j'\cp f'=f\cp j$. We will be done if we can show $i'\cp j'=\id_{A'}$. One checks that $(i'\cp j')\cp i'=i'$ and that $(i'\cp j')\cp f'=i'\cp f\cp j=f'\cp i\cp j=f'$. But $\id_{A'}$ also satisfies those properties: $\id_{A'}\cp i'=i'$ and $\id_{A'}\cp f'=f'$, so by the uniqueness part of the universal property, $(i'\cp j')=\id_{A'}$.
	\item We need to show that the following diagram is a pullback:
		\[
	\begin{tikzcd}[ampersand replacement=\&]
		A\ar[r, "f"]\ar[d, equal]\&
		B\ar[d, equal]\\
		A\ar[r, "f"']\&
		B\ar[ul, phantom, very near end, "\lrcorner"]
	\end{tikzcd}
	\]
	So take any object $X$ and morphisms $g\colon X\to A$ and $h\colon X\to B$ such that $g\cp f=h\cp\id_B$. We need to show there is a unique morphism $r\colon X\to A$ such that $r\cp\id_A=g$ and $r\cp f=h$. That's easy: the first requirement forces $r=g$ and the second requirement is then fulfilled.
\end{enumerate}
}

\sol{exc.monos_pb_pasting}{
Let $\cat{C}$ be a category and suppose the following diagram is a pullback in $\cat{C}$:
\[
\begin{tikzcd}[ampersand replacement=\&]
	A'\ar[r]\ar[d, "f'"']\&A\ar[d, tail, "f"]\\
	B'\ar[r]\&B\ar[ul, phantom, very near end, "\lrcorner"]
\end{tikzcd}
\]
Use \cref{prop.pullback_pasting,exc.pullback_iso_iso} to show that if $f$ is a monomorphism, then so is $f'$.
}{
Consider the diagram shown left, in which all three squares are pullbacks:
\[
\begin{tikzcd}[ampersand replacement=\&, sep=small]
	\&\&
	A\ar[rd, equal]\ar[dd, equal]\\
	\&A'\ar[rr, crossing over, pos=.25, "g"]\&\&
	A\ar[dd, pos=.25, "f"]\\
	A'\ar[rr, pos=.75, "g"']\ar[dr, "f'"']\&\&
	A\ar[rd, "f"]\\
	\&B'\ar[from=uu, crossing over, pos=.25, "f'"']\ar[rr, "h"']\&\&
	B
\end{tikzcd}
\hspace{1in}
\begin{tikzcd}[ampersand replacement=\&, sep=small]
	A'\ar[rr, "g"]\ar[dd, equal]\ar[rd, equal]\ar[drrr, gray, dotted]\&\&
	A\ar[rd, equal]\ar[dd, equal]\\
	\&A'\ar[rr, crossing over, pos=.25, "g"]\&\&
	A\ar[dd, pos=.25, "f"]\\
	A'\ar[rr, pos=.75, "g"']\ar[dr, "f'"']\ar[drrr, gray, dotted]\&\&
	A\ar[rd, "f"]\\
	\&B'\ar[from=uu, crossing over, pos=.25, "f'"']\ar[rr, "h"']\&\&
	B
\end{tikzcd}
\]
The front and bottom squares are the same---the assumed pullback---and the right-hand square is a pullback because $f$ is assumed monic. We can complete it to the commutative diagram shown right, where the back square and top square are pullbacks by \cref{exc.pullback_iso_iso}. Our goal is to show that the left-hand square is a pullback.

To do this, we use two applications of the pasting lemma, \cref{exc.pullback_pasting}. 
Since the right-hand face is a pullback and the back face is a pullback, the diagonal rectangle (lightly drawn) is also a pullback. Since the front face is a pullback, the left-hand face is also a pullback.
}

\sol{exc.epi_mono_practice}{
Factor the following function $f\colon \ord{3}\to \ord{3}$ as an epimorphism followed by a monomorphism.
\[
  \begin{tikzpicture}
		\foreach \x in {0,...,2} 
			{\draw (0,.4-.4*\x) node (X0\x) {$\bullet$};}
		\node[draw, ellipse, inner sep=0pt, fit=(X00) (X02)] (X0) {};
		\foreach \x in {0,...,2} 
			{\draw (2,.4-.4*\x) node (Y0\x) {$\bullet$};}
		\node[draw, ellipse, inner sep=0pt, fit=(Y00) (Y02)] (Y0) {};
		\draw[mapsto] (X00.center) -- (Y01.center);
		\draw[mapsto] (X01.center) -- (Y01.center);
		\draw[mapsto] (X02.center) -- (Y02.center);
  \end{tikzpicture}
\]
}
{
The following is an epi-mono factorization of $f$:
\[
  \begin{tikzpicture}
		\foreach \x in {0,...,2} 
			{\draw (0,.4-.4*\x) node (X0\x) {$\bullet$};}
		\node[draw, ellipse, inner sep=0pt, fit=(X00) (X02)] (X0) {};
		\foreach \x in {1,...,2} 
			{\draw (2,.4-.4*\x) node (Y0\x) {$\bullet$};}
		\node[draw, ellipse, inner sep=0pt, fit=(Y01) (Y02)] (Y0) {};
		\foreach \x in {0,...,2} 
			{\draw (4,.4-.4*\x) node (Z0\x) {$\bullet$};}
		\node[draw, ellipse, inner sep=0pt, fit=(Z00) (Z02)] (Y0) {};
		\begin{scope}[short=-2pt, mapsto]
  		\draw (X00) -- (Y01);
  		\draw (X01) -- (Y01);
  		\draw (X02) -- (Y02);
			\draw (Y01) -- (Z01);
			\draw (Y02) -- (Z02);
		\end{scope}
  \end{tikzpicture}
\]
}

\sol{ex.ccposet_quantale}{
Let $\cat{V}=(V,\leq,I,\otimes,\multimap)$ be a (unital, commutative) quantale---see
\cref{def.quantale}---and suppose it satisfies the following for all $v,w,x\in V$:
\begin{itemize}
	\item $v\leq I$,
	\item $v\otimes w\leq v$ and $v\otimes w\leq w$
	\item $if $x\leq v$ and $x\leq w$ then $x\leq v\otimes w$.
\end{enumerate}
\begin{enumerate}
	\item Show that $\cat{V}$ is a cartesian closed category, in fact a cartesian closed preorder.
	\item Can every cartesian closed preorder be obtained in this way?
\qedhere
\end{enumerate}
}{
\begin{enumerate}
  \item If $\cat{V}$ is a quantale with the stated properties, then
  \begin{itemize}
  	\item $I$ serves as a top element: $v\leq I$ for all $v\in V$. 
		\item $v\otimes w$ serves as a meet operation, i.e.\ it satisfies the same universal property as $\wedge$, namely $v\otimes w$ is a greatest lower bound for $v$ and $w$.
	\end{itemize}
	Now the $\multimap$ operation satisfies the same universal property as exponentiation (hom-object) does, namely $v\leq (w\multimap x)$ iff $v\otimes W\leq x$. So $\cat{V}$ is a cartesian closed category, and of course it is a preorder.
	\item Not every cartesian closed preorder comes from a quantale with the stated properties, because quantales have all joins and cartesian closed preorders need not. Finding a counterexample---a cartesian closed preorder that is missing some joins---takes some ingenuity, but it can be done. Here's one we came up with:
	\[
	\begin{tikzcd}[ampersand replacement=\&, sep=small, font=\small]
		(0,0)\ar[from=r]\ar[from=d]\&
		(0,1)\ar[from=r]\ar[from=d]\&
		(0,2)\ar[from=r]\ar[from=d]\&
		(0,3)\ar[from=r, -, dotted]\ar[from=d, -, dotted]
		\&{}\\
		(1,0)\ar[from=r]\ar[from=d]\&
		(1,1)\ar[from=r]\ar[from=d]\&
		(1,2)\ar[from=r, -, dotted]\ar[from=d, -, dotted]
		\&{}\\
		(2,0)\ar[from=r]\ar[from=d]\&
		(1,1)\ar[from=r, -, dotted]\ar[from=d, -, dotted]
		\&{}\\
		(3,0)\ar[from=r, -, dotted]\ar[from=d, -, dotted]
		\&{}\\
		{}
	\end{tikzcd}
	\]
	This is the product preorder $\nn\op\times\nn\op$: its objects are pairs $(a,b)\in\nn\times\nn$ with $(a,b)\leq(a',b')$ iff, in the usual ordering on $\nn$ we have $a'\leq a$ and $b'\leq b$. But you can just look at the diagram.\\
	
	It has a top element, $(0,0)$, and it has binary meets, $(a,b)\wedge(a',b')=(\max(a,a'),\max(b,b'))$. But it has no bottom element, so it has no empty join. Thus we will be done if we can show that for each $x,y$, the hom-object $x\multimap y$ exists. The formula for it is $x\multimap y=\bigvee\{w\mid w\wedge x\leq y\}$, i.e.\ we need these particular joins to exist. Since $y\wedge x\leq y$, we have $y\leq x\multimap y$. So we can replace the formula with $x\multimap y=\bigvee\{w\mid y\leq w\text{ and }w\wedge x\leq y\}$. But the set of elements in $\nn\op\times\nn\op$ that are bigger than $y$ is finite and nonempty.%
	\footnote{If $y=(a,b)$ then there are exactly $(a+1)*(b+1)$ elements $y'$ for which $y\leq y'$.}
	So this is a finite nonempty join, and $\nn\op\times\nn\op$ has all finite nonempty joins: they are given by $\inf$.
\end{enumerate}
}

\sol{exc.characteristic_practice}{
  Let $X=\NN=\{0,1,2,\ldots\}$ and $Y=\ZZ=\{\ldots,-1,0,1,2,\ldots\}$; we have
  $X\ss Y$, so consider it as a monomorphism $m\colon X\to Y$. It has a
  characteristic function $\corners{m}\colon Y\to\BB$, as in
  \cref{def.subobject_classifier}.
  \begin{enumerate}
    \item What is $\corners{m}(-5)\in\BB$?
    \item What is $\corners{m}(0)\in\BB$?  
	\end{enumerate}
}{
Let $m\colon \ZZ\to\BB$ be the characteristic function of the inclusion $\nn\ss\zz$.

\begin{enumerate*}[itemjoin=\hspace{1in}]
	\item $\corners{m}(-5)=\false$.
	\item $\corners{m}(0)=\true.$
\end{enumerate*}
}

\sol{exc.simple_char_funs}{
  \begin{enumerate}
    \item Consider the identity function $\id_\NN\colon \NN\to \NN$. It is an
      injection, so it has a characteristic function $\corners{\id_\NN}\colon
      \NN\to\BB$.  Give a concrete description of $\corners{\id_\NN}$, i.e.\ its
      exact value for each natural number $n\in\NN$.
    \item Consider the unique function $!_\NN\colon\varnothing\to\NN$ from the
      empty set. Give a concrete description of $\corners{!_\NN}$.  
\end{enumerate}
}{
\begin{enumerate}
	\item The characteristic function $\corners{\id_\NN}\colon\NN\to\BB$ sends each $n\in\nn$ to $\true$.
	\item Let $!_\NN\colon\varnothing\to\nn$ be the inclusion of the empty set. The characteristic function $\corners{!_\NN}\colon\NN\to\BB$ sends each $n\in\nn$ to $\false$.
\end{enumerate}
}

\sol{exc.neg_char}{
Every boolean has a negation, $\neg\false=\true$ and $\neg\true=\false$. The function $\neg\colon\BB\to\BB$ is the characteristic function of some thing, (*?*).
\begin{enumerate}
	\item What sort of thing should (*?*) be? For example, should $\neg$ be the characteristic function of an object? A topos? A morphism? A subobject? A pullback diagram?
	\item Now that you know the sort of thing (*?*) is, which thing of that sort is it?
\end{enumerate}
}{
\begin{enumerate}
	\item The sort of thing (*?*) we're looking for is a subobject of $\BB$, say $A\ss\BB$. This would have a characteristic function, and we're trying to find the $A$ for which the characteristic function is $\neg\colon\BB\to\BB$.
	\item The question now asks ``what is $A$?'' The answer is $\{\false\}\ss\BB$.
\end{enumerate}
}

\sol{exc.implies_char}{
Given two booleans $P,Q$, define $P\imp Q$ to mean $P=(P\wedge Q)$.
\begin{enumerate}
	\item Write down the truth table for the statement $P=(P\wedge Q)$:
	\[
	\begin{array}{cc||c|c}
		P&Q&P\wedge Q&P=(P\wedge Q)\\
		\true&\true&\?&\?\\
		\true&\false&\?&\?\\
		\false&\true&\?&\?\\
		\false&\false&\?&\?\\
	\end{array}
	\]
	\item If you already have an idea what $P\imp Q$ should mean, does it agree with the last column of table above?
	\item What is the characteristic function $m\colon \BB\times\BB\to\BB$ for $P\imp Q$?
	\item What subobject does $m$ classify?
\end{enumerate}
}{
\begin{enumerate}
	\item Here is the truth table for $P=(P\wedge Q)$:
	\begin{equation}\label{eqn.truth_table_imp}
	\begin{array}{cc||c|c}
		P&Q&P\wedge Q&P=(P\wedge Q)\\
		\true&\true&\true&\true\\
		\true&\false&\false&\false\\
		\false&\true&\false&\true\\
		\false&\false&\false&\true\\
	\end{array}
	\end{equation}
	\item Yes!
	\item The characteristic function for $P\imp Q$ is the function $\corners{\imp}\colon\BB\times\BB\to\BB$ given by the first, second, and fourth column of \cref{eqn.truth_table_imp}.
	\item It classifies the subset $\{(\true,\true),(\false,\true),(\false,\false)\ss\BB\times\BB$.
\end{enumerate}
}

\sol{exc.even_prime_10}{
Consider the sets $E\coloneqq\{n\in\NN\mid n\text{ is even}\}$, $P\coloneqq\{n\in\NN\mid n\text{ is prime}\}$, and $T\coloneqq\{n\in\NN\mid n\geq 10\}$. Each is a subset of $\NN$, so defines a function $\NN\to\BB$.
\begin{enumerate}
	\item What is $\corners{E}(17)$?
	\item What is $\corners{P}(17)$?
	\item What is $\corners{T}(17)$?
	\item Name the smallest three elements in the set classified by $(\corners{E}\wedge\corners{P})\vee\corners{T}$.
\end{enumerate}
}{
Say that $\corners{E},\corners{P},\corners{T}\colon\nn\to\bb$ classify respectively the subsets $E\coloneqq\{n\in\NN\mid n\text{ is even}\}$, $P\coloneqq\{n\in\NN\mid n\text{ is prime}\}$, and $T\coloneqq\{n\in\NN\mid n\geq 10\}$ of $\nn$.
\begin{enumerate}
	\item $\corners{E}(17)=\false$ because $17$ is not even.
	\item $\corners{P}(17)=\true$ because $17$ is prime.
	\item $\corners{T}(17)=\true$ because $17\geq 10$.
	\item The set classified by $(\corners{E}\wedge\corners{P})\vee\corners{T}$ is that of all natural numbers that are either above 10 or an even prime. The smallest three elements of this set are $2, 10, 11$.
\end{enumerate}
}

\sol{ex.usual_top_R}{
\begin{enumerate}
	\item What is the 1-dimensional analogue of $\epsilon$-balls as found in \cref{ex.usual_R}? That is, for each $x\in\RR$, define $B(x,\epsilon)$.
	\item When is an arbitrary subset $U\ss\RR$ called open, in analogy with \cref{ex.usual_R}?
	\item Find three open sets $U_1$, $U_2$, and $U$ in $\RR$, such that $(U_i)_{i\in\{1,2\}}$ covers $U$.
	\item Find an open set $U$ and a collection $(U_i)_{i\in I}$ of opens sets where $I$ is infinite, such that $(U_i)_{i\in I}$ covers $U$.
\end{enumerate}
}{
\begin{enumerate}
	\item The 1-dimensional analogue of an $\epsilon$-ball around a point $x\in\RR$ is $B(x,\epsilon)\coloneqq\{x'\in\RR\mid |x-x'|<\epsilon\}$, i.e.\ the set of all points within $\epsilon$ of $x$.
	\item A subset $U\ss\RR$ is open if, for every $x\in U$ there is some $\epsilon>0$ such that $B(x,\epsilon)\ss U$.
	\item Let $U_1\coloneqq\{x\in\RR\mid 0<x<2\}$ and $U_2\coloneqq\{x\in\RR\mid 1<x<3\}$. Then $U\coloneqq U_1\cup U_2=\{x\in\RR\mid 0<x<3\}$.
	\item Let $I=\{1,2,3,4,\ldots\}$ and for each $i\in I$ let $U_i\coloneqq\{x\in\RR\mid \frac{1}{i}<x<1\}$, so we have $U_1\ss U_2\ss U_3\ss\cdots$. Their union is $U\coloneqq\bigcup_{i\in I}U_i=\{x\in\RR\mid 0<x<1\}$.
\end{enumerate}
}

\sol{exc.course_fine}{
\begin{enumerate}
	\item Verify that for any set $X$, what we called $\Op_{\mathrm{crse}}$ in \cref{ex.coarse_fine} really is a topology, i.e.\ satisfies the conditions of \cref{def.topological_space}.
	\item Verify also that $\Op_{\mathrm{fine}}$ really is a topology.
	\item Show that if $(X,\powset(X))$ is discrete and $(Y,\Op_Y)$ is any topological space, then every function $X\to Y$ is continuous.
\end{enumerate}
}{
\begin{enumerate}
	\item The coarse topology on $X$ is the one whose only open sets are $X\ss X$ and $\varnothing\ss X$. This is a topology because it contains the top and bottom subsets, it is closed under finite intersection (the intersection $A\cap B$ is $\varnothing$ iff one or the other is $\varnothing$), and it is closed under arbitrary union (the union $\bigcup_{i\in I}A_i$ is $X$ iff $A_i=X$ for some $i\in I$).
	\item The fine topology on $X$ is the one where every subset $A\ss X$ is considered open. All the conditions on a topology say ``if such-and-such then such-and-such is open,'' but these are all satisfied because everything is open!
	\item If $(X,\powset(X))$ is discrete, $(Y,\Op_Y)$ is any topological space, and $f\colon X\to Y$ is any function then it is continuous. Indeed, this just means that for any open set $U\ss Y$ the preimage $f\inv(U)\ss X$ is open, and everything in $X$ is open.
\end{enumerate}
}

\sol{exc.opens_Sierp}{
Recall the Sierpinski space, say $(X,\Op_1)$ from \cref{ex.Sierpinski}.
\begin{enumerate}
	\item Write down the Hasse diagram for its preorder of opens.
	\item Write down all the coverings.
\end{enumerate}
}{
\begin{enumerate}
	\item The Hasse diagram for the Sierpinsky topology is \fbox{$\varnothing\to\{1\}\to\{1,2\}$}.
	\item A set $(U_i)_{i\in I}$ covers $U$ iff either
	\begin{itemize}
		\item $I=\varnothing$ and $U=\varnothing$; or
		\item $U_i=U$ for some $i\in I$.
	\end{itemize}
	In other words, the only way that some collection of these sets could cover another set $U$ is if that collection contains $U$ or if $U$ is empty and the collection is also empty.
\end{enumerate}
}

\sol{exc.subspace_topology}{
  Given any topological space $(X,\Op)$, any subset $Y\subseteq X$ can be given the
  \emph{subspace topology}, call it $\Op_{?\cap Y}$. This topology defines any $A \subseteq Y$ to be open, $A\in\Op_{?\cap Y}$,
if there is an open set $B\in\Op$ such that $A = B \cap Y$.
\begin{enumerate}
	\item Find a $B\in\Op$ that shows that the whole set $Y$ is open, i.e.\ $Y\in\Op_{?\cap Y}$.
	\item Show that $\Op_{?\cap Y}$ is a topology in the sense of \cref{def.topological_space}.%
	\footnote{Hint 1: for any set $I$, collection of sets $(U_i)_{i\in I}$ with $U_i\ss X$, and set $V\ss X$, one has $\left(\bigcup_{i\in I}U_i\right)\cap V=\bigcup_{i\in I}(U_i\cap V)$. Hint 2: for any $U, V, W\ss X$, one has $(U\cap W)\cap (V\cap W)=(U\cap V)\cap W$.}
	\item Show that the inclusion function $Y \hookrightarrow X$ is a
	  continuous function.
\end{enumerate}
}{
Let $(X,\Op)$ be a topological space, suppose that $Y\ss X$ is a subset, and consider the subspace topology $\Op_{?\cap Y}$.
\begin{enumerate}
	\item We want to show that $Y\in\Op_{?\cap Y}$. We need to find $B\in\Op$ such that $Y=B\cap Y$; this is easy, you could take $B=Y$ or $B=X$, or anything in between.
	\item We still need to show that $\Op_{?\cap Y}$ contains $\varnothing$ and is closed under finite intersection and arbitrary union. $\varnothing=\varnothing\cap Y$, so according to the formula, $\varnothing\in\Op_{?\cap Y}$. Suppose that $A_1,A_2\in\Op_{?\cap Y}$. Then there exist $B_1,B_2\in\Op$ with $A_1=B_1\cap Y$ and $A_2=B_2\cap Y$. But then $A_1\cap A_2=(B_1\cap Y)\cap (B_2\cap Y)=(B_1\cap B_2)\cap Y$, so it is in $\Op_{?\cap Y}$ since $B_1\cap B_2\in\Op$. The same idea works for arbitrary unions: given a set $I$ and $A_i$ for each $i\in I$, we have $A_i=B_i\cap Y$ for some $B_i\in\Op$, and
	\[\bigcup_{i\in I}A_i=\bigcup_{i\in I}(B_i\cap Y)=\left(\bigcup_{i\in i}B_i\right)\cap Y\in\Op_{?\cap Y}.\]
\end{enumerate}
}

\sol{exc.top_sp_quantale}{
In \cref{subsec.Lawv_metric_spaces,subsec.preorders_Bool_enriched} we discussed how $\Bool$-categories are preorders and $\Cost$-categories are Lawvere metric spaces, and in \cref{subsec.variations_quantale} we imagined interpretations of $\cat{V}$-categories for other quantales $\cat{V}$.

If $(X,\Op)$ is a topological space and $\cat{V}$ the corresponding quantale as in \cref{rem.top_sp_quantale}, how might we imagine a $\cat{V}$-category? 
}{
Let's imagine a $\cat{V}$-category $\cat{C}$, where $\cat{V}$ is the quantale corresponding to the open sets of a topological space $(X,\Op)$. Its Hasse diagram would be a set of dots and some arrows between them, each labeled by an open set $U\ss\Op$. It might look something like this:
\[
\begin{tikzpicture}[font=\scriptsize, x=1cm]
	\node (a) {$\LMO{A}$};
	\node[right=1 of a] (b) {$\LMO{B}$};
	\node[below=1 of a] (c) {$\LMO[under]{C}$};
	\node[right=1 of c] (d) {$\LMO[under]{D}$};
	\draw[->] (c) to node[above left=-1pt and -1pt] {$U_5$} (b);
	\draw[bend right,->] (a) to node[left] {$U_1$} (c);
	\draw[bend left,->] (d) to node[below] {$U_2$} (c);
	\draw[bend right,->] (b) to node[above] {$U_3$} (a);
	\draw[bend left,->] (b) to node[right] {$U_4$} (d);
	\node[draw, inner sep=20pt, fit=(a) (b) (c) (d)] (X) {};
	\node[left=0 of X, font=\normalsize] {$\cat{C}\coloneqq$};
\end{tikzpicture}
\] 
Recall from \cref{sec.enrichment} that the `distance' between two points is computed by taking the join, over all paths between them, of the monoidal product of distances along that path. For example, $\cat{C}(B,C)=(U_3\wedge U_1)\vee(U_4\wedge U_2)$, because $\wedge$ is the monoidal product in $\cat{V}$.\\

In general, we can thus imagine the open set $\cat{C}(a,b)$ as a kind of `size restriction' for getting from $a$ to $b$, like bridges that your truck needs to pass under. The size restriction for getting from $a$ to itself is $X$: no restriction. In general, to go on any given route (path) from $a$ to $b$, you have to fit under every bridge in the path, so we take their meet. But we can go along any path, so we take the join over all paths.
}

\sol{exc.fiber_practice}{
Consider the function $f\colon X\to Y$ shown in \cref{eqn.sections_of_function}.
\begin{enumerate}
	\item What is the fiber of $f$ over $a$?
	\item What is the fiber of $f$ over $c$?
	\item What is the fiber of $f$ over $d$?
	\item Gave an example of a function $f'\colon X\to Y$ for which every fiber has either one or two elements.
\end{enumerate}
}{
\begin{equation}\label{eqn.sections_of_function_redraw}
\begin{tikzpicture}[y=.35cm, baseline=(f), 
	every label quotes/.style={font=\scriptsize, above, label distance=-5pt}]
	\node["$a$"] (Ya) {$\bullet$};
	\node[right=1 of Ya,  "$b$"] (Yb) {$\bullet$};
	\node[right=1 of Yb,  "$c$"] (Yc) {$\bullet$};
	\node[right=1 of Yc,  "$d$"] (Yd) {$\bullet$};
	\node[right=1 of Yd,  "$e$"] (Ye) {$\bullet$};
	\node[draw, inner ysep=4pt, fit={(Ya) ($(Yb.north)+(0,1ex)$) (Ye)}] (Y) {};
	\node[left=0 of Y] (Ylab) {$Y\coloneqq$};
%
  \node[above=4 of Ya, "$a_1$"] (Xa1) {$\bullet$};
  \node[above=1 of Xa1, "$a_2$"] (Xa2) {$\bullet$};
  \node[above=4 of Yb, "$b_1$"] (Xb1) {$\bullet$};
  \node[above=1 of Xb1, "$b_2$"] (Xb2) {$\bullet$};
  \node[above=1 of Xb2, "$b_3$"] (Xb3) {$\bullet$};
  \node[above=4 of Yc, "$c_1$"] (Xc1) {$\bullet$};
  \node[above=4 of Ye, "$e_1$"] (Xe1) {$\bullet$};
  \node[above=1 of Xe1, "$e_2$"] (Xe2) {$\bullet$};
	\node[draw, inner ysep=3pt, fit={(Xa2) ($(Xb3.north)+(0,1ex)$) (Xe1)}] (X) {};
	\node[left=0 of X] {$X\coloneqq$};
%
	\draw[->, shorten <=3pt, shorten >=3pt] (X) to node[left] (f) {$f$} (Y);
\end{tikzpicture}
\end{equation}
\begin{enumerate}
	\item The fiber of $f$ over $a$ is $\{a_1,a_2\}$.
	\item The fiber of $f$ over $c$ is $\{c_1\}$.
	\item The fiber of $f$ over $d$ is $\varnothing$.
	\item A function $f'\colon X\to Y$ for which every fiber has either one or two elements is shown below.
\end{enumerate}
\[
\begin{tikzpicture}[y=.35cm, every label quotes/.style={font=\scriptsize, above, label distance=-5pt}, baseline=(f)]
	\node["$a$"] (Ya) {$\bullet$};
	\node[right=1 of Ya,  "$b$"] (Yb) {$\bullet$};
	\node[right=1 of Yb,  "$c$"] (Yc) {$\bullet$};
	\node[right=1 of Yc,  "$d$"] (Yd) {$\bullet$};
	\node[right=1 of Yd,  "$e$"] (Ye) {$\bullet$};
	\node[draw, inner ysep=4pt, fit={(Ya) ($(Yb.north)+(0,1ex)$) (Ye)}] (Y) {};
	\node[left=0 of Y] (Ylab) {$Y\coloneqq$};
%
  \node[above=4 of Ya, "$e_1$"] (Xa1) {$\bullet$};
  \node[above=1 of Xa1, "$a_2$"] (Xa2) {$\bullet$};
  \node[above=4 of Yb, "$b_1$"] (Xb1) {$\bullet$};
  \node[above=1 of Xb1, "$c_1$"] (Xb2) {$\bullet$};
  \node[above=4 of Yc, "$b_2$"] (Xc1) {$\bullet$};
  \node[above=4 of Yd, "$b_3$"] (Xe1) {$\bullet$};
  \node[above=4 of Ye, "$a_1$"] (Xe1) {$\bullet$};
  \node[above=1 of Xe1, "$e_2$"] (Xe2) {$\bullet$};
	\node[draw, inner ysep=3pt, fit={(Xa2) ($(Xb3.north)+(0,1ex)$) (Xe1)}] (X) {};
	\node[left=0 of X] {$X\coloneqq$};
%
	\draw[->, shorten <=3pt, shorten >=3pt] (X) to node[left] (f) {$f$} (Y);
\end{tikzpicture}
\]
}

\sol{exc.presheaf_ex_cont}{
Refer to \cref{eqn.sections_of_function}.
\begin{enumerate}
	\item Let $V_1=\{a,b,c\}$. Draw all the sections over it, i.e.\ all elements of $\Fun{Sec}_f(V_1)$, as we did in \cref{eqn.all_six_sections}.
	\item Let $V_2=\{a,b,c,d\}$. Again draw all the sections, $\Fun{Sec}_f(V_2)$.
	\item Let $V_3=\{a,b,d,e\}$. How many sections (elements of $\Fun{Sec}_f(V_3)$) are there?
\end{enumerate}
}{
Refer to \cref{eqn.sections_of_function_redraw}.
\begin{enumerate}
	\item Here is a drawing of all six sections over $V_1=\{a,b,c\}$:
	\[
  	\begin{tikzpicture}[y=.25cm, x=.1cm, every label/.style={font=\scriptsize}, trim left=2cm]
  	\node (Ya6) {};
  	\foreach \i [remember=\i as \lasti (initially 6)] in {0,...,5} {
    	\node[label={[below=6.5pt]:$a$}, right=17 of Ya\lasti]   (Ya\i)  {$\bullet$};
    	\node[label={[below=5pt]:$b$}, right=1 of Ya\i]  (Yb\i)  {$\bullet$};
    	\node[label={[below=6.5pt]:$c$}, right=1 of Yb\i]  (Yc\i)  {$\bullet$};
      \node[above=4 of Ya\i]  (Xa1\i) {$\bullet$};
      \node[above=1 of Xa1\i] (Xa2\i) {$\bullet$};
      \node[above=4 of Yb\i]  (Xb1\i) {$\bullet$};
      \node[above=1 of Xb1\i] (Xb2\i) {$\bullet$};
      \node[above=1 of Xb2\i] (Xb3\i) {$\bullet$};
      \node[above=4 of Yc\i]  (Xc1\i) {$\bullet$};
      \node[draw, fit=(Ya\i) (Yb\i) (Yc\i)] (Y\i) {};
      \node[draw, fit=(Xa1\i) (Xb3\i) (Xc1\i)] (X\i) {};
  		\draw[->, shorten <=3pt, shorten >=3pt] (X\i) -- (Y\i);
  		\tikzmath{
    		int \a, \b, \ii;
    		\a = 1+div(\i, 3);
    		\b = 1+mod(\i,3);
  			\ii=1+\i;
  		}
      \node[above=0 of X\i] {$g_{\ii}$};
  		\begin{scope}[|->, dashed, thick, blue, shorten >=-2pt]
    		\draw[ bend left=10pt] (Ya\i) to (Xa\a\i);
    		\draw[bend right=10pt] (Yb\i) to (Xb\b\i);
				\draw[bend right=10pt] (Yc\i) to (Xc1\i);
  			\node[circle, dotted, draw] at (Xa\a\i) {};
  			\node[circle, dotted, draw] at (Xb\b\i) {};
  			\node[circle, dotted, draw] at (Xc1\i) {};
  		\end{scope}
  	}
  \end{tikzpicture}
\]
	\item When $V_2=\{a,b,c,d\}$, there are no sections: $\Fun{Sec}_f(V_2)=\varnothing$.
	\item When $V_3=\{a,b,d,e\}$, the set $\Fun{Sec}_f(V_3)$) has $2*3*1*2=12$ elements.
\end{enumerate}
}

\sol{exc.section_practice}{
\begin{enumerate}
	\item Write out the sets of sections $\Fun{Sec}_f(\{a,b,c\})$ and $\Fun{Sec}_f(\{a,c\})$.
	\item Draw lines from the first to the second to indicate the restriction map.
\end{enumerate}
}{
$\Fun{Sec}_f(\{a,b,c\})$ and $\Fun{Sec}_f(\{a,c\})$ are drawn as the top row (six-element set) and bottom row (two-element set) below, and the restriction map is also shown:
\[
\begin{tikzcd}[ampersand replacement=\&, column sep=8pt]
	(a_1,b_1,c_1)\ar[dr, |->]\&
	(a_1,b_2,c_1)\ar[d, |->]\&
	(a_1,b_3,c_1)\ar[dl, |->]\&
	(a_2,b_1,c_1)\ar[dr, |->]\&
	(a_2,b_2,c_1)\ar[d, |->]\&
	(a_2,b_3,c_1)\ar[dl, |->]\\
	\&(a_1,c_1)\&\&\&
	(a_2,c_1)
\end{tikzcd}
\]
}

\sol{exc.sections_agree_overlap}{
Again let $U_1=\{a,b\}$ and $U_2=\{b,e\}$, so the overlap is $U_1\cap U_2=\{b\}$.
\begin{enumerate}
	\item Find a section $g_1\in\Fun{Sec}_f(U_1)$ and a section $g_2\in\Fun{Sec}_f(U_2)$ that \emph{do not} agree on the overlap.
	\item For your answer ($g_1,g_2)$ in part 1, can you find a section $g\in\Fun{Sec}_f(U_1\cup U_2)$ such that $\restrict{g}{U_1}=g_1$ and $\restrict{g}{U_2}=g_2$?
	\item Find a section $h_1\in\Fun{Sec}_f(U_1)$ and a section $h_2\in\Fun{Sec}_f(U_2)$ that \emph{do} agree on the overlap, but which are different than our choice in \cref{eqn.one_section}.
	\item Can you find a section $h\in\Fun{Sec}_f(U_1\cup U_2)$ such that $\restrict{h}{U_1}=h_1$ and $\restrict{h}{U_2}=h_2$?
\end{enumerate}
}{
\begin{enumerate}
	\item Let $g_1\coloneqq(a_1,b_1)$ and $g_2\coloneqq(b_2,e_1)$; these do not agree on the overlap.
	\item No, there's no section $g\in\Fun{Sec}_f(U_1\cup U_2)$ for which $\restrict{g}{U_1}=g_1$ and $\restrict{g}{U_2}=g_2$.
\end{enumerate}
\[
	\begin{tikzpicture}[y=.4cm, x=.4cm, every label quotes/.style={font=\scriptsize, above, label distance=-5pt}, baseline=(f)]
  	\node["$a$"] (Ya)  {$\bullet$};
  	\node[right=1 of Ya, red, "$b$" red]  (Yb)  {$\bullet$};
    \node[above=4 of Ya,  "$a_1$"]  (Xa1) {$\bullet$};
    \node[above=1 of Xa1, "$a_2$"] (Xa2) {$\bullet$};
    \node[above=4 of Yb,  "$b_1$"]  (Xb1) {$\bullet$};
    \node[above=1 of Xb1, "$b_2$"] (Xb2) {$\bullet$};
    \node[above=1 of Xb2, "$b_3$"] (Xb3) {$\bullet$};
    \node[draw, fit={(Ya) ($(Yb.north east)+(0,1)$)}] (Y) {};
    \node[draw, fit={(Xa1) ($(Xb3.north east)+(0,1)$)}] (X) {};
		\draw[->, shorten <=3pt, shorten >=3pt] (X) to node (f) {} (Y);
    \node[above=0 of X] {$h_1$};
		\begin{scope}[|->, dashed, thick, blue, shorten <=2pt, shorten >=-2pt]
  		\draw[bend left] (Ya) to (Xa2);
  		\draw[red, bend right] (Yb) to (Xb3);
			\node[circle, dotted, draw] at (Xa2) {};
			\node[red, circle, dotted, draw] at (Xb3) {};
		\end{scope}
		\node[right=5 of Yb, red, "$b$" red]   (Ybb)  {$\bullet$}; 
  	\node[right=1 of Ybb,  "$e$"]   (Yee)  {$\bullet$};
    \node[above=4 of Ybb,  "$b_1$"] (Xbb1) {$\bullet$};
    \node[above=1 of Xbb1, "$b_2$"] (Xbb2) {$\bullet$};
    \node[above=1 of Xbb2, "$b_3$"] (Xbb3) {$\bullet$};
    \node[above=4 of Yee,  "$e_1$"] (Xee1) {$\bullet$};
    \node[above=1 of Xee1, "$e_2$"] (Xee2) {$\bullet$};
    \node[draw, fit={(Ybb) ($(Yee.north east)+(0,1)$)}] (YY) {};
    \node[draw, fit={(Xee1) ($(Xbb3.north west)+(0,1)$)}] (XX) {};
		\draw[->, shorten <=3pt, shorten >=3pt] (XX) -- (YY);
    \node[above=0 of XX] {$h_2$};
		\begin{scope}[|->, dashed, thick, blue, shorten <=2pt, shorten >=-2pt]
  		\draw[red, bend left] (Ybb) to (Xbb3);
  		\draw[green!50!black, bend right] (Yee) to (Xee1);
			\node[red, circle, dotted, draw] at (Xbb3) {};
			\node[green!50!black, circle, dotted, draw] at (Xee1) {};
		\end{scope}		
		\node[above left=0 and 1 of Y.south west] {3.};
%
  	\node[right=12 of Ybb, "$a$"] (Ya)  {$\bullet$};
  	\node[right=1 of Ya, red, "$b$" red]  (Yb)  {$\bullet$};
  	\node[right=1 of Yb,  "$e$"]   (Yee)  {$\bullet$};
    \node[above=4 of Ya,  "$a_1$"]  (Xa1) {$\bullet$};
    \node[above=1 of Xa1, "$a_2$"] (Xa2) {$\bullet$};
    \node[above=4 of Yb,  "$b_1$"]  (Xb1) {$\bullet$};
    \node[above=1 of Xb1, "$b_2$"] (Xb2) {$\bullet$};
    \node[above=1 of Xb2, "$b_3$"] (Xb3) {$\bullet$};
    \node[above=4 of Yee,  "$e_1$"] (Xee1) {$\bullet$};
    \node[above=1 of Xee1, "$e_2$"] (Xee2) {$\bullet$};
    \node[draw, fit={(Ya) ($(Yb.north east)+(0,1)$) (Yee)}] (Y) {};
    \node[draw, fit={(Xa1) ($(Xb3.north east)+(0,1)$) (Xee1)}] (X) {};
		\draw[->, shorten <=3pt, shorten >=3pt] (X) -- (Y);
    \node[above=0 of X] {glued section};
		\begin{scope}[|->, dashed, thick, blue, shorten <=2pt, shorten >=-2pt]
  		\draw[bend left] (Ya) to (Xa2);
  		\draw[red, bend right] (Yb) to (Xb3);
	 		\draw[bend right, green!50!black] (Yee) to (Xee1);
			\node[circle, dotted, draw] at (Xa2) {};
			\node[circle, red, dotted, draw] at (Xb3) {};
			\node[circle, green!50!black, dotted, draw] at (Xee1) {};
		\end{scope}
		\node[above left=0 and 1 of Y.south west] {4.};
	\end{tikzpicture}
\]
}

\sol{exc.whats_the_sheaf}{
If $M$ is a sphere as in \cref{ex.tangent_bundle}, we know from \cref{def.sheaf} that we can consider the category $\Shv(M)$ of sheaves on $M$; in fact, such categories are toposes and these are what we're getting to.

But are the sheaves on $M$ the vector fields? That is, is there a one-to-one correspondence between sheaves on $M$ and vector fields on $M$? If so, why? If not, how are sheaves on $M$ and vector fields on $M$ related?
}
{
No, there is not a one-to-one correspondence between sheaves on $M$ and vector fields on $M$. The relationship between sheaves on $M$ and vector fields on $M$ is that the \emph{set of all} vector fields on $M$ corresponds to \emph{one} sheaf, namely $\Fun{Sec}_\pi$, where $\pi\colon TM\to M$ is the tangent bundle as described in \cref{ex.tangent_bundle}. There are so many sheaves on $M$ that they don't even form a set (it's just a `collection'); again, one member of this gigantic collection is the sheaf $\Fun{Sec}_\pi$ of all possible vector fields on $M$.
}

\sol{exc.sierpinski}{
Consider the Sierpinski space $(\{1,2\},\Op_1)$ from \cref{ex.Sierpinski}.
\begin{enumerate}
	\item What is the category $\Op$ for this space? (You may have already figured this out in \cref{exc.opens_Sierp}; if not, do so now.)
	\item What does a presheaf on $\Op$ consist of?
	\item What is the sheaf condition for $\Op$?
	\item How do we identify a sheaf on $\Op$ with a function?
\end{enumerate}
}{
\begin{enumerate}
	\item The Hasse diagram for the Sierpinsky topology is \fbox{$\varnothing\to\{1\}\to\{1,2\}$}\,.
	\item A presheaf $F$ on $\Op$ consists of any three sets and any two functions $F(\{1,2\})\to F(\{1\})\to F(\varnothing)$ between them.
	\item Recall from \cref{exc.opens_Sierp} that the only non-trivial covering (a covering of $U$ is \emph{non-trivial} if it does not contain $U$) occurs when $U=\varnothing$ in which case the empty family over $U$ is a cover.
	\item As explained in \cref{ex.empty_cover}, $F$ will be a sheaf iff $F(\varnothing)\cong\{1\}$. Thus we the category of sheaves is equivalent to that of just two sets and one function $F(\{1,2\})\to F(\{1\})$.
\end{enumerate}
}

\sol{exc.booleans_as_subspace_1}{
Let $X=\Cat{1}$ be the one point space. We said above that its subobject classifier is the set $\bb$ of booleans, but how does that align with the definition of $\Omega$ given in \cref{eqn.omega_as_opens}?
}
{
The one-point space $X=\{1\}$ has two open sets, $\varnothing$ and $\{1\}$, and every sheaf $S\in\Shv(X)$ assigns $S(\varnothing)=\{()\}$ by the sheaf condition (see \cref{ex.empty_cover}). So the only data in a sheaf $S\in\Shv(X)$ is the set $S(\{1\})$. This is how we get the correspondence between sets and sheaves on the one point space.

According to \cref{eqn.omega_as_opens}, the subobject classifier $\Omega\colon\Op(X)\op\to\smset$ in $\Shv(X)$ should be the functor where $\Omega(\{1\})$ is the set of open sets of $\{1\}$. So we're hoping to see that there is a one-to-one correspondence between the set $\Op(\{1\})$ and the set $\bb=\{\true,\false\}$ of booleans. Indeed there is: there are two open sets of $\{1\}$, as we said, $\varnothing$ and $\{1\}$, and these correspond to $\false$ and $\true$ respectively.
}

\sol{exc.Omega_functorial}{
\begin{enumerate}
	\item Show that the definition of $\Omega$ given above in \cref{eqn.omega_as_opens,eqn.omega_open_rest} is functorial, i.e., that whenever $W\ss V\ss U$, the restriction map $\Omega(U)\to\Omega(V)$ followed by the restriction map $\Omega(V)\to\Omega(W)$ is the same as the restriction map $\Omega(U)\to\Omega(W)$.
	\item Is that all that's necessary to conclude that $\Omega$ is a presheaf?
\end{enumerate}
}{
By \cref{eqn.omega_as_opens,eqn.omega_open_rest} the definition of $\Omega(U)$ is $\Omega(U)\coloneqq\{U'\in\Op\mid U'\ss U\}$, and the definition of the restriction map for $V\ss U$ is $U'\mapsto U'\cap V$.
\begin{enumerate}
	\item It is functorial: given $W\ss V\ss U$ and $U'\ss U$, we indeed have $(U'\cap V)\cap W=U'\cap W$, since $W\ss V$. For functoriality, we also need preservation of identities, and this amounts to $U'\cap U=U'$ for all $U'\ss U$.
	\item	Yes, a presheaf is just a functor; the above check is enough.
\end{enumerate}
}

\sol{exc.classify_subgraph}{
Consider the subgraph $G'\ss G$ shown here:
\[
\boxCD{
\begin{tikzcd}[ampersand replacement=\&]
	\LMO{A}\ar[r]\&\LMO{B}\&\LMO{C}
\end{tikzcd}
}
\quad\ss\quad
\boxCD{
\begin{tikzcd}[ampersand replacement=\&]
	\LMO{A}\ar[r, shift left, "f"]\&\LMO{B}\ar[l, shift left, "g"]\ar[r, "h"]\&\LMO{C}\ar[r, "i"]\&\LMO{D}
\end{tikzcd}
}
\]
Find the graph homomorphism $\corners{G'}\colon G\to\Omega$ classifying it. See \cref{ex.subobject_classifier_graphs}.
}{
We need a graph homomorphism of the following form:
\[
\boxCD{
\begin{tikzcd}[ampersand replacement=\&]
	\LMO{A}\ar[r, shift left, "f"]\&\LMO{B}\ar[l, shift left, "g"]\ar[r, "h"]\&\LMO{C}\ar[r, "i"]\&\LMO{D}
\end{tikzcd}
}
\To{\quad\corners{G'}\quad}
\boxCD{
\begin{tikzcd}[column sep=70pt, ampersand replacement=\&]
	0
		\ar[loop left, "{(0,0;\ 0)}"]
		\ar[r, bend left=15pt, "{(0, V;\ 0)}"]
\&
	V
		\ar[loop above, "{(V, V;\ 0)}"]
		\ar[loop below, "{(V, V;\ A)}"]
		\ar[l, bend left=15pt, "{(V, 0;\ 0)}"]
\end{tikzcd}
}
\]
There is only one that classifies $G'$, and here it is. Let's write $\gamma\coloneqq\corners{G'}$.
\begin{itemize}
	\item Since $D$ is missing from $G'$, we have $\gamma(D)=0$ (vertex: missing).
	\item Since vertices $A,B,C$ are present in $G'$ we have $\gamma(A)=\gamma(B)=\gamma(C)=V$ (vertex: present). 
	\item The above forces $\gamma(i)=(V,0;0)$ (arrow from present vertex to missing vertex: missing).
	\item Since the arrow $f$ is in $G'$, we have $\gamma(f)=(V,V; A)$ (arrow from present vertex to present vertex: present).
	\item Since the arrows $g$ and $h$ are missing in $G'$, we have $\gamma(g)=\gamma(h)=(V,V; 0)$ (arrow from present vertex to present vertex: missing).
\end{itemize}
}


\sol{exc.real_line_logic}{
Consider the real line $\RR$ as a topological space, and consider the open subset $U=\RR-\{0\}$.
\begin{enumerate}
	\item What open subset is $\neg U$?
	\item What open subset is $\neg\neg U$?
	\item Is it true that $U\ss\neg\neg U$?
	\item Is it true that $\neg\neg U\ss U$?
\end{enumerate}	
}{
With $U=\RR-\{0\}\ss\RR$, we have:
\begin{enumerate}
	\item The complement of $U$ is $\RR-U=\{0\}$ and $\neg U$ is its interior, which is $\neg U=\varnothing$.
	\item The complement of $\neg U$ is $\RR-\varnothing=\RR$, and this is open, so $\neg\neg U=\RR$.
	\item It is true that $U\ss\neg\neg U$.
	\item It is false that $\neg\neg U\ss^?U$.
\end{enumerate}
}

\sol{exc.top_bot_practice}{
Let $(X,\Op)$ be a topological space.
\begin{enumerate}
	\item Suppose the symbol $\top$ corresponds to an open set such that for any open set $V\in\Op$, we have $(\top\wedge V)=V$. Which open set is it?
	\item Other things we should expect from $\top$ include $(\top\vee V)=\top$ and $(V\imp \top)=\top)$ and $(\top\imp V)=V$. Do these hold for your answer to 1?
	\item The symbol $\bot$ corresponds to an open set $U\in\Op$ such that for any open set $V\in\Op$, we have $(\bot\vee V)=V$. Which open set is it?
	\item Other things we should expect from $\bot$ include $(\bot\wedge V)=\bot$ and $(\bot\imp V)=\top)$. Do these hold for your answer to 1?
\end{enumerate} 
}{
\begin{enumerate}
	\item If for any $V\in\Op$ we have $\top\wedge V=V$ then when $V=X$ we have $\top\wedge X\coloneqq\top\cap X=X$, but anything intersected with $X$ is itself, so $\top=\top\cap X=X$.
	\item $(\top\vee V)\coloneqq (X\cup V)=X$ holds and $(V\imp X)=\bigcup_{\{R\in\Op\mid R\cap V\ss X\}}R=X$ holds because $(X\cap V)\ss X$.
	\item If for any set $V\in\Op$ we have $(\bot\vee V)=V$, then when $V=\emptyset$ we have $(\bot\vee\varnothing)=(\bot\cup\varnothing)=\varnothing$, but anything unioned with $\varnothing$ is itself, so $\bot=\bot\cup\varnothing=\varnothing$.
	\item $(\bot\wedge V)=(\varnothing\cap V)=\varnothing$ holds, and $(\bot\imp V)=\bigcup_{\{R\in\Op\mid R\cap\varnothing\ss V\}}R=X$ holds because $(X\cap\varnothing)\ss V$.
\end{enumerate}
}

\sol{exc.weather_bob}{
Just now we described how a predicate $p\colon S\to\Omega$, such as ``\dots likes the weather,'' acts on sections $s\in S(U)$, say $s=\mathrm{Bob}$. But by \cref{def.subobject_classifier}, any predicate $p\colon S\to\Omega$ to the subobject classifier also defines a subobject of $\{S\mid p\}\ss S$. Describe the sections of this subsheaf.
}{
$S$ is the sheaf of people, the set of which changes over time: a section in $S$ over any interval of time is a person who is alive throughout that interval. A section in the subobject $\{S\mid p\}$ over  any interval of time is a person who is alive \emph{and likes the weather} throughout that interval of time.
}

\sol{exc.vdash}{
Give an example of a space $X$, a sheaf $S\in\Shv(X)$, and two predicates $p,q\colon S\to\Omega$ for which $p(s)\vdash_{s:S} q(s)$ holds. You do not have to be formal.
}
{
We need an example of a space $X$, a sheaf $S\in\Shv(X)$, and two predicates $p,q\colon S\to\Omega$ for which $p(s)\vdash_{s:S} q(s)$ holds. Take $X$ to be the one-point space, take $S$ to be the sheaf corresponding to the set $S=\nn$, let $p(s)$ be the predicate ``$24\leq s\leq 28$,'' and let $q(s)$ be the predicate ``$s$ is not prime.'' Then $p(s)\vdash_{s:S} q(s)$ holds.\\

As an informal example, take $X$ to be the surface of the earth, take $S$ to be the sheaf of vector fields as in \cref{ex.tangent_bundle} thought of in terms of wind-blowing. Let $p$ be the predicate ``the wind is blowing due east at somewhere between 2 and 5 kilometers per hour'' and let $q$ be the predicate ``the wind is blowing at somewhere between 1 and 5 kilometers per hour.'' Then $p(s)\vdash_{s:S} q(s)$ holds. This means that for any open set $U$, if the wind is blowing due east at somewhere between 2 and 5 kilometers per hour throughout $U$, then the wind is blowing at somewhere between 1 and 5 kilometers per hour throughout $U$ as well.
}
\sol{exc.predicate_practice}{
In the topos $\Cat{Set}$, where $\Omega=\BB$, consider the predicate $p\colon\NN\times\ZZ\to\BB$ given by
\[
  p(n,z)=
  \begin{cases}
    \true&\tn{ if }n\leq|z|\\
    \false&\tn{ if }n>|z|.
  \end{cases}
\]
\begin{enumerate}
	\item What is the set of $n\in\NN$ for which the predicate $\forall(z:\ZZ)\ldotp p(n,z)$ holds?
	\item What is the set of $n\in\NN$ for which the predicate $\exists(z:\ZZ)\ldotp p(n,z)$ holds?
	\item What is the set of $z\in\ZZ$ for which the predicate $\forall(n:\NN)\ldotp p(n,z)$ holds?
	\item What is the set of $z\in\ZZ$ for which the predicate $\exists(n:\NN)\ldotp p(n,z)$ holds?	
\end{enumerate}
}{
We have the predicate $p\colon\NN\times\ZZ\to\BB$ given by $p(n,z)$ iff $n\leq|z|$.
\begin{enumerate}
	\item The predicate $\forall(z:\ZZ)\ldotp p(n,z)$ holds for $\{0\}\ss\NN$.
	\item The predicate $\exists(z:\ZZ)\ldotp p(n,z)$ holds for $\NN\ss\NN$.
	\item The predicate $\forall(n:\NN)\ldotp p(n,z)$ holds for $\varnothing\ss\ZZ$.
	\item The predicate $\exists(n:\NN)\ldotp p(n,z)$ holds for $\ZZ\ss\ZZ$.
\end{enumerate}
}

\sol{exc.worrying_news_universal}{
Suppose $s$ is a person alive throughout the interval $U$. Apply the above definition to the example $p(s,t)=$ ``person $s$ is worried about news $t$'' from above. Here, $T(V)$ is the set of items that are in the news throughout the interval $V$.
\begin{enumerate}
  \item What open subset of $U$ is $\forall(t:T)\ldotp p(s,t)$ for a person $s$?
  \item Does it have the semantic meaning you'd expect, given the less formal description in \cref{subsec.quantification}?
\end{enumerate}
}{
Suppose $s$ is a person alive throughout the interval $U$. Apply the above definition to the example $p(s,t)=$ ``person $s$ is worried about news $t$'' from above.
\begin{enumerate}
	\item The formula says that $\forall(t:T)\ldotp p(s,t)$ ``returns the largest open set $V\ss U$ for which $p(\restrict{s}{V},t)=V$ for all $t\in T(V)$.'' Note that $T(V)$ is the set of items that are in the news throughout the interval $V$. Substituting, this becomes ``the largest interval of time $V\ss U$ over which person $s$ is worried about news $t$ for every item $t$ that is in the news throughout $V$.'' In other words, for $V$ to be nonempty, the person $s$ would have to be worried about \emph{every single item of news} throughout $V$. My guess is that there's a festival happening or a happy kitten somewhere that person $s$ is not worried about, but maybe I'm assuming that person $s$ is sufficiently mentally ``normal.'' There may be people who are sometimes worried about literally everything in the news; we ask you to please be kind to them. 
	\item Yes, it is exactly the same description.
\end{enumerate}
}


\sol{exc.worrying_news_existential}{
Apply the above definition to the ``person $s$ is worried about news $t$'' example above. \begin{enumerate}
  \item What open set is $\exists(t:T)\ldotp p(s,t)$ for a person $s$?
  \item Does it have the semantic meaning you'd expect?
\end{enumerate}
}{
Suppose $s$ is a person alive throughout the interval $U$. Apply the above definition to the example $p(s,t)=$ ``person $s$ is worried about news $t$'' from above.
\begin{enumerate}
	\item The formula says that $\exists(t:T)\ldotp p(s,t)$ ``returns the union $V=\bigcup_iV_i$ of all the open sets $V_i$ for which there exists some $t_i\in T(V_i)$ satisfying $p(\restrict{s}{V_i},t_i)=V_i$.'' Substituting, this becomes ``the union of all time intervals $V_i$ for which there is some item $t_i$ in the news about which $s$ is worried throughout $V_i$.'' In other words it is all the time that $s$ is worried about at least one thing in the news. Perhaps when $s$ is sleeping or concentrating on something, she is not worried about anything, in which case intervals of sleeping or concentrating would not be subsets of $V$. But if $s$ said ``there's been such a string of bad news this past year, it's like I'm always worried about something!,'' she is saying that it's like $V=$``this past year.''
	\item This seems like a good thing for ``there exists a piece of news that worries $s$'' to mean: the news itself is allowed to change as long as the person's worry remains. Someone might disagree and think that the predicate should mean ``there is one piece of news that worries $s$ throughout the whole interval $V$.'' In that case, perhaps this person is working within a different topos, e.g.\ one where the site has fewer coverings. Indeed, it is the notion of covering that makes existential quantification work the way it does.
\end{enumerate}
}

\sol{exc.two_defs_of_closure}{
Suppose $j\colon\Omega\to\Omega$ is a morphism of sheaves on $X$, such that $p\leq j(p)$ holds for all $U\ss X$ and $p\in\Omega(U)$. Show that for all $q\in\Omega(U)$ we have $j(j(q))\leq j(q)$ iff $j(j(q))=j(q)$.
}
{
It is clear that if $j(j(q))=j(q)$ then $j(j(q))\leq j(q)$ by reflexivity. On the other hand, assume the hypothesis, that $p\leq j(p)$ for all $U\ss X$ and $p\in\Omega(U)$. If $j(j(q))\leq j(q)$, then letting $p\coloneqq j(q)$ we have both $j(p)\leq p$ and $p\leq j(p)$. This means $p\cong j(p)$, but $\Omega$ is a poset (not just a preorder) so $p=j(p)$, i.e.\ $j(j(q))= j(q)$ as desired.
}

\sol{exc.check_modality}{
Let $S$ be the sheaf of people as in \cref{subsec.predicates}, and let $j\colon\Omega\to\Omega$ be ``assuming Bob is in San Diego...'' 
\begin{enumerate}
	\item Name any predicate $p\colon S\to\Omega$, such as ``likes the weather.''
	\item Choose a time interval $U$. For an arbitrary person $s\in S(U)$, what sort of thing is $p(s)$, and what does it mean?
	\item What sort of thing is $j(p(s))$ and what does it mean?
	\item Is it true that $p(s)\leq j(p(s))$? Explain briefly.
	\item Is it true that $j(j(p(s))=j(p(s)$? Explain briefly.
	\item Choose another predicate $q\colon S\to\Omega$. Is it true that $j(p\wedge q)=j(p)\wedge j(q)$? Explain briefly.
\end{enumerate}
}
{
Let $S$ be the sheaf of people and $j$ be ``assuming Bob is in San Diego...''
\begin{enumerate}
	\item Take $p(s)$ to be ``$s$ likes the weather.''
	\item Let $U$ be the interval 2019/01/01 -- 2019/02/01. For an arbitrary person $s\in S(U)$, $p(s)$ is a subset of $U$, and it means the subset of $U$ throughout which $s$ likes the weather.
	\item Similarly $j(p(s))$ is a subset of $U$, and it means the subset of $U$ throughout which, assuming Bob is in San Diego, $s$ liked the weather. In other words, $j(p(s))$ is true whenever Bob is not in San Diego, and it is true whenever $s$ likes the weather.
	\item It is true that $p(s)\leq j(p(s))$, by the `in other words' above.
	\item It is true that $j(j(p(s))=j(p(s)$, because suppose given a time during which ``if Bob is in San Diego then if Bob is in San Diego then $s$ likes the weather.'' Then if Bob is in San Diego during this time then $s$ likes the weather. But that is exactly what $j(p(s))$ means.
	\item Take $q(s)$ to be ``$s$ is happy.'' Suppose ``if Bob is in San Diego then both $s$ likes the weather and $s$ is happy.'' Then both ``if Bob is in San Diego then $s$ likes the weather'' and ``if Bob is in San Diego then $s$ is happy'' are true too. The converse is equally clear.
\end{enumerate}
}


\sol{exc.explain_intervals}{
\begin{enumerate}
	\item Explain why $[2,6]\in o_{[0,8]}$.
	\item Explain why $[2,6]\not\in o_{[0,5]}\cup o_{[4,8]}$.
\end{enumerate}
}{
We have $o_{[a,b]}\coloneqq\{[d,u]\in\IR\mid a<d\leq u<b\}$.
\begin{enumerate}
	\item Since $0\leq 2\leq 6\leq 8$, we have $[2,6]\in o_{[0,8]}$ by the above formula.
	\item In order to have $[2,6]\in^? o_{[0,5]}\cup o_{[4,8]}$, we would need to have either $[2,6]\in^? o_{[0,5]}$ or $[2,6]\in^? o_{[4,8]}$. But the formula does not hold in either case.
\end{enumerate}
}

\sol{exc.R_subsp_IR}{
Show that a subset $U\ss\RR$ is open in the subspace topology of $\RR\ss\IR$ iff $U\cap\RR$ is open in the usual topology on $\RR$ defined in \cref{ex.usual_R}.
}{
A subset $U\ss\RR$ is open in the subspace topology of $\RR\ss\IR$ iff there is an open set $U'\ss\IR$ with $U=U'\cap\RR$. We want to show that this is the case iff $U$ is open in the usual topology.

Suppose that $U$ is open in the subspace topology. Then $U=U'\cap\RR$, where $U'\ss\IR$ is the union of some basic opens, $U'=\bigcup_{i\in I}o_{[a_i,b_i]}$, where $o_{[a_i,b_i]}=\{[d,u]\in\IR\mid a_i<d<u<b_i\}$. Since $\RR=\{[x,x]\in\IR\}$, the intersection $U'\cap\RR$ will then be
\[U=\bigcup_{i\in I}\{x\in\RR\mid a_i<x<b_i\}\]
and this is just the union of open balls $B(m_i,r_i)$ where $m_i\coloneqq\frac{a_i+b_i}{2}$ is the midpoint and $r_i\coloneqq\frac{b_i-a_i}{2}$ is the radius of the interval $(a_i,b_i)$. The open balls $B(m_i,r_i)$ are open in the usual topology on $\RR$ and the union of opens is open, so $U$ is open in the usual topology.

Suppose that $U$ is open in the usual topology. Then $U=\bigcup_{j\in J}B(m_j,\epsilon_j)$ for some set $J$. Let $a_j\coloneqq m_j-\epsilon_j$ and $b_j\coloneqq m_j+\epsilon_j$. Then \[U=\bigcup_{j\in J}\{x\in\RR\mid a_j<x<b_j\}=\bigcup_{j\in J}(o_{[a_j,b_j]}\cap\RR)=\left(\bigcup_{j\in J}o_{[a_j,b_j]}\right)\cap\RR\]
which is open in the subspace topology.
}


\sol{exc.interval_domain_top}{
Fix any topological space $(X,\Op_X)$ and any subset $R\ss\IR$ of the interval domain. Define $H_X(U)\coloneqq\{f\colon U\cap R\to X\mid f\text{ is continuous}\}$.
\begin{enumerate}
	\item Is $H_X$ a presheaf? If not, why not; if so, what are the restriction maps?
	\item Is $H_X$ a sheaf? Why or why not?
\end{enumerate}
}{
Fix any topological space $(X,\Op_X)$ and any subset $R\ss\IR$ of the interval domain. Define $H_X(U)\coloneqq\{f\colon U\cap R\to X\mid f\text{ is continuous}\}$.
\begin{enumerate}
	\item $H_X$ is a presheaf: given $V\ss U$ the restriction map sends the continuous function $f\colon U\cap R\to X$ to its restriction along the subset $V\cap R\ss U\cap R$.
	\item It is a sheaf: given any family $U_i$ of open sets with $U=\bigcup_iU_i$ and a continuous function $f_i\colon U_i\cap R\to X$ for each $i$, agreeing on overlaps, they can be glued together to give a continuous function on all of $U\cap R$, since $U\cap R=(\bigcup_iU_i)\cap R=\bigcup_i(U_i\cap R)$.
\end{enumerate}
}

\finishSolutionChapter

\end{document}
